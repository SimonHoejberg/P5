\section{Sensor Testing}
To find out which sensors are the best to use, this chapter will be used to test
the different sensors. The testing of the sensors will also be used to test how
precise they are.\nl

First the testing testing methodology will be explained, then this methodology
will be used to devise tests for the different sensors. 

\subsection{Testing Methodology}
The testing will be divided into to two parts, one part is testing the two
distance sensors and the second part is testing the camera. 

\subsection{Ultrasonic Sensor}
The test of the ultrasonic sensor was conduted using a tape measure and a wall
made of LEGO. The test then proceeded by taking measurements at specific
lengths \fix{Table with results}
{INPUT TABLE HERE}

\subsection{Infrared}
The test of the infrared sensor was not able to be conduted since there were
some issues with the different sensors. The first sensor did not work at all,
since it was an infrared sensor a camera was used to see wether the infrared LED
lid up. Since it did not light we concluded that the sensor was broken. \nl

When trying to test the other sensor, first it did not give a reading so the
same test was conducted using a camera to see if the sensor was broken. It was
concluded that the sensor was not broken. It was then discovered that it was a
problem with the software, where the address for the reading the sensor on the
I2C port was changed. This meant that it was not possible to get a proper
reading. The readings were either 0 or a reading above 60.000 depending on which
port was used. \nl

Another issue with using the infrared sensor is the range, the medium range
infrared sensor only has a range from 10 to 80 cm. Therefor, this sensor could
become a very limiting factor as to how far the target can be placed from the
cannon. 

\subsection{Camera tracking}%Will most likely need a better name