\chapter{Sensor Testing} 

\section{Sensor Testing} 
To find out which sensors are the optimal sensors for the project,
this chapter will be used to test the different sensors. The testing of the
sensors will also be used to find out how precise they are.\nl

First the testing methodology will be explained, then this methodology
will be used to devise tests for the different sensors. 

\subsection{Testing Methodology}
The testing will be divided into to two parts, one part is testing the two
distance sensors and the second part is testing the camera. 

\subsection{Ultrasonic Sensor}
The test of the ultrasonic sensor was divided into 4 different experiments, the
difference between the experiments are the size of the object being tracked and
whether the sensor is shielded or not. The different object sizes was tested
with a shielded sensor and a non shielded sensor. \nl

\graphL{Actual distance [cm]}{Measured distance
[cm]}{shieldedBoxStanding,shieldedBoxLandscape}{Box Standing, Box
Landscape}{Results with a shielded sensor and a box h.
14.2cm and w. 9cm}{shieldedTestBox}

\graphL{Actual distance [cm]}{Measured distance
[cm]}{unshieldedBoxStanding,unshieldedBoxLandscape}{Box Standing, Box
Landscape}{Results with a unshielded sensor and a box h.
14.2cm and w. 9cm}{unshieldedTestBox}

\graphL{Actual distance [cm]}{Measured distance
[cm]}{shieldedPaperStanding,shieldedPaperLandscape}{Paper Standing, Paper
Landscape}{Results with a shielded sensor and a A4 Paper}{shieldedTestPaper}

\graphL{Actual distance [cm]}{Measured distance
[cm]}{unshieldedPaperStanding,unshieldedPaperLandscape}{Paper Standing, Paper
Landscape}{Results with a unshielded sensor and a A4 Paper}{unshieldedTestPaper}

This sensor has a range of 0 to 255 cm and will not be too limiting factor
compared to how far the turret is able to shoot.\nl

As seen in \autoref{graph:shieldedTestBox}, \autoref{graph:unshieldedTestBox},
\autoref{graph:unshieldedTestPaper} and \autoref{graph:shieldedTestPaper} the
sensor was able to determine the distance with better precision when it is
shielded and the object was in the landscape position. \nl

It can then be concluded that the shielded sensor performed a lot better than
the unshielded, it can also be concluded that the placement of the object can
have a big impact. This is due to the fact that the sensor does not have a very
large field of view. Therefore, the object should be wider rather than taller. 

\subsection{Infrared}
The test of the infrared sensor was not able to be conducted since there were
some issues with the different sensors. The first sensor did not work at all.
Since it was an infrared sensor, a camera was used to see if the infrared
LED lid up. Since it did not light we concluded that the sensor was broken. \nl

When trying to test the other sensor, first it did not give a reading so the
same test was conducted using a camera to see if the sensor was broken. It was
concluded that the sensor was not broken. It was then discovered that it was a
problem with the software, where the address for the reading the sensor on the
I2C port was changed. This meant that it was not possible to get a proper
reading. The readings were either 0 or a reading above 60.000 depending on which
port was used. \nl

Another issue with using the infrared sensor is the range. The medium range
infrared sensor only has a range from 10 to 80 cm. Therefore this sensor could
become a very limiting factor in regards to how far the target can be placed
from the cannon. 

\subsection{Camera tracking}%Will most likely need a better name
Testing the camera will consist of several parts, these parts will heavily rely
on the technical specifications of the camera. Therefore the first part of this
section will be on the technical specifications of the camera and the needed
software.\nl

The camera runs at 30 fps and with a resolution of 88 x 144. This could
potentialy become an issue due to the fact that it is a very low resolution.
Although this is a could be a problem, it depends on how large the
object/objects being tracked are.
The camera also has the ability to track up to 8 different objects. \nl

The setup of the camera is quite simple and is done using a computer, to setup
which color it is suposed to track. The process of choosing the color is done
using some software that comes with the camera, here a colormap is made which
tells the preprocessor on the camera which colors to track. 
