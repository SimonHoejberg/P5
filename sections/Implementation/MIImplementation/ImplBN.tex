\chapter{Implementing a Belief Network}
This chapter will be used to describe how the belief network could be
implemented. There will also be a discussion of the pros and cons compared to
the current setup.

\section{Implementation}

We did not implement a belief network, therefor this section will be used to
show how it could be implemented. To implement the network there are two
options, one where it could be implemented on the NXT and the other where it is
implemented on a PC.

% Introduction: We didnt really implement the MI-part.
% \textbf{If} it was to be implemented we had two options:
%
% \begin{enumerate}
% \item Implement the belief network on the NXT.
% \item Run the MI-agent on a computer and establish communication between the brick and the computer.
% \end{enumerate}

\subsection{Option One}
One of the first issues with implementing the model on the NXT is the amount of
RAM for the computing the network. The primary limitation for the NXT is the
low amount of RAM, namely 64KB, see \autoref{PlatformC}. Although the flash
storage on the NXT is larger, at 256KB, it is still necessary to load the code
into the RAM, so the real limiting factor is still the amount of available RAM.
In order to implement the network on the NXT, it is necessary to reduce the
complexity of the model or else it wont be possible to implement it on the NXT.
Since the first network had a state space of $3^2*5^6 = 140625$ states, this
would theoretically need 560KB of RAM given that each state is 4 byte. The 4
bytes are assumed as the NXT uses 32 bit to represent floating points which is
the largest type used. However if we remove some of the states which are not
necessary according to the implementation, the state space could be reduced to
$3^1*5^4 = 1875$ which given a size of the state of each space at 4KB would need
7.5KB in the worst case. Given this state space it would be possible to
implement the belief network on the NXT.

\subsection{Option Two}
The second option involves using a computer for computing the
belief network. Using this approach would require the computation to be fully
automated in order to still consider the system embedded.\nl

This approach requires the following things to be setup
\begin{itemize}
\item The belief network modeled on the computer
\item Establish a serial connection between the NXT and the PC
\item Figure out what data structures should be sent and received
\end{itemize}

The first part was realized using Hugin Lite to model the network described
in \autoref{MID}. The model of the network is exported via Hugin as a file.
Using the Hugin API the logic for the network was constructed in C\#.\nl

In order to establish a serial connection between the NXT and the PC, there are
two alternatives. The NXT is capable of communicating with the computer either
via either USB or Bluetooth. The USB approach would require a cable connecting
the NXT to the computer.
This could complicate certain things like the horizontal rotation, since the
cable could potentially restrict the movement by adding more resistance.\nl

Using Bluetooth for communication would have been preferred, since there are
no cables to restrict the turret when it turns.
This approach would require a PC with Bluetooth connectivity. Once a connection
has been established, it uses the NXT built-in mailbox system for transferring
messages between the brick and the PC.\nl

Next step would be to figure out in what format the communication should occur.
The NXT should send the sensor input needed for evidence to the PC. Since our
sensor values are defined in the model in intervals, we need to convert the
sensor value to the corresponding interval in the node. This can be achieved two ways. Either
the NXT does the conversion and sends the modified value to the PC, or the NXT
sends the full sensor value and the conversion is done on the computer before
being fed as evidence to the model.\nl

The other part of the communication is the response the PC should generate. The
network generates a vector that needs to be sent back to the NXT, to be used for
calculating a firing position. The idea was to use the NXT mailbox system's
built in \texttt{SendString}, so we send the information as strings and then
construct functions to package and unwrap the information.

% Our idea for implementing the network is using a computer running a program
% written in C\# which takes an input from the NXT, this input would be gotten
% via serial communication. This communication should also be used for
% communicating the results from the belief network. In the suggested
% implementation Hugin API will be doing the calculations and the program will act
% as an interface between the calculations made by Hugin and the NXT. \nl

% Another way of potentially implementing the network, would be to implement it
% directly on the NXT. This however presents a new set of issues, the first
% issue being the runtime of the calculations. This is an issue since the NXT is
% not a very powerful platform and doing complex calculations might take a long
% time. The second issue is how much RAM and flash memory the NXT has. Because of these
% issues the network would have to be very simplified and thus not as effective
