\chapter{Implementation of a Belief Network}
This chapter will be used to describe how the belief network could be
implemented. There will also be a discussion of the pros and cons compared to
the different solutions.

\section{Implementation}
We did not implement a belief network, therefore this section will be used to
discuss how it could have been implemented. To implement the network there are
two options, one where it could be implemented on the NXT and the other where it is
implemented on a PC.

% Introduction: We didnt really implement the MI-part.
% \textbf{If} it was to be implemented we had two options:
%
% \begin{enumerate}
% \item Implement the belief network on the NXT.
% \item Run the MI-agent on a computer and establish communication between the brick and the computer.
% \end{enumerate}

\subsection{Option One}
One of the first issues with implementing the model on the NXT is the amount of
RAM for the computing the network. The primary limitation for the NXT is the
low amount of RAM, namely 64KB, see \autoref{PlatformC}. Although the flash
storage on the NXT is larger, at 256KB, it is still necessary to load the code
and data into the RAM, so the real limiting factor is still the amount of
available RAM. In order to implement the network on the NXT, it is necessary to
reduce the complexity of the model or else it wont be possible to implement it on the NXT.
Since the first network had a joint probability of $3^2*5^6 = 140625$, the
calculation is made using the number of states in each node. In this case it
means that there are 2 nodes with 3 states and 6 node with 5 states, the
grouping of these can be seen in \autoref{BNGrouped}. The joint probability
yields a theoretical size of 560KB given that each state is 4 bytes. But
since this is the joint probability for the entire network, this calculation is
the worst case. It would thus be possible to use variable elimination to simplify the needed
tables. Another thing to consider is loading some of the tables into the flash
storage of the NXT, this could potentially make it possible to implement a more
complex network. However this also introduces the need for swapping in and out
of memory.
%  if we remove some of
% the states which are not necessary according to the implementation, the state space could be reduced
% to $3^1*5^4 = 1875$ which given a size of the state of each space at 4 bytes
% would need 7.5KB in the worst case. Given this state space it would be possible
% to implement the belief network on the NXT.\nl

A possible way to implement the network on the NXT would be to choose another
programming language for the brick. If for example a programming language such
as leJos was used on the NXT it would in theory be possible to implement the
Hugin API directly on the NXT. This language could potentially be used as the
language is a tiny Java Virtual Machine (JVM) \cite{LeJos}. One of the
advantages of using the JVM is that it is platform independent, This means
that the JVM uses the same instruction set on the PC and on the brick. This
solution however does present some issues, the first being the amount of flash
memory on the NXT and the second being the runtime, this is an issue since the
processor in the NXT is not very powerful compared to a PC.
The third issue is that there are no guaranties that the Hugin API will work with it.

\figx{BNGrouped}{Done Bayesian Network Grouped}

\subsection{Option Two}
The second option involves using a PC for computing the
belief network. Using this approach would require the computation to be fully
automated in order to still consider the system embedded.\nl

This approach requires the following things to be setup
\begin{itemize}
\item The belief network modeled on the computer
\item Establish a serial connection between the NXT and the PC
\item Figure out what data structures should be sent and received
\end{itemize}

The first part was realized using Hugin Lite to model the network described
in \autoref{MID}. The model of the network is exported via Hugin as a file.
Using the Hugin API the logic for the network was constructed in C\#.\nl

In order to establish a serial connection between the NXT and the PC, there are
two alternatives. The NXT is capable of communicating with the computer either
via either USB or Bluetooth. The USB approach would require a cable connecting
the NXT to the computer.
This could complicate certain things like the horizontal rotation, since the
cable could potentially restrict the movement by adding more resistance.
Another issue that complicates it even further is how fast we will be able to
send the data to and from the NXT.\nl

Using Bluetooth for communication would have been preferred, since there are
no cables to restrict the turret when it turns.
This approach would require a PC with Bluetooth connectivity. Once a connection
has been established, it uses the NXT built-in mailbox system for transferring
messages between the brick and the PC.\nl

Next step would be to figure out in what format the communication should occur.
The NXT should send the sensor input needed for evidence to the PC. Since our
sensor values are defined in the model in intervals, we need to convert the
sensor value to the corresponding interval in the node. This can be achieved two ways. Either the NXT does the conversion and sends the modified value to the PC, or the NXT
sends the full sensor value and the conversion is done on the computer before
being fed as evidence to the model.\nl

The other part of the communication is the response the PC should generate. The
network generates a vector that needs to be sent back to the NXT, to be used for
calculating a firing position. The idea was to use the NXT mailbox system's
built in \textc{SendString}, so we send the information as strings and then
construct functions to package and unwrap the information.

\subsection{Conclusion}
The two solutions each present some issues, option 1 might not solve the problem
as there is a lot of uncertainty since there are no guarantees that the Hugin
API will work. Another issue with option 1 is the limited memory and storage on
the NXT. Option 2 has a greater chance of working as the only issue we
encountered was establishing serial communication between the NXT and the PC.
This problem could however with some more time and research be solved.
Therefore, the second option is a better way to implement it.

% Our idea for implementing the network is using a computer running a program
% written in C\# which takes an input from the NXT, this input would be gotten
% via serial communication. This communication should also be used for
% communicating the results from the belief network. In the suggested
% implementation Hugin API will be doing the calculations and the program will act
% as an interface between the calculations made by Hugin and the NXT. \nl

% Another way of potentially implementing the network, would be to implement it
% directly on the NXT. This however presents a new set of issues, the first
% issue being the runtime of the calculations. This is an issue since the NXT is
% not a very powerful platform and doing complex calculations might take a long
% time. The second issue is how much RAM and flash memory the NXT has. Because of these
% issues the network would have to be very simplified and thus not as effective
