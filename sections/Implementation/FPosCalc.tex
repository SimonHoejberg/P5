\section{Calculate Future Position}
As the turret needs to calculate the future position of the target, we need to
implement a model for executing these computations to an acceptable degree. In
\name this is done by calculating vectors between different points where the
target was observed, and then using these to predict its future location. All
code used to implement this feature can be found in the CalcFPos library.\nl

The first step is to observe the target in a number of positions. These
positions are represented as coordiates in a x,y coordinate system as described
in \autoref{RepScheme}. Positions are saved with information about coordiates,
timestamp and turret-angle. These positions are exemplified below in
\autoref{FPosCalc1}.

\figx[0.35]{FPosCalc1}{A number of target positions have been observed.}

After observing a number of locations, vectors are defined between each set
locations. Each vector is assigned a speed based on the length of the vector and
the time diffence between the two used points. The assignment of these vectors
are shown below in \autoref{FPosCalc2}, where the last variable is the time
difference.

\figx[0.35]{FPosCalc2}{Vectors are made between sets of positions.}

Each of the vectors can now be seen as having a direction and a speed. These
vectors are then added together, and an average direction and speed is found.
The purpose of this is average is mostly to use multiple locations to accurately
determine the direction, as misreadings cause greater inaccuracy the less number
of observations are used. The average of the vectors can be seen in
\autoref{FPosCalc3}.

\figx[0.35]{FPosCalc3}{The average vector is calculated.}

Now, having found the average direction and speed between all observed points,
we will have create a function which can be used to predict a future position.
In order to do this, the speed and direction is divided by the average time
interval. As such, we can create a unit vector, which indicates the distance on
the x,y-axis per milisecond. This vector can now simply be multiplicated with
the time in which we want to determine a location. 

\figx[0.35]{FPosCalc4}{The unit vector is multiplied, and a future position is
calculated.}

In order to calculate the future position, it is important to add the last
observed target position to the result, such that the multiplied vector has its
start in this point rather than the center of the turret. In the example in
\autoref{FPosCalc4} the unit vector is multiplied by 1000 miliseconds appended
to the last known location. This results in the future position (57,22).




