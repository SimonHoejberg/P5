\subsection{Tracking}
Another important task of the program is \textc{Track}, which is responsible
for using the NXTCam to track a target. As described in \autoref{SensorTheory},
the camera is capable of identifying blobs of a given colour, and then output
data relating to where in the camera's field of view the object is located.
Whenever the camera registers a target, it returns data specifying the position
of the blob in the frame. These variables are stored in arrays which can contain
data for 10 different blobs per frame. This can be seen on \textbf{lines 2-7}
in the code example in \autoref{TrackVar}. Based on the data received over time,
the turret is capable of determining the direction of the target. The variables
for this feature can be found \textbf{lines 9-12} in \autoref{TrackVar}.\nl
 
\begin{minipage}[H]{\linewidth}
\begin{lstlisting}[caption = Variables used on the \texttt{Track} task., label
= TrackVar] task Track(){
    int bt[10]; //Blob top coordinate
    int bl[10]; //Blob left coordinate
    int bc[10]; //Blob colour
    int bb[10]; //Blob bottom coordinate
    int br[10]; //Blob right coordinate
    int nblobs; //Number of blobs
    int blobsMissedCounter = 0;  //How many times a call to the camera has returned no blobs
    int oldBlobCenter = 0;
    int blobCenter = 0;
    lib_direction = 0; // [-1]Left, [1]Right, [0]Nothing
    int directionCount = 0; //How many times have we seen movement in a direction
    ...
}
\end{lstlisting}
\end{minipage}

The general behaviour is contained in a while loop seen in \autoref{TrackFirst},
which is executed based on the value of the boolean variable
\textc{track} which defines whether or not the tracking phase is still
ongoing. The behaviour starts on \textbf{line 3} by using the
\textc{NXTCam\_GetBlobs} function to update the blob variables with
the newest frame from the camera. If the number of blobs are greater than zero,
the center of the largest blob, corresponding to the first entry in the array,
is determined on \textbf{line 6}.\\ 
The boolean on \textbf{line 3} indicates that the blob is in the view of the
camera, this boolean is used in the \textc{GetDataAndCalculate} task
in \autoref{GetDataAndCalc} to gather data.\nl
  
\begin{minipage}[H]{\linewidth}
\begin{lstlisting}[caption = First part of the general tracking behaviour., label = TrackFirst]
 while(track){
	NXTCam_GetBlobs(CAM_PORT, CAM_ADDR, nblobs, bc, bl, bt, br, bb);
    if(nblobs > 0){
    	blobInCam = true;
        blobsMissedCounter = 0;
        oldBlobCenter = blobCenter;
        blobCenter = (bl[0] + UpperBound(br[0]))/2;
        int tempSpeed = 0;
        Off(ROTATE_MOTOR);
        ...
	}
...
}          
\end{lstlisting}
\end{minipage}

After receiving data from the camera, the turret attempts to track the target by
rotating at a speed matching the target's movement. This is shown in
\autoref{DetSpeed}. This operation is executed after the turret has
determined in which direction the target is moving. If the direction has
been determined (\texttt{lib\_direction!= NO\_DIRECTION}), the
\textc{GetSpeed} function is called in order to determine the speed of the
target. This function is shown in \autoref{GetZeSpeed}.\\
In \autoref{DetSpeed} on \textbf{lines 3-6}, if the determined rotation speed
matches the direction in which the target is moving, the motor is set to rotate at this speed.\nl

\begin{minipage}[H]{\linewidth}
\begin{lstlisting}[caption = Determine the targets speed at rotate to match it, label = DetSpeed] 
if(lib_direction != NO_DIRECTION){
	tempSpeed = GetSpeed(blobCenter );
    //Makes sure that the turret can only turn in the direction the target is driving
    if(tempSpeed > 0 && lib_direction == RIGHT_DIRECTION)
    	OnFwd(ROTATE_MOTOR, tempSpeed);
    else if(tempSpeed < 0 && lib_direction == LEFT_DIRECTION)
    	OnFwd(ROTATE_MOTOR, tempSpeed);
}
\end{lstlisting}
\end{minipage}

If the camera is not able to register any targets, the
\texttt{blobsMissedCounter} is incremented, and the turrets rotation is stopped.
This is shown in \autoref{PurgeData}. This counter is incremented continually, until its value
matches the variable \\
\textc{MAX\_COUNTER\_TO\_PURGE\_DATA}, after which all
values involved in tracking are reset, such that the turret can start a new
tracking phase.\nl

\begin{minipage}[H]{\linewidth}
\begin{lstlisting}[caption = Reset data and turn off motor is no targets are found., label = PurgeData] 
...
}else{
	blobInCam = false;
    blobsMissedCounter++;
    //If enough blobs are missed set all the different global values such that the turret can run again
    if(blobsMissedCounter == MAX_COUNTER_TO_PURGE_DATA){
    	blobsMissedCounter = 0;
        posData =  NULL;
        posDataCounter = 0;
        ResetLibVariables();
        directionCount = 0;
	}
    Off(ROTATE_MOTOR);
}
\end{lstlisting}
\end{minipage}

After receiving data from the camera, the turret uses this data to determine the
direction in which the target is moving, either left or right. This is done by
calling the \textc{DetermineDirection} function as shown in
\autoref{TrackDir}. This is only done once for each tracking phase, which is
indicated by the 'if' statement in \autoref{DetSpeed} where the direction
must not be equal to \texttt{NO\_DIRECTION}.\nl

\begin{minipage}[H]{\linewidth}
\begin{lstlisting}[caption = Call \texttt{DetermineDirection}., label =
TrackDir] 
if(nblobs > 0){
...
}else{
...
}
//Used to determine what direction the target is driving in
DetermineDirection(blobCenter, oldBlobCenter, directionCount);
\end{lstlisting}
\end{minipage}

When the \textc{DetermineDirection} method is called, it checks
whether or not a direction has already been determined. Whenever a blob is
registered by the camera, it is determined in what direction the blob has
moved. This is done on \textbf{lines 3-7} where the center of the
current blob is compared to the old one. When the blob moves in a given
direction, \textc{directionCount} is either incremented or decremented.
When this variable reaches a certain positive/negative threshold, the direction
is determined.\nl

\begin{minipage}[H]{\linewidth}
\begin{lstlisting}[caption = Determine the targets direction., label = DeterDir]
void DetermineDirection(int center, int oldCenter, int &directionCount){
    //If direction not found. Check if blob-center has moved left/right for directionTreshold * frames
    if(lib_direction == NO_DIRECTION){
        //Check what direction the target is moving
        if(center > oldCenter){
            directionCount++;
        }
        else if (center < oldCenter) {
            directionCount--;
        }
        //Set direction if count above threshold
        if(directionCount > DIRECTION_THRESHOLD){
            lib_direction = RIGHT_DIRECTION;
        }
        else if(directionCount < DIRECTION_THRESHOLD * (-1)){
            lib_direction = LEFT_DIRECTION;
        }
    }
}
\end{lstlisting}
\end{minipage}

Aside from determining the direction, the turret also needs to determine at
which speed it should rotate to match the target. This is calculated in the
\textc{GetSpeed} method, as shown in \autoref{GetZeSpeed}. This method
takes the pixel count between the center of the camera and the blob, and outputs
the resulting rotating speed between 1 and \texttt{MaxSpeed}. The sine function
maps the 144 pixel width of the camera to 90 degrees for a sine function. This is shown below in
\autoref{sineF}.

\begin{equation}\label{sineF}
speed = sin(\frac{90}{PixelWidth})*MaxSpeed
\end{equation} 

By using this approach, the speed should follow a sine curve, and as such
should get slower the closer to the target, and faster the farther away.\nl

\begin{minipage}[H]{\linewidth}
\begin{lstlisting}[caption = Determine speed to rotate using a sine function, label = GetZeSpeed] 
float GetSpeed(int blobcenter){
    int length = blobcenter - CAM_CENTER;
    float lib_speed = sind(MODIFIER*length)* MAX_HORIZONAL_SPEED;
    return lib_speed;
}
\end{lstlisting}
\end{minipage}

In conclusion, the tracking phase consists of reading input from the camera and
using this data to track the target. This is done by determining the target's
direction, and using a sine function to map the pixel-difference in blob
positions to a respective rotational speed.