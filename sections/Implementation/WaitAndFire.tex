\subsection{WaitAndFire}
When \name has collected enough data to calculate where the target is going to
be in the future, the task \textc{WaitAndFire} is performed. This task makes
\name stop collecting data, loads the canon, turns to the calculated angle, angles the
canon vertically and waits until the right time to fire. When it is done fireing
it starts to track again.\nl

This task calls three functions to get it ready to fire. The first function is
\textc{LoadRound} which pushes the piston out so that the projectile is locked
in place. The second is \textc{RotateH} which turns \name horisontally to the
future position of the target. The third function called \textc{Angle} angles
the canon vertically based on the distance to the target. These three function calls can be seen in
\autoref{Funccall}\nl

\begin{minipage}[H]{\linewidth}
\begin{lstlisting}[caption = Function calls in WaitAndFire, label = Funccall] 
task WaitAndFire(){
	collecting = false;
    LoadRound();
    Off(ROTATE_MOTOR);
    //Rotate and angles the turret
    RotateH(fire.angleH);
    Angle(fire.angleV);
    ...
}
\end{lstlisting}
\end{minipage}

The data contained in the struct called \textc{fire} is calculated in the the
function \textc{CalcFireData} and \textc{CalcFuturePos}, which is explained in
\autoref{RepScheme}

The functionality that determines when to fire is contained in a while loop with
an if statement that checks if the current time is bigger than or equal to the
time to fire which is contained in the struct called \textc{fire}. This is seen
in \autoref{Whentofire}\nl
 
\begin{minipage}[H]{\linewidth}
\begin{lstlisting}[caption = While loop that determines when to fire, label =
Whentofire]
 ...
//Waits for the right time to fire
while(true){
	if(CurrentTick() >= fire.timeToHit && !alreadyFired){
    	alreadyFired = true;
        Fire();
        break;
    }
}
\end{lstlisting}
\end{minipage}





