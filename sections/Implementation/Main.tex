\section{Tasks}\fix{}{quite a low level walk though}
The developed software is controlled by the execution of a number of
tasks. On the NXT system a task is simply a collection of code which can be
executed on a thread. The turret makes use of four central threads:
\texttt{Main}, \texttt{GetDataAndCalculate}, \texttt{WaitAndFire} and
\texttt{Tracking}.

\subsection{Main}
The execution of the program begins in the \textc{Main} thread, which is shown
in \autoref{MainThread}. First, the library method
\textc{PosRegEnable} is used to enable the motors ability to set
an absolute position using a PID controller. This is used to allow the turret to
set its angle instead of simply turning on/off the motor.
The \textc{SetMotorRegulationTime} method determines the interval between when
the motor should attempt to regulate its angle to the respective value. This
value is set to 10 milliseconds, as we have observed that higher values result
in the turret being incapable of standing still.

\begin{minipage}[H]{\linewidth}
\begin{lstlisting}[caption = Entry point for the program execution., label = MainThread] 
task main(){
    PosRegEnable(ROTATE_MOTOR);
    PosRegEnable(ANGLE_MOTOR);
    SetMotorRegulationTime(10);
    NXTCam_Init(CAM_PORT, CAM_ADDR);
    NXTCam_SendCommand(CAM_PORT, CAM_ADDR, 'A');
    NXTCam_SendCommand(CAM_PORT, CAM_ADDR, 'E');
    SetSensorUltrasonic(SENSOR_LEFT);
    SetSensorUltrasonic(SENSOR_RIGHT);
    Precedes(Track,GetDataAndCalculate);
}
\end{lstlisting}
\end{minipage}

On \textbf{lines 5-7} the NXTCam is initialized, and two commands are
sent to the camera. The command on \textbf{line 6} configures the camera
to sort the returned blobs by size, such that the largest blob is the first
element of the returned arrays. \textbf{Line 7} is used to enable the camera,
and start the blob tracking. \textbf{Line 8-9} are used to
initialize the two sensor ports \textc{SENSOR\_LEFT} and
\textc{SENSOR\_RIGHT} as they are used for ultrasonic distance sensors.
Finally, the \textc{Precedes} method is used to initialize the
threads: \textc{Track} and \textc{GetDataAndCalculate}.
