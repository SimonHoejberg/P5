\section{Programming Language}
The software used in the \name project has been written in the language NXC,
which is a domain specific language specialized for writing programs for the
LEGO NXT platform \cite{NXCIntro}. NXC is similar to the language C, and as such
uses many of the same constructs, including the imperative programming approach. As the
language is domain specific, it comes with its own libraries and their
respective documentation \cite{NXCIntro}. While NXC is similar to C, the main
downside is that it is not possible to make use of the standard C libraries, due
to the syntactical differences in elements such as pointers. As described in the NXC documentation,
the language has a limited selection of data types \cite{NXCVariables}. An
example of a short program written in NXC can be seen below in
\autoref{NXCExample}.\nl

\begin{minipage}[H]{\linewidth}
\begin{lstlisting}[caption = Exaple of a program written in the NXC language, label = NXCExample] 
#include "nxtcamlib-default.nxc"
const byte ROTATE_MOTOR = OUT_A;
task search();

task main(){
    Precedes(search);
    RotateMotor(ROTATE_MOTOR, 50, 50);
}

task search(){
	bool track = true;
	if(TESTVALUE == 10)
	while(track){
        NXTCam_GetBlobs(CAM_PORT, CAM_ADDR, nblobs, bc, bl, bt, br, bb);
	}
}
\end{lstlisting}
\end{minipage}

As exemplified in \autoref{NXCExample} the NXC is a C like language, and
the most notable differences come in the form of the inclusion of standard
libraries, and specialized thread handling in the form of tasks.
