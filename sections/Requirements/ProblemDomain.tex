\chapter{Problem Domain}\label{ProblemDomain}
As the project has no actual customer to set the requirements for the project,
the product requirements will be based on the Ren\'e's proposed challenge and
the educational goals for the semester.\nl

In this chapter we construct an ideal scenario. This scenario is then analyzed in order to
understand the problem domain, and finally a summary is given.

\section{Ideal Scenario}
Based on the initial problem it is now possible to describe how the turret is
supposed to operate. We construct an ideal scenario in order to better
visualize the initial problem and to serve as a basis for the requirements engineering.
The ideal scenario involves the two actors the ``Turret'' and the ``Target'',
 and
describes how they interact with each other.

\begin{center}
\colorbox{diff}{
\begin{minipage}{0.8\linewidth}
\textbf{Ideal Scenario:}

The turret is set down on an even floor and activated. It then turns searching
for a target by turning left and right, scanning an area for possible
targets.\nl

The target is moving on an even plane at a constant but unknown speed driving
in a straight line, but entering the firing zone from an unknown angle. The
target moves at a speed such that the turret will miss if it fires when the
target is directly in front of the turret.\nl

The turret detects a target, it begins to predict the targets movements in
order to anticipate the targets position at any given time.\nl

The turret moves into a firing position based on the predicted position. It then
fires at the appropriate time, in order to hit the target at the predicted
position.\nl

The turret stops sensing the target and start scanning the area again.
\end{minipage}
}
\end{center}

This scenario can be visualized with \autoref{turretScenario} which shows process of
finding and shooting a target.

\figx[0.45]{turretScenario}{A representation of the ideal scenario.}

\section{Phases}
Based on the ideal scenario three distrinct phases are identified,
``Scanning'', ``Tracking'' and ``Shooting''. The ``Scanning'' phase is where
the turret searches for a target by turning. The ``Tracking'' phase is where the
turret collects data and calculates the trajectory for the target. The
``Shooting'' phase is where the turret moves to a firing position and shoots at
the correct time. These phases will now be examined in order to identify the
problems in their respective domain.

\subsection{Scanning}
The ``Scanning'' phase where the turret monitors an area presents two different
situations. The first situation is when no target is detected and it continuosly
scans for a target. The second situation is when a target is detected and it
switches to the ``Tracking'' phase.\nl

This leads to the following problems:
\begin{itemize}
  \item How can an area for the turret to monitor be defined?
  \item How does the turret identify a target?
\end{itemize}

\subsection{Tracking}
The ``Tracking'' phase which occurs when a target is found, is used to follow a
target and gather data about its movements. This information is then used to
predict where the target will be at any given time as well as to calculate a
shooting position.\nl

This leads to the following problems:
\begin{itemize}
  \item How can the turret predict a targets movement?
  \item How does the turret track the target?
\end{itemize}

\subsection{Shooting}
The ``Shooting'' phase occurs when the cannon has determined where the target
will be, when it will be there and how the turret will hit it. The turret then moves
into the a shooting position and proceeds to shoot at the correct.\nl

This leads to the following problems:
\begin{itemize}
  \item How does the turret move into a shooting position?
  \item How does the turret shoot?
  \item How does the turret hit a target at varying distances?
  \item What deadlines does the system need in order to hit a target?
\end{itemize}

\section{Summary}
In conclusion, we construct an ideal scenario based on the initial problem. We
already determined the following as hard requirements based on the semester
topic:

\begin{enumerate}
	\item The system must be embedded
	\item The system must use techniques from either the machine intelligence or
	real-time systems fields or both
\end{enumerate}

Three distinct phases were identified, ``Scanning", ``Tracking'' and
``Shooting''. Each phase was analysed in order to find the problems associated
with it. These problems which are mechanical, real-time and reasoning based
problems are used as the foundation for the analysis. In the next chapter
we will examine the what an embedded system is, what makes a real-time system
and how machine intelligence will integrate with the project.