\chapter{Problem Domain}
Fluff and red thread
(Go indepth with the phases)


\section{Ideal Scenario}
Based on the initial problem it is now possible to describe how the turret is
supposed to operate. An ideal scenario is constructed in order to better
visualize it, and to serve as a basis for the requirements engineering. The
ideal scenario involves the two actors ``Turret'' and ``Target'' and
describes how they interact with each other.

\begin{center}
\colorbox{diff}{
\begin{minipage}{0.8\linewidth}
\textbf{Ideal Scenario:}

The turret is set down on an even floor and turned on. It then turns searching
for a target by turning left and right scanning an area for possible targets.

If the turret detects a target, it should be able to predict the targets
movement, in order to anticipate the targets position at any given time.

The target is moving on an even plane at a constant speed driving in a straight
line(perpendicular?). The target moves at a speed such that the turret will miss
if it fires when the target is directly in front of the turret.

The turret moves into a firing position based on the predicted position and
then fires at the appropriate time, in order to hit the target at the predicted
position. 

If the turret stops sensing targets, it should start scanning the area again.

\end{minipage}
}
\end{center}


\section{Phases}
Based on the ideal scenario three distrinct phases are identified,
``Scanning'', ``Tracking'' and ``Shooting''. The ``Scanning'' phase was where
the turret searched for a target by turning. The ``Tracking'' phase where the
turret collects data and calculates the trajectory for the target. The
``Shooting'' phase where the turret moves to a firing position and shoots at the
correct time. These phases will now be examined in order to identify the
problems in their respective domain.

\subsection{Scanning}
The turret needs to be able to monitor an area and detect if a target enters
within range. In order to accomplish this, we need to diffentiate between two
situations 
-No target detected: The turret does not detect any targets and continuously
scans for target 
-target detected: a target has been identified within the firing-zone. The
turret should then start the ``tracking'' phase.

This presents us with the following problems:
-How does the turret identify a target?

% -A: through sensor input we monitor the area and if a change occurs in the 
% 
% In order for the turret to detect a target, it first needs to be in range.
% 
% In order for our turret to be able to hit a target, it first needs to detect
% it.
% -Differentiate between background and targets? 
% -Have a zone it must protect
% -know a way to normalize after firing
% -Identify a target 
\subsection{Tracking}
If a target has been detected, the turret must track it to gather
information in order to properly predict the targets movement and calculate a
firing-position for the turret.

 -What data is needed in order to properly
calculate a trajectory and predicted positions of a target
-What data is needed in order to properly calculate a firing-position and
firing-time for the turret in order to hit the target.
- How does the turret track the target?

\subsection{Shooting}
Once sufficient data is collected from the tracking-phase the turret needs to be
able to move into a firing-position such that it is able to hit the target at
the predicted position. It needs to be able to time the shot in order to take
the travel time of the shot into consideration.

Deadline questions:
- What deadlines does the system need to have in order to hit a target?
(Software deadlines and mechanical deadlines)
Mechanical questions:
- How does the turret move into a firing position?
- How does turret shoot?
- How does the turret hit a target a varying distances?

\subsection{Target}
-How big is the target?
-how fast is the target?
-What colout is the target?

\section{Scenarions}
Based on the on the 


Based on the initial problem it is now possible to describe how the turret is
supposed to operate. In order to better visualize it, two concrete scenarios
have been created with the objective to illustrate it from different angles.

\subsection{Concrete Scenario: Single Target}
The turret is set down on an even floor and turned on. It then turn searches
for a target by turning 90 degrees to the left. It then starts turning 180
degrees to the right and repeats this by switching between left and right.\nl

The turret detects a target at a distance of 5m moving at a speed of 2 m/s, it
then collects data concerning the targets speed, and predicts where when and
where it will be within shooting range. It then turns to the appropriate angle
and waits until it is able to hit the target with acceptable accuracy. The
turret repeats firing, calculating and turning until the target can no longer
be seen.\nl

The turret then repeats returns to searching for a target.\nl

\figx{1Scenario.png}{How the different aspects are seen - not final
pic.}{}

\todo{Pic is not final}

\subsection{Concrete Scenario: Multiple Targets}




