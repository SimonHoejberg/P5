\chapter{Problem Domain}\label{ProblemDomain}
As the project has no actual customer to set the requirements for the project,
the product requirements will be based on the Ren\'e's proposed challenge and
the educational goals for the semester.\nl 
As such, this chapter will be used to construct an ideal scenario for the
turrets operation, which will then be analyzed in order to define the problem
domain and initial goals for the project.

\section{Ideal Scenario}
Based on the proposed challenge, it is possible to describe an ideal scenario
of how the turret should operate. We construct an ideal scenario in order to better
visualize the initial problem and to serve as a basis for specifying the
requirements. The ideal scenario involves the two actors the ``Turret'' and the
``Target'', and describes how they should interact with each other.

\begin{center}
\colorbox{diff}{
\begin{minipage}{0.8\linewidth}
\textbf{Ideal Scenario:}

The turret is set down on an even floor and activated. It then rotates
horizontally while searching for a target by turning left and right, scanning an
area for possible targets.\nl

The target is moving on an even plane at a constant but unknown speed, while
driving in a straight line. It enters the turrets the firing zone from an
unknown angle. The target moves at a speed such that the turret will miss if it
fires when the target is directly in front of the turret.\nl

The turret detects a target, and begins to track its movements. The turret then
begins to predict the targets movements in order to anticipate the targets
position at any future point in time.\nl

The turret moves into a firing position based on the predicted position. It then
fires at the appropriate time, in order to hit the target when it moves to the
predicted position.\nl

The turret stops sensing the target and start scanning the area again.
\end{minipage}
}
\end{center}

This scenario is visualized in \autoref{turretScenario}, which shows the process
of finding and shooting a target.

\figx[0.45]{turretScenario}{A representation of the ideal scenario.}

\section{Phases}
Based on the ideal scenario three distrinct phases are identified,
\textbf{Scanning}, \textbf{Tracking} and \textbf{Shooting}. The Scanning phase
consists of the turret turning left and right in order to search for a target.
The Tracking phase is where the turret collects data and calculates the
trajectory for the target. The Shooting phase consists of the turret moving to
a firing position and shoots at the predicted location at the correct time.
These phases represent the turrets entire behaviour, and will now be examined in
order to identify any problems in the respective phases.

\subsection{Scanning}
The \textbf{Scanning} phase, where the turret monitors an area, presents two
different outcomes. The first outcome is when no target is detected and it
keeps continuosly scanning for a target. The second outcome is when a target
is detected, and the turret switches to the \textbf{Tracking} phase.\nl

This leads to the following problems:
\begin{itemize}
  \item How can an area for the turret to monitor be defined?
  \item How does the turret identify a target?
\end{itemize}

\subsection{Tracking}
The ``Tracking'' phase which occurs when a target is found, is used to follow a
target and gather data about its movements. This information is then used to
predict where the target will be at any given time as well as to calculate a
shooting position.\nl

This leads to the following problems:
\begin{itemize}
  \item How can the turret predict a targets movement?
  \item How does the turret track the target?
\end{itemize}

\subsection{Shooting}
The \textbf{Shooting} phase begins when the cannon has determined where the
target will be, when it will be there and how the turret will hit it. With
these informations, the turret moves into the calculated shooting position and
proceeds to shoot at the determined time.\nl

This leads to the following problems:
\begin{itemize}
  \item How does the turret move into a shooting position?
  \item How does the turret shoot?
  \item How does the turret hit a target at varying distances?
  \item What deadlines does the system need in order to hit a target?
\end{itemize}

\section{Summary}
In conclusion, we have constructed an ideal scenario based on the initial
problem, and analyzed this scenario in order to gain an insight into what
design problems must be overcome. We already determined the following as hard
requirements based on the semester topic:

\begin{enumerate}
	\item The developed system must be classifiable as an embedded system
	\item The developed system must use techniques and models taught in either the
	machine intelligence or real-time systems courses
\end{enumerate}

During the analysis, three distinct phases were identified, \textbf{Scanning},
\textbf{Tracking} and \textbf{Shooting}. Each phase was analysed in order to
determine what design problems could be associated with it. These problems
which are mechanical, real-time and reasoning based problems, are used as the
foundation for the analysis. In the next chapter we will examine what an
embedded system is, what constitutes a real-time system and how machine
intelligence can be integrated with the project.
