\chapter{Problem Domain}
Fluff and red thread
(Go indepth with the phases)


\section{Ideal Scenario}
Based on the initial problem it is now possible to describe how the turret is
supposed to operate. An ideal scenario is constructed in order to better
visualize it, and to serve as a basis for the requirements engineering. The
ideal scenario involves the two actors ``Turret'' and ``Target'' and
describes how they interact with each other.

\begin{center}
\colorbox{diff}{
\begin{minipage}{0.8\linewidth}
\textbf{Ideal Scenario:}

The turret is set down on an even floor and turned on. It then turns searching
for a target by turning left and right scanning an area for possible targets.

If the turret detects a target, it should be able to predict the targets
movement, in order to anticipate the targets position at any given time.

The target is moving on an even plane at a constant speed driving in a straight
line(perpendicular?). The target moves at a speed such that the turret will miss
if it fires when the target is directly in front of the turret.

The turret moves into a firing position based on the predicted position and
then fires at the appropriate time, in order to hit the target at the predicted
position. 

If the turret stops sensing targets, it should start scanning the area again.

\end{minipage}
}
\end{center}


\section{Phases}
Based on the ideal scenario three distrinct phases are identified,
``Scanning'', ``Tracking'' and ``Shooting''. The ``Scanning'' phase was where
the turret searched for a target by turning. The ``Tracking'' phase where the
turret collects data and calculates the trajectory for the target. The
``Shooting'' phase where the turret moves to a firing position and shoots at the
correct time. These phases will now be examined in order to identify the
problems in their respective domain.

\subsection{Scanning}
The ``Scanning'' phase which is used when the turret is idle, monitors an
area. This presents two different situations. The first situation is when no
target is detected and it continuosly scans for a target. The second situation
is when a target is detected and it switches to the ``Tracking'' phase.\nl

This leads to the following problems:
\begin{itemize}
  \item How can an area for the turret to monitor be defined?
  \item How does the turret identify a target?
\end{itemize}

% -A: through sensor input we monitor the area and if a change occurs in the 
% 
% In order for the turret to detect a target, it first needs to be in range.
% 
% In order for our turret to be able to hit a target, it first needs to detect
% it.
% -Differentiate between background and targets? 
% -Have a zone it must protect
% -know a way to normalize after firing
% -Identify a target 

\subsection{Tracking}
The ``Tracking'' phase which occurs when a target is found, is used to follow a
target and gather data about its movements. This information is then used to
predict where the target will be at any given time as well as to calculate a
shooting position.\nl

This leads to the following problems:
\begin{itemize}
  \item How can the turret predict a targets movement?
  \item How does the turret track the target?
\end{itemize}

\subsection{Shooting}
The ``Shooting'' phase occurs when it is determined where the target will be,
when it will be there and how the turret will hit it. The turret then moves
into the a shooting position and proceed to shoot at the correct.\nl

This lead to the following problems:
\begin{itemize}
  \item How does the turret move into a shooting position?
  \item How does the turret shoot?
  \item How does the turret hit a target a varying distances?
  \item What deadlines does the system need to have in order to hit a target?
\end{itemize}

\section{Summary}
In conclusion, an ideal scenario was constructed based on the initial problem.
Certain hard requirements can be determined already based on the semester topic
\begin{enumerate}
	\item The system must be embedded
	\item The system must use techniques from either the MI or the RTS course, or both
\end{enumerate}


In addition several problems were identified in the problemdomain:
\begin{itemize}
	\item How will the cannon detect the target?
	\item What information is needed to properly calculate and predict the trajectory of a target
	\item What information is needed in order to calculate a firing-position and firing-time of the cannon in order to reliably hit the target
	\item How does the turret move into firing position?
	\item How does the turret shoot the target
	\item How does the turret hit targets at varying distances
\end{itemize}
These questions will serve as the foundation for the analysis, and help to refine the initial problem into a problemstatement. In the next chapter the theory behind embedded systems will be explained in order to properly understand the constraints imposed by such a system. 

