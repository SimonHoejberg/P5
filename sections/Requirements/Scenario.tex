\chapter{Requirements}
\section{Scenarios}
\subsection{Ideal scenario}
An ideal scenario for the turret is presented:\nl

The turret is able to turn 180 degrees.\nl

The turret is able to adjust its firing angle 45 degrees.\nl
 
The turret has a max shooting range of 350 cm.\nl

The turret is able to detect targets at a range of up to 500 cm.\nl

The turret is able to track a target once detected, in order to collect data.\nl

The turret is able to predict where a moving target will be in order to hit it,
both when the target has constant speed and when the speed varies.\nl

The turret is able to hit a moving a target traveling with a speed that
requires it to adjust its cannon in order to hit the target.\nl

The turret is able to prioritize and handle multiple targets.\nl

The target has a size of 10cm x 10cm.\nl

The target is able to move at a constant speed of up 2.5 m/s.\nl

The target is able to accelerate and decelerate.\nl

The target is traveling on an even plane.\nl

\figx{BaseScenario.png}{How the different aspects are seen - not final
pic.}{}

\todo{Pic is not final}

\subsection{Concrete scenario 1 - single target:}
The turret is set down on an even floor and turned on. It then turn searches
for a target by turning 90 degrees to the left. It then starts turning 180
degrees to the right and repeats this by switching between left and right.\nl

The turret detects a target at a distance of 5m moving at a speed of 2 m/s, it
then collects data concerning the targets speed, and predicts where when and
where it will be within shooting range. It then turns to the appropriate angle
and waits until it is able to hit the target with acceptable accuracy. The
turret repeats firing, calculating and turning until the target can no longer
be seen.\nl

The turret then repeats returns to searching for a target.\nl

\figx{1Scenario.png}{How the different aspects are seen - not final
pic.}{}

\todo{Pic is not final}

insert a sketch.
\subsection{Concrete scenario 2 - multiple targets:}

The turret is set down on an even floor and turned on. It then turn searches
for a target by turning 90 degrees to the left. It then starts turning 180
degrees to the right and repeats this by switching between left and right.\nl

The turret detects a target T1 at distance of 5 meters to the right, moving at
a speed of 1 m/s. It then tracks and collects data about T1's position and
speed.\nl

The turret predicts whether or not the target will enter its firing distance,
and calculates that it will indeed enter the firing-range.\nl

After predicting that T1 will enter the firing-rage, the turret will start
calculating the firing-position in order to be able to hit the target. While
calculating this position, the turret detects another incoming target T2, at a
distance of 150cm travelling at a speed 1.6m/s due west. (The cannon should
prioritize targets sequentially) insert a sketch.\nl

The cannon continues targeting T1, and moves to the calculated
shooting-position, it then waits for T1 and shots when it has an acceptable
chance of hitting the target. It then calculates a secondary position for T1
and moves into a suitable position and fires when it is acceptable.\nl

It then senses that T2 is still within shooting range and proceeds to calculate
suitable shooting position, where upon it fires at the designated time. As T2
leaves the shooting range the turret the turret begins to scan for targets
again.\nl

% \subsection{Scenario1 - Tracking}
% 
% Den bliver sat p� en plan overflade og derefter t�ndt ved at trykke p� en knap.
% 
% Den begynder nu at langsomt at dreje 45 grader til hver side fra sit udgangs
% punkt. 
% 
% Den ser bev�gelse i baggrunden og begynder at f�lge m�let som bev�ger sig i en
% bue 2 meter uden om turreten. M�let bliver ved med at holde en vis
% distance til turreten.
% 
% Turreten f�lger m�let indtil det har bev�get sig mere end 60 grader v�k fra
% nulpunkts vinklen. Turreten g�r da tilbage til nulpunkts vinklen.
% 
% \subsection{scenario2 - Idle}
% 
% Den bliver sat p� en plan overflade og derefter t�ndt ved at trykke p� en knap.
% 
% Den begynder nu at langsomt at dreje 45 grader til hver side fra sit udgangs
% punkt.
% 
% Den ser ingen bev�gelse og fors�tter med at dreje 45 grader til hver side.
% 
% \subsection{scenario3 - Shooting}
% 
% Den bliver sat p� en plan overflade og derefter t�ndt ved at trykke p� en knap.
% 
% Den begynder nu langsomt at dreje 45 grader til hver side fra sit udgangspunkt.
% 
% Der ser bev�gelse i baggrunden og begynder at f�lge m�let. Den finder frem til
% at m�let er kommet ind i dens killzone.
% 
% Den begynder nu at observere og derefter forudsige hvor m�let ved v�re om kort
% tid. 
% 
% Efter dette er beregnet stiller den sig op s� den vil kunne ramme m�let n�r den
% skyder.
% 
% Den gentager nu dette indtil m�let har bev�get sig uden for r�kkevidde,
% hvorefter den stopper med at skyde. Hvis m�let ikke bev�ger sig videre,
% fors�tter den med at f�lge m�let. Ellers g�r den tilbage til at idle.
