\chapter{Problem Statement}


Throughout the analysis several observations and conclusions were made in
regards to the design and future implementation of the autonomous turret. In
\autoref{ProblemDomain} it was stated that based on the turrets ideal scenario,
it would need to implement a feature that would allow it to sense its
surroundings. Due to this, it was chosen in \autoref{SensorChoice} that the
turret would make use of two sensor types: a camera for tracking, and an
ultrasonic sensor for determining the distance to the target.\nl


% prob needs to be rewritten as to take the conclusion of system description
% more into account
In relation to the constraints imposed by the hardware limitations of
embedded systems, it was determined in \autoref{EmbConc} that the turret should
be:
\begin{itemize}
  \item Single-functioned - It should be constructed with the singular purpose
  of hitting a target in motion.
  \item Tightly-constrained - The system should be designed to adhere to the
  imposed hardware limitations.
  \item Reactive - The turrets functions should wait for when the sensory input
  dictates that it should react, e.g. shooting a target that moves into range. 
\end{itemize}\nl

In regards to hitting a target in motion, \autoref{ProjPhys} was used to
clarify the necessary theory for the project. This information is used to
establish a formula for predicting where and when a projectile will hit. This
formula was tested in \autoref{AppendixDistTest} and can be used by the turret
during its operation.\nl

%Are these the reasons behind our choice?
In \autoref{PlatformC} it was chosen that the Lego NXT placform would be the
better choice for this project. This choice was based on the following
conclusions:
\begin{itemize}
  \item Constructing a physical prototype is easily done using Lego.
  \item The cost of the platform/sensors is irrelevant, as all parts are
  supplied by the university.
  \item As the modules are specifically made for the platform, more time can be
  spent on developing the software.
\end{itemize}\nl


Based on these observations and conclusions, we can establish the following
problem statement:

\begin{center}
\colorbox{diff}{
\begin{minipage}{0.8\linewidth}
\textit{How can an autonomous turret be constructed, designed and implemented
such that it is capable of hitting a moving target.}
\end{minipage}
}
\end{center}

Based on this problem statement a number of refined requirements have been made.
These can be found below in \autoref{FeatAndReq}. 
