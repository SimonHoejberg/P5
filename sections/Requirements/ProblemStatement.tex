\chapter{Problem Statement}

Throughout the analysis several observations and conclusions were made in
regards to the design and future implementation of the autonomous turret. 

In \autoref{ProblemDomain} it was stated that based on the turrets ideal
scenario, it would need a feature that would allow it to sense its
surroundings. Due to this, it was decided in \autoref{SensorChoice} that the
turret would make use of two sensor types: a camera for tracking, and an
ultrasonic sensor for determining the distance to the target.\nl

The system is an embedded system which means that it is constrained due to
hardware limitations, and that it should be constructed for a single purpose,
in this case: hitting a moving target. The system also needs to be an
intelligent agent, which is only able to partially observe the states. This
results in the need to use a belief network to predict the likelyhood of
possible outcomes.

In regards to hitting a target in motion, \autoref{ProjPhys} was used to
clarify the necessary theory for the project. This information is used to
establish a formula for predicting where and when a projectile will hit. This
formula was tested in \autoref{AppendixDistTest} and can be used by the turret
during its operation.\nl

In \autoref{PlatformC} the Arduino and NXT were compared, and Lego NXT
platform was chosen as the prefered choice for this project. This choice was
based on the following conclusions:
\begin{itemize}
  \item Constructing a physical prototype is easily done using Lego.
  \item The cost of the platform/sensors is irrelevant, as all parts are
  supplied by the university.
  \item As the modules are specifically made for the platform, more time can be
  spent on developing the software.
\end{itemize}

Based on these conclusions, we can establish the following
problem statement:

\begin{center}
\begin{minipage}{0.8\linewidth}
\textit{How can an autonomous turret be constructed, designed and implemented
such that it is capable of hitting a moving target.}
\end{minipage}
\end{center}

Based on this problem statement a number of refined requirements have been made.
