\section{Requirements}

Based on\ldots, we can determine the requirements for \namep. These requirements
can further be divided into the groups ``Need to have'' and ``Nice to have''.

\subsection{Need to have}
The following requirements have been deemed necessary for the system:

\begin{itemize}
  \item The turret needs to be an embedded system and incorparate elements from
  the courses MI and RTS. This is necessary as it is required by the semester.
  \item The turret needs to be able turn a minimum of 90 degrees. Since the
  turret should monitor an area and be able to hit a moving target, it needs to
  be able to turn. If it is not able to turn, line of sight as well as the
  timing in order to hit a target will be severely limited. In regards to how
  far it should be able to turn, a minimum is set to 90 degrees, as this will
  provide a decent sized area in front of the turret.
  \item The turret needs to be able to sense a target at a range of atleast
  250cm, since the \fix{current available sensors have a maximum range of
  255cm}{Should be changed depending on max range of optical}, as long as the
  lighting is good and the target is suitable.
  \item The turret needs to be able to search for targets, as otherwise it would
  have to wait for a target to move into its line of sight. This would limit
  the window of time the turret has in order to hit the target.
  \item The turret needs to be able to track a target in order to collect
  information, such as the speed and direction of the target. This is necessary
  as the turret needs some information in order to determine when and where to
  fire at.
  \item The turret should have a maximum firing range of atleast 250cm. This
  number is determined by the maximum sensor range as well as the maximum of
  the design. It can be argued that it could be useful to shoot further than
  what the turret can see, but this is not a necessary feature.
  \item The turret needs to be able to reliably hit a moving target with a
  constant speed atleast 60\% of the time. The turret needs to be able to hit a
  moving target, it is not enough to hit the target sometimes, the turret should
  be able to hit it reliably. The speed should be fast enough to cause the
  turret to miss it, if the turret fires once it is in front of it.
  \item The turret needs to be able to predict where the target will be at a given point
in time. This is necessary as it would otherwise be very unlikely that the
turret would be able to hit the target.
  \item The turret needs to be able to determine the direction, distance and speed of
the target. This requirement is necessary as it would otherwise be impossible to
calculate and predict where the target will be.
\end{itemize}

\subsection{Nice to have}
The following requirement have been deemed to be less important to the overall
success of the project:
\begin{itemize}
  \item The turret needs to be able to reliably hit a moving target with a
variable speed atleast 60\% of the time. This requirement is for a
more realistic scenario, as it is much more likely for a living target to
change its movement speed. The turret is able to reliably hit a target with a
variable speed.
  \item The turret is able to prioritize between several targets. In a realistic
scenario it would be possible for more than one target to be present. This would
then lead to a need to prioritize the targets, in order to maximize the
amount of targets hit.
\end{itemize}

\subsection{Target Requirements}
Blabla some target requirements

\begin{itemize}
  \item The target should be atleast \fix{10cm*10cm}{what is the optimal size?}, as it
would otherwise be too small to be reliably seen by the turret sensors.
  \item The target is able to move at a constant speed of up to 0.8m/s. In order to
have a suitable ratio between the turrets maximum range and the targets
speed, 0.8m/s was chosen as the speed limit.
  \item The target is able to accelerate and decelerate. In order to make for a
  more realistic scenario, the target should be able to speed up and slow down.
  \item The target needs to travel on an even plane.
\end{itemize}

\section{Michaels kravspec ;)}
The turret needs to be able to observe an area
The length of the area is 250 cm since this is the max range that our sensors function at
The turret will be able to search an an angle of 180 degrees

The turret needs to be able to detect a target entering the area

The turret needs to be able to determine the speed and direction of a target
Observing a target should yield information about the direction and speed of the target

The turret needs to be able to predict the location of the target
In order to predict the target, it will need information about the speed and direction of the target

The turret needs to be able to shoot and hit a target in motion
Be able to change position horizontally and vertically in order to adjust the trajectory of the projectile in order to hit a target at varying lengths and angles
The turret should be able to hit anything withing its “area”

The turret needs to be able to function autonomously, without input from an user
