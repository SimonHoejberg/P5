\section{Features and Requirements}\label{FeatAndReq}
In this section the features and requirements for this project will be formed,
this is done based on the analysis in \autoref{PlatformC} and
\autoref{SensorTest}.

The following features and requirements where defined for the project:
\begin{itemize}
  \item The turret needs to be an embedded system and incorporate elements from
  the fields of machine intelligence and real time systems. This leads to the
  following requirement:
  \begin{itemize}
    \item The system needs to use the Lego NXT as the embedded
    system. This platform is chosen based on the findings in
    \autoref{PlatformC}. 
  \end{itemize}
  \item The Turret needs to be able to hit a moving target. This leads to the
  following requirements:
  \begin{itemize}
    \item Since the turret should monitor an area and be able to hit a moving
    target, it needs to be able to turn. If it is not able to turn, line of
    sight as well as the timing in order to hit a target will be severely
    limited. In regards to how far it should be able to turn, a minimum is set
    to 90 degrees, as this will provide a decent sized area in front of the turret.
    \item In order to perceive the target without receiving information from an
    outside source, the turret will need to be able to sense the target. Based
    on the results from the sensor tests, see \autoref{SensorTest}, the sensors
    will be able to see approximately 200 cm.    
    \item The turret needs to be able to shoot in order to hit a
    target. This has to be done within a range of 200 cm as it is the max
    capability of the sensors. 
    
%     \fix{The turret should have a maximum firing range of atleast 250cm.
%     This number is determined by the maximum sensor range as well as the maximum of
%     the design. It can be argued that it could be useful to shoot further than
%     what the turret can see, but this is not a necessary feature.}{Needs to be
%     updated/changed to the above.}
  \end{itemize}  
  \item The target needs to move fast enough such that the turret is unable to
  hit it, without calculating a firing offset. This leads to the following
  requirements:
  \begin{itemize}
    \item In addition to the target moving fast enough to require a firing
    offset, the target also needs to move on an even plane and with a constant
    speed. These two additional requirements are needed to ensure that there are
    as few factors as possible that are not accounted for in the scenario.
    \item The turret needs to be able to calculate the firing offset to ensure
    a high accuracy. It is not enough to hit the target sometimes, the turret
    should be able to hit it reliably. It should be able to hit a moving target
    at least 80\% of the time.
  
  \end{itemize}
  \end{itemize}
  
%   \subsection{Need to have}
% The following requirements have been deemed necessary for an autonomous turret:
% \begin{itemize}
%   \item The turret needs to be an embedded system and incorparate elements from
%   the courses MI and RTS. This is necessary as it is required by the semester.
%   \item The turret needs to be able turn a minimum of 90 degrees. Since the
%   turret should monitor an area and be able to hit a moving target, it needs to
%   be able to turn. If it is not able to turn, line of sight as well as the
%   timing in order to hit a target will be severely limited. In regards to how
%   far it should be able to turn, a minimum is set to 90 degrees, as this will
%   provide a decent sized area in front of the turret.
%   \item The turret needs to be able to sense a target at a range of atleast
%   250cm, since the \fix{current available sensors have a maximum range of
%   255cm}{Should be changed depending on max range of optical}, as long as the
%   lighting is good and the target is suitable.
%   \item The turret needs to be able to search for targets, as otherwise it would
%   have to wait for a target to move into its line of sight. This would limit
%   the window of time the turret has in order to hit the target.
%   \item The turret should have a maximum firing range of atleast 250cm. This
%   number is determined by the maximum sensor range as well as the maximum of
%   the design. It can be argued that it could be useful to shoot further than
%   what the turret can see, but this is not a necessary feature.
%   \item The turret needs to be able to reliably hit a moving target with a
%   constant speed atleast 60\% of the time. The turret needs to be able to hit a
%   moving target, it is not enough to hit the target sometimes, the turret should
%   be able to hit it reliably. The speed should be fast enough to cause the
%   turret to miss it, if the turret fires once it is in front of it.
%   \item The turret needs to be able to predict where the target will be at a given point
% in time. This is necessary as it would otherwise be very unlikely that the
% turret would be able to hit the target.
%   \item The turret needs to be able to determine the direction, distance
%   and speed of the target. This requirement is necessary as it would otherwise be impossible to
% calculate and predict where the target will be.
%   \item The target should be atleast \fix{10cm*10cm}{what is the optimal size?}, as it
% would otherwise be too small to be reliably seen by the turret sensors.
%   \item The target is able to move at a constant speed of up to 0.8m/s. In order to
% have a suitable ratio between the turrets maximum range and the targets
% speed, 0.8m/s was chosen as the speed limit.
%   \item The target needs to travel on an even plane.
% \end{itemize}
% 
% \subsection{Nice to have}
% The following requirement have been deemed to be less important to the overall
% success of the project:
% \begin{itemize}
%   \item The turret needs to be able to reliably hit a moving target with a
% variable speed atleast 60\% of the time. This requirement is for a
% more realistic scenario, as it is much more likely for a living target to
% change its movement speed. The turret is able to reliably hit a target with a
% variable speed.
%   \item The turret is able to prioritize between several targets. In a realistic
% scenario it would be possible for more than one target to be present. This would
% then lead to a need to prioritize the targets, in order to maximize the
% amount of targets hit.
%   \item The target is able to accelerate and decelerate. In order to make for a
%   more realistic scenario, the target should be able to speed up and slow down.
% \end{itemize}
%   
