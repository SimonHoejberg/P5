\chapter{Angle Test}\label{angleTest}
The turret has one main variable it can change in order to accurately hit the
target, namely the angle of firing. Due to the importance of being able to
precisely set the angle of firing, tests have been made to ensure that the
turrets accuracy is within an acceptable margain of error.

\subsection{Test Setup}
The setup for this test was to use a protractor to measure the angle of the
turret in relation to the horizontal. In order to have a stable and
consistant position to measure the angle from, a lego structure was added to the
turrets base. As the turret is shifted in relation to the pivot point, a lego
brick was added which connects to the pivot point and is parallel to the turret.
This brick is used to measure the turrets angle. This setup can be seen in
\autoref{AngleTestSetup}.

\figx[0.13]{AngleTestSetup}{Setup for testing the turrets angle.}

In order to test the turrets ability to set its angle correctly, two different
tests were used to test the most likely scenarios. The first test has the turret
retract to its resting position before setting the angle of firing. The second
test has the turret set the angle of firing without retracting to the resting
position before the next test. For these tests the resting position is measured
to be 5 degrees.

\subsection{Results: Resting Position}
This test had the turret retract to its resting position before setting the
specified angle of fire. The data presented below in \autoref{AngleTest1Result}
is the average of five repetitions.

\begin{table}[H]
\centering
\begin{tabular}{ll}
\multicolumn{1}{l|}{Expected} & Actual \\ \hline
\multicolumn{1}{l|}{7}        & 7     \\
\multicolumn{1}{l|}{14}       & 13     \\
\multicolumn{1}{l|}{21}       & 20     \\
\multicolumn{1}{l|}{28}       & 27     \\
\multicolumn{1}{l|}{35}       & 34     \\
\multicolumn{1}{l|}{42}       & 42     \\
\multicolumn{1}{l|}{49}       & 49     \\
\multicolumn{1}{l|}{56}       & 55     \\
\multicolumn{1}{l|}{63}       & 64     \\
\multicolumn{1}{l|}{70}       & 72
\end{tabular}
\caption{Test results from resetting the position between tests.}
\label{AngleTest1Result}
\end{table}

From these data we can determine that the maximum difference in angles is 3
degrees.  

\subsection{Results: Dynamic Position}
This test had the turret set the angle as an offset based on the previously set
angle. This means that the turret is not reset to its resting position between
tests. This is done in order to determine the turrets ability to dynamically and
accurately set its angle of fire. The data presented below in
\autoref{AngleTest2Result} is the average from 5 repetitions.

\begin{table}[H]
\centering
\begin{tabular}{ll}
Test Num:                     & 4      \\ \hline
\multicolumn{1}{l|}{Expected} & Actual \\ \hline
\multicolumn{1}{l|}{7}        & 9     \\
\multicolumn{1}{l|}{14}       & 16     \\
\multicolumn{1}{l|}{21}       & 23     \\
\multicolumn{1}{l|}{28}       & 29     \\
\multicolumn{1}{l|}{35}       & 35     \\
\multicolumn{1}{l|}{42}       & 43     \\
\multicolumn{1}{l|}{49}       & 49     \\
\multicolumn{1}{l|}{56}       & 57     \\
\multicolumn{1}{l|}{63}       & 64     \\
\multicolumn{1}{l|}{70}       & 71
\end{tabular}
\caption{Test results from dynamically setting the angle.}
\label{AngleTest1Result}
\end{table}

From these data we can determine that the maximum difference in angles is 2
degrees, which is slightly more accurate than the previous test.

\subsection{Conclusion: Angle Test}
From the 10 tests we can conclude that the turret capabilities when it comes to
setting the correct angle are satisfactory. While there is a difference of up to
3 degrees between the expected and actual angle, we conclude that this is well
within the acceptable variance, especaially since the turret will not have to
hit a target over a longer distance.   