\chapter{Target Speed Test}\label{speedotesto}
In order to test the speed of the target a test was devised.
hvad er det for en test og vhvad vil vi have ud af det
\subsection{Test setup}
First a track was constructed. The track is simply a 5 meter long line of chalk,
with a indicator for every meter on the line.
\subsection{Test execution}
The target is then turned on the speed setting to be tested. Using a stopwatch
the target starts on the track and everytime it passes a meter indicator, we note
the mean-time. For a track with a length of 5 meters, we get 5 measuring points indicating
the time in seconds it took for the target to travel the distance.Using this data
we can also calculate the speed in meters per second and an average of time it takes to
travel a meter.
\subsection{Test results}
\fix{}{should this not be inside testing chapter}

\dataTable{speed}{Target speed results}{speed}
\dataTable{avgSpeed}{Target speed in m/s}{speedMS}

% \begin{table}[ht]
% \caption{} % title of Table
% \centering % used for centering table
% \begin{tabular}{c c c c} % centered columns (4 columns)
% \hline\hline %inserts double horizontal lines
% Meters & Speed 50 & Speed 75 & Speed 100 \\ [0.5ex] % inserts table
% %heading
% \hline % inserts single horizontal line
% 1 & 2.72 & 2.1 & 970 \\ % inserting body of the table
% 2 & 2.86 & 1.72 & 230 \\
% 3 & 2.88 & 1.79 & 415 \\
% 4 & 2.88 & 1.88 & 2356 \\
% 5 & 2.95 & 1.93 & 556 \\ [1ex] % [1ex] adds vertical space
% \hline %inserts single line
% Speed(m/s) & 0.3498950315 & 0.53078556 & 0.75987841 \\
% \end{tabular}
% \label{table:speed} % is used to refer this table in the text
% \end{table}
