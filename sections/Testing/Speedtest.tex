\chapter{Target Speed Test}\label{speedotesto}
Following the design of the target in \autoref{targetDesign}, a test was
conducted in order to determine the targets ability to move. This test tries to
answer two questions about the target, namely:
\begin{itemize}
  \item Does the target move at a constant speed?
  \item Does the target move in a straight line?
\end{itemize}

These questions will be answered as the requirements in \autoref{FeatAndReq}
state that the target needs to move at a constant speed, and because calculating
future positions require the target to move in a straight line.

\section{Test Setup}
A chalk line was marked on the floor for every meter, and a stopwatch was used
to measure the time. The test was conducted by setting the speed and placing the
target on the floor, turning it on, and letting it run for 5 meters.

\section{Test Execution}
The target is turned on, and the speed is set to the specified value for the
test. The target is then placed on the floor, and the stopwatch is started.
Every time the target passes one of the meter indicators, the mean-time is
noted.
For a track with a length of 5 meters, we get 5 measuring points indicating the
time in seconds it took for the target to travel that distance. Using this data
we can calculate the speed in meters per second, and an average of time it
takes to travel a meter.

\section{Test Results}
The results from the test can be seen below in \autoref{table:speed}, and the
resulting average speed is shown in \autoref{table:speedMS}. 

\dataTable{speed}{Target speed results}{speed}
\dataTable{avgSpeed}{Target speed in m/s}{speedMS}

The data in \autoref{table:speed} indicates that speed is more or less constant,
or at least that there is no noticeable acceleration. While it appears as if
there is an acceleration at 50\% speed, this is not shown at the other two speeds.

\section{Conclusion}
Based on the collected data, and observations made during the test, we can
conclude that the target almost at a near constant speed, and that the target
moves in a straight line. While the data does not support the conclusion that
the target moves at a constant speed, any observed acceleration is so small,
that we conclude that it does not have an impact on the turrets ability to
accurately hit the target.


% \begin{table}[ht]
% \caption{} % title of Table
% \centering % used for centering table
% \begin{tabular}{c c c c} % centered columns (4 columns)
% \hline\hline %inserts double horizontal lines
% Meters & Speed 50 & Speed 75 & Speed 100 \\ [0.5ex] % inserts table
% %heading
% \hline % inserts single horizontal line
% 1 & 2.72 & 2.1 & 970 \\ % inserting body of the table
% 2 & 2.86 & 1.72 & 230 \\
% 3 & 2.88 & 1.79 & 415 \\
% 4 & 2.88 & 1.88 & 2356 \\
% 5 & 2.95 & 1.93 & 556 \\ [1ex] % [1ex] adds vertical space
% \hline %inserts single line
% Speed(m/s) & 0.3498950315 & 0.53078556 & 0.75987841 \\
% \end{tabular}
% \label{table:speed} % is used to refer this table in the text
% \end{table}
