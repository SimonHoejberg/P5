\section{Unit test}

To test the software we decided to make unit tests, however since there is no
framework for making this kind of test we had to make our own. The setup is
quite simple and is build on the same principle as unit testing in fx. C\#.
However, when testing an object oriented language, the testing is done on class
level this gives a clear distinction between the different methods. This however
is not the case in NXC. In our testing setup, only functions with a return
value is tested and functions which uses input from the sensors are not
tested. The basic principle is Arrange, Act and Assert. In the arrange step the
input data is constucted as well as what the expected output is. In the act
step, the functions are called and the inputs are saved. Then finaly in the
assert step the outputs are verified arcording to the expected output. 

\begin{minipage}[H]{\linewidth}
\begin{lstlisting}{caption = "", label = ""}
bool CombineVectorsTest(){
  DirectionVector input1;
  DirectionVector input2;
  input1.speedX = 10;
  input1.speedY = 10;
  input1.xPos = 15;
  input1.yPos = 15;
  
  input2.speedX = 25;
  input2.speedY = 25;
  input2.xPos = 20;
  input2.yPos = 20;
  DirectionVector dirVecArray[2];
  dirVecArray[0] = input1;
  dirVecArray[1] = input2;
  DirectionVector out = CombineVectors(dirVecArray, 2);
  
  if(out.speedX == 22.5 && out.speedY == 22.5 && out.xPos == 20, out.yPos == 20){
    return true;
  }
  else {
    return false;
  }
}
\end{lstlisting}
\end{minipage}
