\chapter{Unit test}

To test the \name software we decided to make unit tests, however since have no
knowledge of any integrated framework for unit testing in the NXT IDE, we
have chosen to make our own. The setup is quite simple and is build on the same
principle as unit testing in fx. C\#. When testing an object oriented
language, the testing is done on class by class level as this gives a clear
distinction between the different methods. This however is not the case in NXC,
as the language follows the imperative programming paradigm. In our testing
setup, only functions with a return value is tested and functions which takes input from
sensor inside of the function are not tested, also functions that controls
motors are not tested in this chapter those are tested by condunting tests in
the physical world.
The basic principle is Arrange, Act and Assert. In the arrange step the input
data is constucted as well as what the expected output is. In the act step, the
functions are called and the inputs are saved. Then finaly in the assert step
the outputs are verified arcording to the expected output. \nl

\begin{minipage}[H]{\linewidth}
\begin{lstlisting}[caption =Unit Test for Combine Vectors,label=combineVectorsTest] 
bool CombineVectorsTest(){
  DirectionVector input1;
  DirectionVector input2;
  input1.speedX = 10;
  input1.speedY = 10;
  input1.xPos = 15;
  input1.yPos = 15;
  
  input2.speedX = 25;
  input2.speedY = 25;
  input2.xPos = 20;
  input2.yPos = 20;
  DirectionVector dirVecArray[2];
  dirVecArray[0] = input1;
  dirVecArray[1] = input2;
  DirectionVector out = CombineVectors(dirVecArray, 2);
  
  if(out.speedX == 22.5 && out.speedY == 22.5 && out.xPos == 20, out.yPos == 20){
    return true;
  }
  else {
    return false;
  }
}
\end{lstlisting}
\end{minipage}

The function we want to test are

\begin{itemize}
  \item \textc{CalcDirVector}
  \item \textc{UpperBound}
  \item \textc{CheckLength}
  \item \textc{GetSpeed}
  \item \textc{MakePosData}
  \item \textc{CombineVectors}
  \item \textc{CalcFuturePos}
  \item \textc{CalcFireData}
\end{itemize} 

In \autoref{combineVectorsTest} the basic setup can be seen, on \textbf{Line
2-15} the needed information is arranged, however the expected output is not
defined yet. This is defined in the assert part as this was the easiest way to
do it. On \textbf{Line 16}, the tested function is called and in this case the
output is put into a struct called DirectionVector. Then finaly we use an if
statement to check the output against the expected and if the output coresponds
to the expected we return true. This return value is then used to print out a
passed or failed to a log file, which can be seen in \autoref{table:ut_result}.\nl

\dataTable{Results}{The result of the unit test}{ut_result}

To conclude all the tests succeeded, one of the tests we had some issues as our
calculations differed a little bit from the calculations made by the NXT. These
differences was so small that they were likely caused by a rounding error.
The rounding error is about 1 degree and less then a milisecond in the case of
CalcFireData. As the accuracy of the varibles in this part
of the program is not deemed essential, we choose to call it a success.
