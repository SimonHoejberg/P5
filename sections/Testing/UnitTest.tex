\chapter{Unit test}

To test the \name software we decided to make a unit test, however since we
have no knowledge of any integrated framework for unit testing in Bricx Command
Center \cite{BricxCC}, the IDE for NXC, we have chosen to make our own. The
setup is quite simple and is built on the same principle as unit testing in for example C\#. When testing an
object oriented language, the testing is done on a class by class level as this
gives a clear distinction between the different methods. This is however not
the case in NXC, as the language follows the imperative programming paradigm.
In our testing setup, only functions with return values are tested. Functions
which calls the sensors are not tested, in addition functions that controls the
motors are not tested, as these are tested implicitly during the testing of the
hardware.

\section{Testing Model}
The basic principle is \textbf{Arrange}, \textbf{Act} and \textbf{Assert}. In
the arrange step the input data is constructed and the expected
output is degined. In the act step, the functions are called and the outputs are
saved. Then finally in the assert step the outputs are verified according to the
expected output. \nl

\begin{minipage}[H]{\linewidth}
\begin{lstlisting}[caption =Unit Test for
\texttt{CombineVectors},label=combineVectorsTest] 
bool CombineVectorsTest(){
  DirectionVector input1; DirectionVector input2;
  input1.speedX = 10;
  input1.speedY = 10;
  input1.xPos = 15;
  input1.yPos = 15;
  
  input2.speedX = 25;
  input2.speedY = 25;
  input2.xPos = 20;
  input2.yPos = 20;
  DirectionVector dirVecArray[2];
  dirVecArray[0] = input1;
  dirVecArray[1] = input2;
  DirectionVector out = CombineVectors(dirVecArray, 2);
  
  if(out.speedX == 22.5 && out.speedY == 22.5 && out.xPos == 20, out.yPos == 20){
    return true;
  }
  else {
    return false;
  }
}
\end{lstlisting}
\end{minipage}

The tested functions are:

\begin{itemize}
  \item \textc{CalcDirVector}
  \item \textc{UpperBound}
  \item \textc{CheckLength}
  \item \textc{GetSpeed}
  \item \textc{MakePosData}
  \item \textc{CombineVectors}
  \item \textc{CalcFuturePos}
  \item \textc{CalcFireData}
\end{itemize} 

In \autoref{combineVectorsTest} the basic setup can be seen. On \textbf{Line
2-15} the needed information is arranged, however the expected output is not
defined yet. This is defined in the assert part as this was the easiest way to
do it. On \textbf{Line 16}, the tested function is called and in this case the
output is put into a struct called \texttt{DirectionVector}. Then finally we use
an if statement to check the output against the expected and if the output
corresponds to the expected we return true. This return value is then used to
print out whether or not the tested function has passed to a log file, which can
be seen in \autoref{table:ut_result}.\nl

\dataTable{Results}{The result of the unit test}{ut_result}

\section{Conclusion}
In conclusion all the tests succeeded, in one of the tests we had some
issues as our calculations differed slightly from the calculations made by the
NXT. These differences were so small that they were likely caused by a rounding
error. The rounding error is about 1 degree and less then a millisecond in the
case of \texttt{CalcFireData}. As the accuracy of the variables in this part
of the program are not deemed essential, we still consider it to have passed.
