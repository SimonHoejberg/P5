\chapter{Testing}
During the development of the automated turret, a number of tests have been made
in order to document that the turrets features functions as intended. This
chapter will be use to present these tests and their related data, and to
discuss to which degree these results are acceptable according the the product
requiremtents.

\section{Angle Test}
As discussed in \fix{\autoref{}}{Jones} the turret has one main variable it can
change in order to accurately hit the target, namely the angle of firing. Due to the
importance of being able to precisely set the angle of firing, tests have been
made to ensure that the turrets accuracy is within an acceptable margain of
error.

\subsection{Test Setup}
The setup for this test was to use a protractor to measure the angle of the
turret in relation to the horizontal. In order to have a stable and
consistant position to measure the angle from, a lego structure was added to the
turrets base. As the turret is shifted in relation to the pivot point, a lego
brick was added which connects to the pivot point and is parallel to the turret.
This brick is used to measure the turrets angle. This setup can be seen in
\autoref{AngleTestSetup}.

\figx[0.13]{AngleTestSetup}{Setup for testing the turrets angle.}

In order to test the turrets ability to set its angle correctly, two different
tests were used to test the most likely scenarios. The first test has the turret
retract to its resting position before setting the angle of firing. The second
test has the turret set the angle of firing without retracting to the resting
position before the next test. For these tests the resting position is measured
to be 5 degrees.

\subsection{Results: Test 1}
This test had the turret retract to its resting position before setting its
angle of fire. Below will be presented the data from data set 2. The rest
of the data can be found in \fix{\autoref{}}{JONES}.

\begin{table}[H]
\centering
\begin{tabular}{ll}
Test Num:                     & 2      \\ \hline
\multicolumn{1}{l|}{Expected} & Actual \\ \hline
\multicolumn{1}{l|}{7}        & 10     \\
\multicolumn{1}{l|}{14}       & 17     \\
\multicolumn{1}{l|}{21}       & 23     \\
\multicolumn{1}{l|}{28}       & 29     \\
\multicolumn{1}{l|}{35}       & 45     \\
\multicolumn{1}{l|}{42}       & 43     \\
\multicolumn{1}{l|}{49}       & 50     \\
\multicolumn{1}{l|}{56}       & 57     \\
\multicolumn{1}{l|}{63}       & 64     \\
\multicolumn{1}{l|}{70}       & 77
\end{tabular}
\caption{Results from data set 2.}
\label{AngleTest1Result}
\end{table}

\subsection{Results: Test 2}
This test had the turret set the angle as an offset based on the previously set
angle. Below will be presented the data from data set 4.
The rest of the data can be found in \fix{\autoref{}}{DAMN}.

\begin{table}[H]
\centering
\begin{tabular}{ll}
Test Num:                     & 4      \\ \hline
\multicolumn{1}{l|}{Expected} & Actual \\ \hline
\multicolumn{1}{l|}{7}        & 7     \\
\multicolumn{1}{l|}{14}       & 13     \\
\multicolumn{1}{l|}{21}       & 20     \\
\multicolumn{1}{l|}{28}       & 28     \\
\multicolumn{1}{l|}{35}       & 34     \\
\multicolumn{1}{l|}{42}       & 40     \\
\multicolumn{1}{l|}{49}       & 48     \\
\multicolumn{1}{l|}{56}       & 56     \\
\multicolumn{1}{l|}{63}       & 64     \\
\multicolumn{1}{l|}{70}       & 75
\end{tabular}
\caption{Results from data set 4.}
\label{AngleTest1Result}
\end{table}

\chapter{Target Speed Test}\label{speedotesto}
Following the design of the target in \autoref{targetDesign}, a test was
conducted in order to determine the tagets ability to move. This test tries to
answer two questions about the target, namely:
\begin{itemize}
  \item Does the target move at a constant speed?
  \item Does the target move in a straight line?
\end{itemize}

These questions will be answered as the requirements in \autoref{FeatAndReq}
state that the target needs to move at a constant speed, and because calculating
future positions require the target to move in a straight line.

\subsection{Test setup}
The is test was conducted by placing the target on the floor, turning it on, and
letting it run for 5 meters. A chalk line was marked on the floor for every
meter, and a stopwatch was used to measure the time.

\subsection{Test execution}
The target is turned on, and the speed is set to the specified value for the
test. The target is then placed on the floor, and the stopwatch is started.
Everytime the target passes one of the meter indicators, the mean-time is noted.
For a track with a length of 5 meters, we get 5 measuring points indicating the
time in seconds it took for the target to travel that distance. Using this data
we can calculate the speed in meters per second, and an average of time it
takes to travel a meter.

\subsection{Test results}
The results from the test can be seen below in \autoref{table:speed}, and the
resulting average speed is shown in \autoref{table:speedMS}. 

\dataTable{speed}{Target speed results}{speed}
\dataTable{avgSpeed}{Target speed in m/s}{speedMS}

The data in \autoref{table:speed} indicates that more or less constant, or at
least that there is no consistant acceleration. While it appears as if there is
an acceleration at 50\% speed, this is not shown at the other two speed
thresholds.

\section{Conclusion}
Based on the collected data, and observations made during the test, we can
conclude that the target almost moves at a constant speed, and that the target
moves in a straight line. While the data does not support the conclusion that
the target moves at a constant speed, any observed acceleration is so small,
that we conclude that it does not have an impact on the turrets ability to
accurately hit the target.


% \begin{table}[ht]
% \caption{} % title of Table
% \centering % used for centering table
% \begin{tabular}{c c c c} % centered columns (4 columns)
% \hline\hline %inserts double horizontal lines
% Meters & Speed 50 & Speed 75 & Speed 100 \\ [0.5ex] % inserts table
% %heading
% \hline % inserts single horizontal line
% 1 & 2.72 & 2.1 & 970 \\ % inserting body of the table
% 2 & 2.86 & 1.72 & 230 \\
% 3 & 2.88 & 1.79 & 415 \\
% 4 & 2.88 & 1.88 & 2356 \\
% 5 & 2.95 & 1.93 & 556 \\ [1ex] % [1ex] adds vertical space
% \hline %inserts single line
% Speed(m/s) & 0.3498950315 & 0.53078556 & 0.75987841 \\
% \end{tabular}
% \label{table:speed} % is used to refer this table in the text
% \end{table}

