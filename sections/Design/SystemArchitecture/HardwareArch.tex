\section{Hardware Architecture}
<<<<<<< .mine
In the previous chapter the physical design of the turret was presented and
explained. As such, this section will be used to present the technical aspects
of the turret design, namely the hardware architecture. The \name system makes
use of a number of hardware components, which are used for varying
functionality. These components are:

\begin{itemize}
  \item 3 NXT Interactive Servo Motors
  \item 2 NXT Ultrasonic Distance Sensors
  \item 1 NXTCam-v4 Visual Sensor
  \item 1 NXT Intelligent Brick
  \item 6 RJ12 Connector Cables
\end{itemize}

In this setup all components are connected to the NXT Brick, and communicate
through cables. A single servo motor is used to for the fireing mechanism, one
is used to rotate the turret horizontally and one is used to angle the
ball-launcer part of the turret. The ultrasonic sensors are used to gather data
about the distance to the target, and the camera collects data about the
relative position of a target within its vision. A view of the hardware
components can be seen below in \autoref{HARCH}.

\figx{HARCH}{Figure showing the hardware conncetions within the \name turret.}

In this setup all calculations are done by the NXT, which in turn is responsible
for controlling all other hardware components by communicating through RJ12
connector cables.
=======
In the previous chapter the physical design of the turret was presented and
explained. As such, this section will be used to present the technical aspects
of the turret design, namely the hardware architecture. The \name system makes
use of a number of hardware components, which are used for varying
functionality. These components are:

\begin{itemize}
  \item 3 NXT Interactive Servo Motors
  \item 2 NXT Ultrasonic Distance Sensors
  \item 1 NXTCam-v4 Visual Sensor
  \item 1 NXT Intelligent Brick
  \item 6 RJ12 Connector Cables
\end{itemize}

In this setup all components are connected to the NXT Brick, and communicate
through cables. A single servo motor is used to for the fireing mechanism, one
is used to rotate the turret horizontally and one is used to angle the
ball-launcer part of the turret. The ultrasonic sensors are used to gather data
about the distance to the target, and the camera collects data about the
relative position of a target within its vision. A view of the hardware
components can be seen below in \autoref{HARCH}.

\figx{HARCH}{Figure showing the hardware conncetions within the \name turret.}

In this setup all calculations are done by the NXT, which in turn is responsible
for controlling all other hardware components by communicating through RJ12
connector cables.
>>>>>>> .r583
