\section{Software Architecture}
When designing the software for the turret, we decided to split the program
into four tasks, which each control a part of the turret's behaviour. These are
listed below:

\begin{itemize}\label{softArchOver}
  \item \textc{Main} - Initializes sensors and starts \textc{Track} and
  \textc{GetDataAndCalculate}.
  \item \textc{Track} - Reads input from the camera, and rotates the turret to
  follow the target.
  \item \textc{GetDataAndCalculate} - Reads inputs from the distance sensors,
  and calculates a future position of the target.
  \item \textc{WaitAndFire} - Receives data from \textc{GetDataAndCalculate},
  positions the turret and fires at the correct time.
\end{itemize}

In order to present an overview of the explained program flow, a sequence
diagram is shown below in \autoref{SequenceDiagram}.

\figx{SequenceDiagram}{Flow of the \name software.}
\newpage
In addition to the sequence diagram, a model has been made which presents how
different tasks and methods can call each other. This is shown in
\autoref{CallStack}, where it can be seen how the calls in the software are
made, and what tasks make use of which functionality. The software has been
implemented to be modular, such that each task has its own variables and
methods, which for the most part are not shared between tasks.\\
It should be noted, that this model can not take into account the firmware
implemented on the various sensors. This is the case, as this software is not
well documented, and as such, we do not know how it functions.

\figx[0.6]{CallStack}{The method/task call structure of the \name software.}
