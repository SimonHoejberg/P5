\section{Software Architecture}
When designing the software for the turret, we decided to split the program
into four tasks, which each control a part of the turret's behaviour. These are
listed below in \autoref{softArchOver}.

\begin{itemize}\label{softArchOver}
  \item \textc{Main} - Initializes sensors and starts \textc{Track} and
  \textc{GetDataAndCalculate}.
  \item \textc{Track} - Read input from camera, and rotate turret to follow the
  target.
  \item \textc{GetDataAndCalculate} - Read input from distance sensor, and
  calculate future position of the target.
  \item \textc{WaitAndFire} - Recieves data from \textc{GetDataAndCalculate},
  positions the turret and fires at the correct time.
\end{itemize}

In order to present an overview of the explained program flow, a sequence
diagram is shown below in \autoref{SequenceDiagram}.

\figx{SequenceDiagram}{Flow of the \name software.}

In addition to the sequence diagram, a model has been made which presents how
different tasks and methods can call each other. This is shown in
\autoref{CallStack}, where it can be seen how the calls in the software are
made, and what tasks make use of which functionality. The software has been
implemented in a modular fashion such that each task has its own varibles and
methods, which for the most part are not shared between tasks.\nl

\figx[0.6]{CallStack}{The method/task call structure of the \name software.}

It should be noted, that this model can not take into account the firmware
implemented on the various sensors. This is the case, as this software is not
well documented, and as such, we do not know how it functions.
