\chapter{Representation Scheme}
When tracking the target \name needs data in order to do so. This data comes
from the camera and ultrasonic sensors in the form of a distance and an angle.
This data is relative to the turret, which means that in coordinate system it
would be in (0,0). To get the current position of the target an x and a y
coordinate is calculated by multiplying the distance with $sin(a)$ to get the x coordinate and
$cos(a)$ to get the y coordinate, where a is the angle. This is because \name is
centered around the y-axis. When The turret has a number of points, a vector is
calculated to get the distance that the target has travelled. This vector is
then divided by the time it takes the target to travel the distance, to get a new vector that represnts the
velocity of the target in $\frac{cm}{ms}$. This vector is then multiplied by a
number of miliseconds to get a point where the target is going to be in the
future. To get the angle to this point the following formulas are used:
$a_{positive}=90+(tan^{-1}(\frac{y_{pos}}{x_{pos}}))*(-1)$ and
$a_{negative}=(90-(tan^{-1}(\frac{y_{pos}}{x_{pos}})*(-1))*(-1)$. The reason for
using two different formulas to calculate the angle is that the formulas output
the angle that \name should turn to and not how many degrees it should turn.
this means that the first formula only gives the right output when the point is
in the positive half of the coordinate system and the second only gives the
right output when the point is in the negative half.

When \name has turned to the desired angle it needs to angle itself vertically
based on the distance to the target. This distance is calculated using
pythagoras theorem. To calculate the angle the following function is used:
$a_{vertical}=| 0.2573*distance-22.85 |$.
This function comes from a trendline that was extrapolated from a set of test
data, where the angle was plotted against how far the projectile travelled.
\name also needs to know when to shoot in order to hit the target this is
calculated by subtracting the value of \autoref{TimeEq} which is described in
\autoref{ProjMotion}, from the time it takes the target to get to the desired
point, which is calculated from the velocity vector. This gives the time where
\name needs to shoot to be able to hit the target.
