\section{Target Tracking Model}\label{RepScheme}
As \name needs to track the location of an object in a three dimensional world,
it needs a consistant way to represent this data. This data comes in the form
of images from the camera, distance readings from the ultrasonic sensors and
the turrets relative angle, which is read from the motors. This is shown
below in \autoref{SensorsVsDataType}.

\begin{table}[H]
\centering
\begin{tabular}{|p{2cm}|p{12cm}|}
\hline
Sensor     & Data/Unit                                                                                                                                        \\ \hline
Camera     & The frame is a coordinate system of pixels, and the data is given as the coordinates of the edges of the recognized rectangular blobs of colour. \\ \hline
Ultrasonic & Data is an integer between 1 and 255 centimeters.                                                                                                \\ \hline
Motor      & Angle of the motor based on its starting position. Data is given in degrees.                                                                     \\ \hline
\end{tabular}
\caption{Sensors and their respective data types.}
\label{SensorsVsDataType}
\end{table}

Given that all of this data is relative to the turret location, we can create a
representation scheme where the turret is the center of a 2-dimensional
coordinate system. An example of this can be seen below in
\autoref{CoordinateSystem2}.\nl

By using the distance to the target and the angle of the turret, we are able to
calculate the location of the target in our 2-dimensional coordinate system.
As position is given by an (x,y) set of coordinates, we can calculate its
position using sine and cosine, as shown in \autoref{locEq1}.

\begin{equation}\label{locEq1}
x=sin(a)*dist
\end{equation}
\begin{equation}\label{locEq2}
y=cos(a)*dist
\end{equation} 

In this equation the angle \textc{a} is given in radians, and the distance
\textc{dist} is given in centimeters. Contrary to what might be expected, sine
is used to find the x-coordinate and cosine is used to determine the
y-coordinate. This is because the turrets coordinate system is rotated 90
degrees counter clockwise, as the turrets starting direction is read as an angle
of 0 degrees. This calculation is shown below in \autoref{CoordinateSystem2}.

\figx{CoordinateSystem2}{Sine and cosine are used to calculate positions
in the representation scheme.}

When the turret has registered a number of data points, a vector is
calculated to compute the distance that the target has travelled. This vector is
then divided by the time it takes the target to travel the distance, to get a
new vector that represents the velocity of the target in $\frac{cm}{ms}$. This
vector is then multiplied by a given number of milliseconds, in order to
calculate a point in the coordinate system where the target is going to be in
the future. Normally the tangent could be used to calculate the distance
to this point, but because our coordiante system is rotated 90 degrees, we need two
different equations. For this purpose \autoref{locEq4} is used if the angle is
negative, or \autoref{locEq3} if positive.
\begin{equation}\label{locEq3}
a_{positive}=90+(tan^{-1}(\frac{y_{pos}}{x_{pos}}))*(-1)
\end{equation} 
\begin{equation}\label{locEq4}
a_{negative}=(90-(tan^{-1}(\frac{y_{pos}}{x_{pos}})*(-1))*(-1)
\end{equation} 

These equations are represented below in \autoref{CoordinateSystem3}.

\figx[0.4]{CoordinateSystem3}{Coordinate system with two locations representing
a positive and a negative angle.}

When \name has turned to the desired angle it needs to angle its cannon
vertically based on the distance to the target. Due to the complexity of using
the formula for projectile motion to calculate an angle from a distance, we have
chosen to use data from tests to determine a function. As such, to calculate the
angle the following equation is used:

\begin{equation}\label{locEq4}
a_{vertical}=0.2573*distance-22.85
\end{equation} 

This function based on a best fit line that was extrapolated from a the distance
test in \autoref{AppendixAccTest}, where the angle was plotted against how far
the projectile travelled. This approach was chosen because of the complexity
involved in determining the angle based on the distance. This is shown below in
\autoref{RSEQ}, which is the formula determined in \autoref{ProjMotion}. In this
equation, isolating for the angle \texttt{a} has proven to be quite difficult,
as any attempt has resulted in the inclusion of imaginary numbers.
  
\begin{equation}\label{RSEQ}
x\ =\ \frac{v_0*cos(a)}{g}* \left(
v_0*sin(a)+\sqrt{v_0^2*sin(a)^2+2*g*(h+r*sin(a))}\right)+r*cos(a)
\end{equation}

\name also needs to know when to shoot in order to hit the target.
This is calculated by subtracting the value of \autoref{RSTimeEq}, which is described in
\autoref{ProjMotion}, from the time where the target is at the desired point.
\begin{equation}\label{RSTimeEq}
t=\frac{x}{v_0*cos(a)}
\end{equation}
This gives the time where \name needs to shoot to be able to hit the target.
This is calculated in the function CalcFireData.\nl

In conclusion, by using a combination of this representation scheme and
trigonometry, the turret is capable of modelling the world around it such that
it can accurately determine and predict the location of the target. This
approach of mapping data to determine a function does cause som inaccuracies, as
the resulting function is linear, while the actual function would be a parabola.
While this is a problem, the resulting function is still based on data from our
tests, and as such should be accurate enough for this project.
