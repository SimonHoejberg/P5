\fix{\section{Data Representation}}{Better title: contains more than that: Tracking and positioning?}\label{RepScheme}
As \name needs to track the location of an object in a three dimensional world,
it needs a consistant way to represent this data. This data comes in the form
of \fix{images from the camera}{Måske en table, med sensor navn, data og
enheder}, distance readings from the ultrasonic sensors and the turrets relative
angle, which is read from the motors. Given that all of this data is relative to the turret location, we can create a representation
scheme where the turret is the center of a 2-dimensional coordinate system. An
example of this can be seen below in \autoref{CoordinateSystem}.\nl

\figx[0.35]{CoordinateSystem}{Example of the turret standing in the middle of an
x,y coordinate system.}\fix{}{\autoref{CoordinateSystem2} er tilstrækkelig}

By using the distance to the target and the angle of the turret, we are able to
calculate the location of the target in our 2-dimensional coordinate system.
As position is given by an (x,y) set of coordinates, we can calculate its
position using sine and cosine, as shown in \autoref{locEq1}.

\begin{equation}\label{locEq1}
x=sin(a)*dist
\end{equation}
\begin{equation}\label{locEq2}
y=cos(a)*dist
\end{equation} 

In this equation the angle \textc{a} is given in radians, and the distance
\textc{dist} is given in centimeters. Contrary to what might be expected, sine
is used to find the x-coordinate and cosine is used to determine the
y-coordinate. This is because the turrets coordinate system is rotated 90
degrees counter clockwise, as the turrets starting direction is read as an angle
of 0 degrees. This calculation is shown below in \autoref{CoordinateSystem2}.
\fix{}{Tjah forkommer enklere med at ændre start vinklen}

\figx{CoordinateSystem2}{Sine and cosine are used to calculate positions
in the representation scheme.}

When the turret has registered a number of data points, a vector is
\fix{calculated}{compute} to get the distance that the target has travelled.
This vector is then divided by the time it takes the target to travel the distance, to get a
new vector that represents the velocity of the target in $\frac{cm}{ms}$. This
vector is then multiplied by a given number of milliseconds, in order to
calculate a point in the coordinate system where the target is going to be in
the future. Normally the tangent could be used to calculate the distance
to this point, but because our coordiante system is rotated 90 degrees, we need two
different equations. For this purpose \autoref{locEq4} is used if the angle is
negative, or \autoref{locEq3} if positive.
\begin{equation}\label{locEq3}
a_{positive}=90+(tan^{-1}(\frac{y_{pos}}{x_{pos}}))*(-1)
\end{equation} 
\begin{equation}\label{locEq4}
a_{negative}=(90-(tan^{-1}(\frac{y_{pos}}{x_{pos}})*(-1))*(-1)
\end{equation} 

These equations are represented below in \autoref{CoordinateSystem3}.

\figx[0.4]{CoordinateSystem3}{Coordinate system with two locations representing
a positive and a negative angle.}

When \name has turned to the desired angle it needs to angle its cannon
vertically based on the distance to the target. Due to the complexity of using
the formula for projectile motion to calculate an angle from a distance, we have
chosen to use data from tests to determine a function. As such, to calculate the
angle the following equation is used:

\begin{equation}\label{locEq4}
a_{vertical}=0.2573*distance-22.85
\end{equation} 

\fix{}{what error do you introduce}

\fix{This function based on a best fit line that was extrapolated from a set of
test data, where the angle was plotted against how far the projectile
travelled.}{Need a reference to a test} This function is an estimate of the
function needed in order to get the angle. This is because the exact function
is \fix{too complex for the NXT to calculate}{maybe mention why / refere to
physics} in the available time.\nl

\fix{}{Explain Why! trigonometric fucntions may also be costly}

\name also needs to know when to shoot in order to hit the target.
This is calculated by subtracting the value of \autoref{TimeEq}, which is described in
\autoref{ProjMotion}, from the time where the target is at the desired point.
This gives the time where \name needs to shoot to be able to hit the target.
This is calculated in the function CalcFireData.

In conclusion, by using a combination of this representation scheme and
trigonometry, the turret is capable of modelling the world around it such that
it can accurately determine and predict the location of the target.