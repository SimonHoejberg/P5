\chapter{Design}
In this chapter the different parts of the design of \name will be discussed.
The design of the target will also be described.

\section{Turret Design}
In this section the different parts of \namep's design wil be described. This
includes the mechanism for shooting, the design of the base, sensor placement
and the mechanism for rotating.

\subsection{Shooting Mechanism}
When it comes to the shooting mechanism it needs to be able to shoot at least
200 cm as this is one of the requirements. This means that it a significant
amount of power behind each shot. To make this a reality some different designs
were considered. the first design considered was a balista due to the fact that
a balista is able to deliver a lot of power. This design was quickly dismissed
due to the fact that it is hard to make it able to reload itself, which would
mean that it would only be able to fire a single shot. This was deemed as a
major flaw since if it missed it would not be able to try again. Another design
considered utilizes the same principle as a football launcher, where two wheels
are spinning in opposite directions. This design was not realised due to the
fact that we did not have acces to motors powerfull enough to deliver the
necessary power for the mechanism to have momentum enough to shoot. The final
design considered uses a piston that pushes the projectile trough a mouth piece
that clamps around it to apply force to the projectile. This design is the
design we went with due to the fact that it has a sufficient amount of power and
it is easily reloadable.\nl

The shooting mechanism is mounted on a mechanism that is used to angle the
shooting mechanism to make it able to shoot a target at varying distances. This
mechanism works by having a high gearing on a motor and connecting the platform
that the shooting mechanism is mounted on to the big gear. This makes the
platform rise when the motor turns.

\subsection{Design of the base}

\subsection{Sensor Placement}

\subsection{Rotating Mechanism}
motor beside the turret with gearing
mortor connected directly to rotator
motor placed on turret with gearing

\section{Target Design}

