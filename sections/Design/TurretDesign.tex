\section{Turret Design}
In this section the different parts of \namep's design wil be described. This
includes the mechanism for shooting, the design of the base, sensor placement
and the mechanism for rotating.

\subsection{Shooting Mechanism}
When it comes to the shooting mechanism it needs to be able to shoot at least
200 cm as this is one of the requirements. This means that it a significant
amount of power behind each shot. To make this a reality three different designs
were considered. The first design considered was a balista due to the fact that
a balista is able to deliver a lot of power. This design was quickly dismissed
due to the fact that it is hard to make it able to reload itself, which would
mean that it would only be able to fire a single shot. This was deemed as a
major flaw since if it missed it would not be able to try again. Another design
considered utilizes the same principle as a football launcher, where two wheels
placed side by side are spinning in opposite directions and a projectile is
pushed through. 
This design was not realised due to the fact that we did not have acces to motors
powerfull enough to deliver the necessary power for the mechanism to have
momentum enough to shoot. The final design considered uses a piston that pushes
the projectile trough a mouth piece that clamps around it to apply force to the
projectile. This design is the design we went with due to the fact that it has
a sufficient amount of power and it is easily reloadable.\nl

The shooting mechanism is mounted on a mechanism that is used to angle the
shooting mechanism to make it able to shoot a target at varying distances. This
mechanism works by having a high gearing on a motor and connecting the platform
that the shooting mechanism is mounted on to the big gear. This makes the
platform rise when the motor turns.

\subsection{Design of the base}
The base of the turret is a frame where the nxt is mounted and the shooting and
angleing mechanism mounted on top. Two designs were considered for the frame,
one where the NXT is mounted upright on the side of the base and one where the
NXT is laying flat on the back inside the base. In the design where the NXT was
upright the screen and the buttons on the front was accessible, but all the I/O
ports on the end was hard to get to. This design was not chosen mostly because
of the height but also because it was hard to charge the NXT. The height is a
factor due to the fact that the taller the turret gets the more unstaple it
becomes. This design also placed a lot of the weight on one side of turret
which med it more unstable.\nl

In the chosen design where the NXT is laying on the back inside the turret the
I/O ports are fully accessible. This design also makes the turret a lot shorter
and distributes the weight which makes the turret more stable. 
\subsection{Sensor Placement}
The sensor are a vital part of the turret and need to placed in such a way that
the results from the sensors are as good as possible. As the tests described in
\autoref{UltraTestPlacement} states, it is important that the sensor is not
placed to close to the ground due to the fact that it interferes with the
soundwaves.\nl

The final design makes use of two ultrasonic sensors placed side by side since
it was determined that the field of view would be to narrow with only one
sensor. This is due to the fact that the ultrasonic sensor needs to be pointed
almost perpendicular at the target in order to catch the soundwaves that bounces
back. The use of two sensor brings up the problem of interference between the
two sensors but that was solved by alternating between turning on and off aech
sensor.\nl

\name also makes use of a camera to track a target. This camera is mounted on
the front of the base in the middle. This is beause if it was not placed in the
middle the turret's aim would be off.

\subsection{Turning Mechanism}
The turret needs to be able to turn in order to be able to track a moving
target. For this purpose three designs were considered. In the first design a
motor was placed beside the turret. This motor was connected to a stick with a
spiral gear that was conected to a gear on the bottom of the base. This design
made the turning of the turret precise in the sense that the turret was not
prone to over turning when stopping the motor. The problem with this design is
that it limits how much The turret can turn, due to the fact that the cable
connecting the motor to the NXT has to either go on the outside of the turret,
which makes it wrap around and in turn stop the turret from turning, or go on
the inside which makes the cable twist, which also makes the turret stop
turning, although it can turn more than with the cable on the outside.\nl

Another design considered was where the motor was placed inside the base of the
turret and connected directly to the gear that makes the turret turn. This
design does not have the problem of the cable limiting how much the turret can
turn, since the motor is located on the the turret and therefore turns with the
NXT. This design does have a flaw, that is that it overturn whenever it needs to
stop. This means that the turret is not very precise which makes it difficult to
hit a target.\nl

The chosen design is similar to the design with the turret directly connected to
the gear that turns the turret, the difference being that the motor is placed on
the outside of the base and the motor is geared which makes it more precise when
stopping since it does not over turn as much as the design whitout a gear. 

\subsection{Summary}
the final design of the turret consists of a number of different parts. first
and foremost a shooting mechanism with a piston that pushes the projectile
through a mouthpiece that clamps around it. It also has an angeling mechanism
that the shooting mechanism is mounted on. These two parts sit on top of a
base where the NXT is mounted inside to distribute the weight and to make it
shorter. On the front of this bas two ultrasonic sensor placed in order to make
\name able to track and measure distance to a target. All of this sits on a
turning mechanism. A motor is placed on the outside of the
turret with a gear connected to the turning mechanism to make the whole turret
turn. This design is seen in \autoref{}
