\section{Turret Design}
The turret can be considered the product of 3 different design choices, the
shooting mechanism, the base and the sensor placement.


% In this section the different parts of \namep's design wil be described. This
% includes the mechanism for shooting, the design of the base, sensor placement
% and the mechanism for rotating.

\subsection{Shooting Mechanism}
When it comes to the shooting mechanism it needs to be able to shoot at least
200 cm, as this is one of the requirements. This means that it needs a
significant amount of power behind each shot. To make this a reality three
different designs were considered.\nl

The first design which was considered was a ballista-like design due to the fact
that a balista is able to deliver a lot of power to each shot.
\figx[0.5]{ballista}{A sketch of the ballista design.}
This design was quickly dismissed, as while it can deliver a lot power to each
shot, its rate of fire would be slow, as it would be slow to reload. This would
mean that in case of a miss it would most likely not be able to try again.\nl

The second design which was considered, utilizes the same principle as a
tennis ball launcher, two wheels placed side by side spinning in
opposite directions, with the projectile pushed through.\fix{add sketch for a
tennis ball launcher design}{sketch}
This design was not realised due to the fact that we did not have access to
motors powerful enough to deliver the necessary momentum for the shot.\nl

The third and final design uses a piston that pushes the projectile
trough a mouth piece that clamps around it. This means that until enough force
has been applied to the projectile it is held in the mouth piece. When enough
force has been applied by the piston, the projectile will be forced out at high
speed.\fix{Add sketch for piston launcher}{sketch}
This design has several advantages, Lego Technic has a ball launcher which can
store several small ball projectiles. This allows us to have an easily
reloadable and fast firing cannon, which can hold up to 6 shots.\nl

Using this design a test to find how accurate it is at fireing the projectile
to the expected distance was conducted.
In this test the turret was set to an angle before fireing. The distance was
then measured by having the projectile landing in a layer of flour on the
ground. The test was conducted twice The average distance for each angle in
both datasets was calculated and plotted in a graph against the the angle,
which is seen in \autoref{Accuracy}. This shows that

\graphLNo{Angle}{Distance [cm]}{DistVSAngleTestset1,
DistVSAngleTestset2}{Dataset 1, Dataset 2}{Results from accuracy test}{Accuracy}

The cannon is mounted on a mechanism that is used to angle it such that it
is able to shoot a target at varying distances. This mechanism works by having a
high gearing on a motor and connecting the platform that the shooting mechanism
is mounted on to the big gear. This makes the platform rise when the motor
turns.

\subsection{Design of the base}
The base of the turret is designed as a frame where the NXT is mounted and
the shooting and angling mechanism are mounted on the top. Two designs were
considered for the frame, one where the NXT is mounted upright on the side of
the base and one where the NXT is laying flat on the back inside the base. In
the design where the NXT was upright the screen and the buttons on the front
was accessible, but all the I/O ports on the end was hard to get to. This
design was not chosen. This is because of the additional height and the fact
that it is difficult to charge the NXT. The height is a factor due to the fact
that the taller the turret gets the more unstable it becomes. This is because
the base sits in on top of the turning mechanism, which is a narrow point. This
causes the turret to tilt slightly when it turns. This is a problem due to the
fact that if the turret were to make a fast turn it could be overthrown or be
slightly erratic. This design also placed a lot of the weight on one side of
turret which made it more unstable.
In second  design where the NXT is laying on the back inside the
turret, the I/O ports are fully accessible. This design also makes the turret a
lot shorter and distributes the weight which makes the turret more stable. All
of these advantages leads us to choosing this design.

\subsection{Sensor Placement}
The sensor are a vital part of the turret and need to placed in such a way that
the results from the sensors are as good as possible. As the tests described in
\autoref{UltraTestPlacement} states, it is important that the sensor is not
placed to close to the ground due to the fact that it interferes with the
soundwaves. This makes the results imprecise and can potentially make the
turret unable to hit a target.\nl

The final design makes use of two ultrasonic sensors placed side by side since
it was determined that the field of view would be too narrow with only one
sensor. This is due to the fact that the ultrasonic sensor needs to be pointed
almost perpendicular at the target in order to catch the soundwaves that bounces
back. The use of two sensors brings up the problem of interference between the
two sensors but that was solved by alternating between turning on and off each
sensor so that when when one sensor is on the other one is off and vice
versa.\nl

\name also makes use of a camera to track a target. This camera is mounted on
the front of the base in the middle. This is because if it was not placed in the
middle the turret would not aim for the middle of the target. This camera is
calibrated to track the colour red, as this is the colour that the camera
regognizes the most based on the tests in \autoref{CamTest}.

\subsection{Turning Mechanism}
The turret needs to be able to turn in order to be able to track a moving
target. For this purpose three designs were considered. In the first design a
motor was placed beside the turret. This motor was connected to a stick with a
spiral gear that was conected to a gear on the bottom of the base. This design
made the turning of the turret precise in the sense that the turret was not
prone to over turning when stopping the motor. The problem with this design is
that it limits how much the turret can turn, due to the fact that the cable
connecting the motor to the NXT has to either go on the outside of the turret,
which makes it wrap around and in turn stop the turret from turning, or go on
the inside which makes the cable twist, which also makes the turret stop
turning, although it can turn more than with the cable on the outside.\nl

Another design considered was where the motor was placed inside the base of the
turret and connected directly to the gear that makes the turret turn. This
design does not have the problem of the cable limiting how much the turret can
turn, since the motor is located on the the turret and therefore turns with the
NXT. This design does have a flaw, that is that it overturns whenever it needs
to stop. This means that the turret is not very precise which makes it difficult to
aim correctly.\nl

The chosen design is similar to the design with the turret directly connected to
the gear that turns the turret, the difference being that the motor is placed on
the outside of the base and the motor is geared which makes it more precise when
stopping since it does not over turn as much as the design whitout a gear. 

\subsection{Summary}
the final design of the turret consists of a number of different parts. first
and foremost a shooting mechanism with a piston that pushes the projectile
through a mouthpiece that clamps around it. It also has an angeling mechanism
that the shooting mechanism is mounted on. These two parts sit on top of a
base where the NXT is mounted inside to distribute the weight and to make it
shorter. On the front of this bas two ultrasonic sensor placed in order to make
\name able to track and measure distance to a target. All of this sits on a
turning mechanism. A motor is placed on the outside of the
turret with a gear connected to the turning mechanism to make the whole turret
turn. \fix{Add picture of the turret with arrows pointing to the various
parts}{}
