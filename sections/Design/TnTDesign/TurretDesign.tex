\section{Turret Design}
The turret can be considered the product of 3 different design choices:
\textbf{the cannon}, \textbf{the frame} and \textbf{the sensor placement}. These
3 parts of the turret not only have to work on their own, but also work in conjunction with
each other.\nl

The final design of the turret uses a cannon with a piston that pushes the
projectile through a mouthpiece that clamps around it. The piston is mounted on
an angling mechanism, which is positioned on top of the base. The NXT is mounted
inside the frame in order to distribute the weight. Two ultrasonic sensors and a
camera are positioned on the front of the frame, allowing \name to track and
measure the distance to a target. All of this sits on top of the turning
mechanism which is a gear connected to a motor placed on the outside, allowing
the base to turn. The turret can be seen in \autoref{turret}.

\figx[0.16]{turret}{The final turret and its parts.}

In the following subsections the design process and arguments for the
design of each part will be presented.

\subsection{Design of the Cannon}
The goal of the cannon is to be able to shoot at least 200cm, as this is one of
the requirements see \autoref{FeatAndReq}. Thus it needs a lot of power behind
each shot. It also needs to be able to fire multiple times without any human
interaction, so the turret can fire again in case it misses. It will also allow
the potential extension of the turret by enabling other firing modes such as
burst-fire or the ability to fire at multiple enemies, although it is beyond the
scope of the project. Three different designs were considered: a
\textbf{ballista design}, a \textbf{launcher design} and a \textbf{piston
design}.

% \begin{itemize}
%   \item A ballista design
%   \item A launcher design
%   \item A piston design
% \end{itemize}

\subsubsection{The Ballista}

The ballista, see \autoref{ballista}, was considered as depending on how far
back the projectile can be pulled, a shot can be fired with a lot of power. The
obvious drawback is that it cannot fire more than 1 shot before it needs to be
reloaded. Another drawback is that this design requires a separate mechanism for
pulling back the projectile after reloading. This results in a low rate of fire,
which means that in case the turret misses, it will not be able to shoot again.
This design was dismissed in the conceptual phase because of these drawbacks.
\figx[0.4]{ballista}{A sketch of the ballista design.}

\subsubsection{The Launcher}

The launcher, see \autoref{tennisball}, was considered since it is able to
launch projectiles rapidly, and depending on the quality of the wheels, with a
lot of force. It utilizes two wheels spinning in opposite directions, placed
side by side with a small gap in between, for the projectile to be pushed
through. When trying to build this design, it was determined that the available
motors were not powerful enough to deliver the necessary momentum for launching
a projectile, not even if coupled with a gearing. Another problem is that the
wheels available are made of soft rubber, which absorbs a lot of the force when the
projectile is pushed through. For this design to work better, the wheels would
need to be made of a hard material that does not absorb the force and has a lot
of friction, which makes it easier for the wheels to grip the projectile. This
design was dismissed because it was not possible to fire with enough force to
achieve the desired range.
\figx[0.4]{tennisball}{A sketch of the spinning wheel design.}
\subsubsection{The Piston}

The piston, see \autoref{piston}, uses a piston that pushes the
projectile through a mouthpiece that clamps around it. This means that until
enough force has been applied to the projectile it is held in the mouthpiece.
When enough force has been applied by the piston, the projectile will be forced
out at high speed. This design has several advantages, first and foremost LEGO
Technic has a ball shooter available, which can store several small ball
projectiles. This allows us to have an easily reloadable and fast firing cannon,
which can hold up to 6 shots. Based on tests performed in
\autoref{AppendixDistTest}, it can be seen that the cannon is able to shoot the
required minimum of 200cm.
\figx[0.4]{piston}{A sketch of the piston design.}


The piston-design was chosen for the final cannon design, as it is able to
shoot up to the required range of 200cm, and is able to fire up to 6
consecutive shots. In addition, the tests conducted in \autoref{SensorTest} also
showed that the sensors on the turret are incapable of sensing the target any
further than $\sim$ 200cm. Due to this fact, the increased range of the
other designs were considerably less meaningful when choosing a
design.\nl

The precision of the piston-design was tested in \autoref{AppendixAccTest}, and
the conclusion are shown below:

\begin{itemize}
  \item The turret has an average chance to misfire of 15\%.
  \item If we disregard the misfires, the turret has a average width-spread of
  3.357cm at 4 degrees, and up to 6.607cm at 40 degrees.
  \item If we disregard the misfires, the turret has an average length-spread of
  4cm at 4 degrees, and up to 20cm at 40 degrees.
\end{itemize}

In these conclusions, the spread indicates the average deviation from the
intended location.\nl

This test shows that the turret is fairly accurate, but it does
have a considerable chance of misfiring or shooting slightly too far. While the
problem with misfires are not easily solvable, the problem with shooting too far
and too wide can possibly be solved by increasing the size of the target.

% The final design of the turret has the cannon mounted on a motor that is used to
% control its vertical angle. This way it is able to angle itself in order to
% shoot a target at varying distances. The accuracy of this mechanism is ensured
% by having a high gearing on the motor and connecting the platform to the cannon.


% the result can be seen in
% \label{graph:Accuracy}.

% \begin{graphEnv}
% \begin{graphFloat}
% \centering
% \begin{tikzpicture}
% 	\begin{axis}[xlabel={Angle}, ylabel={Distance}, xmin=0,
% 	ymin=0,xmax=60,ymax=350, minor tick num = 3, /pgfplots/grid = both, axis lines
% 	= left,legend style={ at={(1.3,0.5)}, anchor=center}]
% 	\addplot[blue] expression[domain=0:50,samples=300]{-0.0035*x^3 + 0.2498*x^2 -
% 	1.957*x + 114.27};
% 	\addlegendentry{Average}
% 	\addplot[red,only marks]table {data/DistVSAngleTestset1.dat}
% 	[xshift=4pt,yshift=8pt]
% 	node[pos=0]{%
% 		\pgfplotspointplotattime
% 		\footnotesize
% 		$(\pgfmathprintnumber
% 		{\pgfkeysvalueof{/data point/x}},
% 		\pgfmathprintnumber
% 		{\pgfkeysvalueof{/data point/y}})$
% 	}
% 	node[pos=0.07]{%
% 		\pgfplotspointplotattime
% 		\footnotesize
% 		$(\pgfmathprintnumber
% 		{\pgfkeysvalueof{/data point/x}},
% 		\pgfmathprintnumber
% 		{\pgfkeysvalueof{/data point/y}})$
% 	}
% 	node[pos=0.14]{%
% 		\pgfplotspointplotattime
% 		\footnotesize
% 		$(\pgfmathprintnumber
% 		{\pgfkeysvalueof{/data point/x}},
% 		\pgfmathprintnumber
% 		{\pgfkeysvalueof{/data point/y}})$
% 	}
% 	node[pos=0.21]{%
% 		\pgfplotspointplotattime
% 		\footnotesize
% 		$(\pgfmathprintnumber
% 		{\pgfkeysvalueof{/data point/x}},
% 		\pgfmathprintnumber
% 		{\pgfkeysvalueof{/data point/y}})$
% 	}
% 	node[pos=0.28]{%
% 		\pgfplotspointplotattime
% 		\footnotesize
% 		$(\pgfmathprintnumber
% 		{\pgfkeysvalueof{/data point/x}},
% 		\pgfmathprintnumber
% 		{\pgfkeysvalueof{/data point/y}})$
% 	}
% 	node[pos=0.35]{%
% 		\pgfplotspointplotattime
% 		\footnotesize
% 		$(\pgfmathprintnumber
% 		{\pgfkeysvalueof{/data point/x}},
% 		\pgfmathprintnumber
% 		{\pgfkeysvalueof{/data point/y}})$
% 	}
% 	node[pos=0.42]{%
% 		\pgfplotspointplotattime
% 		\footnotesize
% 		$(\pgfmathprintnumber
% 		{\pgfkeysvalueof{/data point/x}},
% 		\pgfmathprintnumber
% 		{\pgfkeysvalueof{/data point/y}})$
% 	}
% 	node[pos=0.49]{%
% 		\pgfplotspointplotattime
% 		\footnotesize
% 		$(\pgfmathprintnumber
% 		{\pgfkeysvalueof{/data point/x}},
% 		\pgfmathprintnumber
% 		{\pgfkeysvalueof{/data point/y}})$
% 	}
% 	node[pos=0.56]{%
% 		\pgfplotspointplotattime
% 		\footnotesize
% 		$(\pgfmathprintnumber
% 		{\pgfkeysvalueof{/data point/x}},
% 		\pgfmathprintnumber
% 		{\pgfkeysvalueof{/data point/y}})$
% 	}
% 	node[pos=0.63]{%
% 		\pgfplotspointplotattime
% 		\footnotesize
% 		$(\pgfmathprintnumber
% 		{\pgfkeysvalueof{/data point/x}},
% 		\pgfmathprintnumber
% 		{\pgfkeysvalueof{/data point/y}})$
% 	}
% 	node[pos=0.7]{%
% 		\pgfplotspointplotattime
% 		\footnotesize
% 		$(\pgfmathprintnumber
% 		{\pgfkeysvalueof{/data point/x}},
% 		\pgfmathprintnumber
% 		{\pgfkeysvalueof{/data point/y}})$
% 	}
% 	node[pos=0.77]{%
% 		\pgfplotspointplotattime
% 		\footnotesize
% 		$(\pgfmathprintnumber
% 		{\pgfkeysvalueof{/data point/x}},
% 		\pgfmathprintnumber
% 		{\pgfkeysvalueof{/data point/y}})$
% 	}
% 	node[pos=0.84]{%
% 		\pgfplotspointplotattime
% 		\footnotesize
% 		$(\pgfmathprintnumber
% 		{\pgfkeysvalueof{/data point/x}},
% 		\pgfmathprintnumber
% 		{\pgfkeysvalueof{/data point/y}})$
% 	}
% 	node[pos=0.91]{%
% 		\pgfplotspointplotattime
% 		\footnotesize
% 		$(\pgfmathprintnumber
% 		{\pgfkeysvalueof{/data point/x}},
% 		\pgfmathprintnumber
% 		{\pgfkeysvalueof{/data point/y}})$
% 	}
% 	node[pos=0.98]{%
% 		\pgfplotspointplotattime
% 		\footnotesize
% 		$(\pgfmathprintnumber
% 		{\pgfkeysvalueof{/data point/x}},
% 		\pgfmathprintnumber
% 		{\pgfkeysvalueof{/data point/y}})$
% 	}
% 	;
% 	\node at(400,340){\scriptsize$y=-0.0035*x^3 + 0.2498*x^2$};
% 	\node at(400,320){\scriptsize$-1.957*x + 114.27$};
% 	\node at(400,300){\scriptsize$R^2=0.682$};
% 	\addlegendentry{Data}
% 	\end{axis}
% \end{tikzpicture}
% \caption{Results from accuracy test}
% \end{graphFloat}
% \label{graph:Accuracy}
% \end{graphEnv}
% 
% \fix{This shows that The precision of these shoots can be seen in
% This shows that the turret is fairly accurate, since the points on
% the graph somewhat follows .. . The tests conducted in \autoref{SensorTest}
% also shows that the sensors that \name makes use can not sense any longer than
% the piston design can shoot, which makes the power that the two other designs
% has obslete.\nl}{need an update flour test}



\subsection{Design of the Frame}
The frame of the turret is designed with two requirements in mind: it should
be able to turn freely and it should be as stable as possible. Two designs were
considered for the frame:

\begin{itemize}
  \item A design with the NXT mounted upright
  \item A design with the NXT mounted laying down
\end{itemize}

In order to not obstruct turning, the frame is designed to acomodate the NXT.
This allows the wires to reach the modules, as they are otherwise not long
enough to connect the NXT and modules.\nl

\subsubsection{Upright Design}
In the upright design the screen and the buttons on the front
are accessible, but all the input ports are hard to reach. This design was not
chosen, in part because the weight distribution was unbalanced due to its height
and also the fact that it is difficult to charge the NXT. The height is a factor
since the taller the turret gets the more unstable it becomes. This is because
the base sits on top of the rotating gear, which is a narrow point, which causes
the turret to tilt slightly when it turns. This is a problem, since if the
turret were to make a fast turn it could potentially tip over. This design also
places a lot of the weight on one side of turret which causes it to be
unstable.

\subsubsection{Flat Design}
In the design where the NXT is lying down, the I/O ports are fully accessible.
This design also makes the turret a lot shorter, than the upright design, and
distributes the weight more evenly, which makes the turret more stable. The
downside of this design is that the access to the screen and buttons is
obstructed. In spite of this, the advantages leads us to choosing this design.

\subsection{Sensor Placement}
The sensors are a vital part of the turret and need to placed in such a way that
the results from the sensors are as good as possible. As the tests described in
\autoref{UltraTestPlacement}, it is important that the ultrasonic sensors
are not placed too close to the ground due to the fact that it interferes with
the sound waves. This makes the results imprecise and can potentially cause
the turret to miss the target.\nl

The final design makes use of two ultrasonic sensors placed side by side since
it was determined that the field of view would be too narrow with only one
sensor. This is due to the fact that the ultrasonic sensor needs to be pointed
almost perpendicular to the target in order to catch the sound waves that
bounces back. The use of two sensors brings up the problem of interference between the
two sensors, but this was solved by alternating between turning on and off each
sensor so that when one sensor is on the other one is off and vice
versa.\nl

\name also makes use of a camera to track a target. The camera is mounted in
the middle of the front. This is to give the camera the best view of the target.
The camera is calibrated to track the colour red, as this is one of the
easiest colours to recognize, and unlike orange it is not as likely to produce
false positives, see \autoref{CamTest}.

\subsection{Turning Mechanism}
The turret needs to be able to turn in order to be able to track a moving
target. For this purpose three designs were considered: a \textbf{spiral gear
design}, a \textbf{direct connection design} and a \textbf{gear design}.

\subsubsection{Spiral Design}
In the spiral gear design a motor was placed beside the turret. This motor was
connected to a stick with a spiral gear that was connected to a gear on the
bottom of the base. This design made the turning of the turret precise in the
sense that the turret was not prone to over turning when stopping. The problem
with is design is that the connecting cable has to go from the turret to the
external motor, which is problematic if the turret turns too much, as it would
either cause the cable to wrap around the turret, or twist and possibly damage
itself.

\subsubsection{Direct Connection Design}
The direct connection design was considered since the motor was placed inside
the base of the turret and connected directly to the gear that makes the turret turn. This
design does not have the problem of the cable limiting how much the turret can
turn, since the motor is located on the turret and therefore turns with the
NXT. This design does have the flaw, that it can overturn whenever it needs
to stop. This is because the stick that is connected to the turning mechanism
is made of soft plastic, which makes it twist when the motor applies force to
it. When the tension in the plastic is large enough the turret begins to turn.
This means that when the motor stops the plastic will untwist which makes the
turret overturn. This means that the turret is very imprecise which makes it
difficult to aim correctly. This could be compensated for in the software by
calculating how much more the turret turns due to the plastic stick twisting,
but since this is difficult to measure and it is relative to how much force the motor applies it was deemed
not worth the effort, when it could also be solved by having a gear between
the motor and the turning mechanism.

\subsubsection{Gear Design}
The gear design is similar to the design with the turret directly connected to
the gear that turns the turret, the difference being that the motor is placed on
the outside of the base and the motor is geared which makes it more precise when
stopping since it does not overturn as much as the design without a gear.

\subsection{Conclusion: Turret Design}
After designing the turret, a number of tests were conducted in order to
determine the quality of the turrets individual parts. These tests are
presented in \autoref{AppendixDistTest} and \autoref{angleTest}, and have
reached the following conclusions:\nl

\begin{itemize}
  \item The test in \autoref{angleTest} determined that the turret is capable of
  accurately setting its angle of fire.
  \item The test in \autoref{AppendixDistTest} shows that the firing mechanism
  is sufficient, as the turret is capable of accurately firing projectiles.
  \item In general, observations throughout these tests confirm that the
  turret's mechanical design is satisfactory.
\end{itemize}
