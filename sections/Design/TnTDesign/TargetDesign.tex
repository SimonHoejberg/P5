\section{Target Design}
In this section the design decisions of the target will be explained. First the
requirements of the target will be presented again. Then the different design
aspects of the target will be discussed along with the reasoning for choosing them.
\subsection{introduction}

Using these requirements from \autoref{FeatAndReq} two aspects of the target can be identified
\begin{itemize}
  \item The turret needs to be able to see the target
  \item The target should be able to move
\end{itemize}

\subsection{making the target trackable}
Since the primary method of tracking is tracking colours using the camera, the
tests in \autoref{CamTest} were conducted in order to find the most suitable
colour for the target. From these tests we came to the conclusion that the target
needs to be orange in order to get the best results from the camera. However it
was also determined that oranges gives to many false positives and therefore red
was chosen as the target colour since it performed second best in the camera test.
\nl
Both a flat and cylindrical design for the target have been considered. The
advantage of the flat design is that it is large and therefore easy for the ultrasonic
sensors to sense when they are pointed directly at the target. The disadvantage
is that if it is at an angle to the ultrasonic sensors line of sight it
becomes less likely that the soundwaves gets catched by the sensor when they
bounce back. Another disadvantage is that if the target is at an angle with the
cameras line of sight it appears smaller to the camera.\nl

The cylindrical design should in theory counteract the problem with the
ultrasonic sensors needing to be pointed directly at the target, due to the
fact that the angle is the same all around the target. The cylindrical design
also ensures that the camera always sees the target as the same size.\nl

\subsection{making the target move}
Since our problem statement says that the target is always travelling the same direction
at a constant speed, there is no need for the target to be able to turn. This allows
for a simple design for the vehicle, using only 3 wheels, fixed in place. The cart
is powered by a NXT brick used to control the motors. The cart has 4 different speeds including standing:
25, 50,75,100 indicating the percentage of power being fed to the motors. In order to
calculate the speed a test was conducted\autoref{speedotesto}
although the speed with a value of 25 was never tested since it was observed that
it was travelling too slow for our requirement from \autoref{FeatAndReq} that states
that the target must travel at a speed such that the turret will miss if it fires when
it sees a target.
