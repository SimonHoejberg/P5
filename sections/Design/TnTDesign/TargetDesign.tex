\section{Target Design}\label{targetDesign}
In this section the design decisions of the target will be explained. Then the
different design aspects of the target will be discussed along with the
reasoning for choosing them.

\subsection{Tracking the Target}
Since the primary method of tracking is recognizing colours using the camera,
the tests in \autoref{CamTest} were conducted in order to find the most suitable
colour for the target. From these tests we reached the conclusion that the
target needs to be orange in order to get the best results from the camera. However it
was also determined that orange produces to many false positives and therefore,
red was chosen as the colour for the target since it performed second best in
the camera test.
\nl

Both a flat and cylindrical design for the target have been considered. The
advantage of the flat design is that it is large and therefore easy for the ultrasonic
sensors to sense when they are pointed directly at the target. The disadvantage
is that if it is at an angle to the ultrasonic sensors line of sight it
becomes less likely that the sound waves are caught by the sensor when they
bounce back. The disadvantage of this design is that if the target not
perpendicular with the camera's line of sight, it appears smaller to
the camera.\nl

The cylindrical design should in theory counteract the problem with the
ultrasonic sensors needing to be pointed directly at the target, due to the
fact that the angle is the same all around the target. The cylindrical design
also ensures that the camera always sees the target as the same size.\nl

\subsection{Moving the Target}
Since our problem statement states that the target is always travelling in the
same direction and at a constant speed, there is no need for the target to be
able to turn. This allows for a simple design of the vehicle, using only 3
wheels, fixed in place. The cart is powered by an NXT brick used to control the
motors. The target can move at a speed between 0\% and 100\% in intervals of
10.
\nl

In order to determine whether or not the target adheres to the requirements, a
test was conducted, which can be found in \autoref{speedotesto}.\\
In the test, a number of speeds were tested, but it should be noted that speeds
at a power-value of less than 25\% were never tested since it was observed that
it was travelling too slow for our requirement from \autoref{FeatAndReq} that
states that the target must travel at a speed such that the turret will miss if
it fires when it sees a target.\nl

The test concluded that the target moves with an almost constant speed, and that
it moves in a straight line.
