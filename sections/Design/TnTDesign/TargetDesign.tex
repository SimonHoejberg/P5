\section{Target Design}
When designing the target a few things need to be considered. As stated in
\autoref{FeatAndReq} The target needs to move fast enough to make \name
calculate an offset. The turret also needs to be able to calculate this offset,
which means that it needs to be able see the target. To determine the
best colour for the target the tests in \autoref{CamTest} was conducted. From
these tests we came to the conclusion that the target needs to be orange in
order to get the best results from the camera.
However it was determined that oranges gives to many false positives and
therefore red was chosen as the target colour since it performed second best in
the camera test.\nl

Both a flat and cylindrical design for the target has been considered. The
advantage of the flat design is that it is large and easy for the ultrasonic
sensors to see when they are pointed directly at the target. The disadvantage
is that if it is at an angle to the ultrasonic sensors line of sight it 
becomes less likely that the soundwaves gets catched by the sensor when they
bounce back. Another disadvantage is that if it is at an angle with the
cameras line of sight it appears smaller to the camera.\nl

The cylindrical design should in theory counteract the problem with the
ultrasonic sensors needing to be pointed directly at the target, due to the
fact that the angle is the same all around the target. The cylindrical design
also ensures that the camera always sees the target as the same size.\nl

In order to make the target move cart was designed. This cart consists of a
platform to place the target on and three wheels that are all fixed so they can
not turn. An NXT is used to control the speed of the cart, to ensure that it
travels at a constant speed.\fix{Add picture of the target}{}



