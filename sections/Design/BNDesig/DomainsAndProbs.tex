\section{Domains and Probability}\label{MID}

The model is designed to predict \texttt{Vector\_2}, to do this we need to
model how the vectors relate to the observations and each other, as well as
define their domains are. There are however limitations we need to consider. The
model is constructed using Hugin Lite \citep{Hugin}, this allows the model to
be constructed easily, but it limits the total number of state to 50 and
the number of cases to 500. An example of a case is a specific value in \texttt{SD\_0}
given a specific value in \texttt{Vector\_1}. This limitation decreases the
precision of the model. If we want to model the entire domain of for example
distance, we would need a state for each possible value the domain
contains. To work around these limits we have divided
the domains into intervals. This means the model lose some precision, but it
allows us to work around the limit of 500 cases. Instead of having only a few
nodes with a continuous set of values, we can consider additional variables,
which can  make up for the loss of accuracy. It also allows us group the values
that seem unlikely to be used often into big intervals.

\subsection{Constructing the Probability Tables}

Most of the domains are easy to determine, for instance the domains for each of
the sensors are determined by what they can output. As such the domain for
distance has values between 0 and 255, as this is what the ultrasound sensor
can sense. The domain for the vectors was determined by gathering data about the
vectors and sensor observations by outputting the observed values, see
\autoref{AppendixMIData}. This data is used to construct the
probability tables by constructing the vectors between each of the
observations. These vectors are then used to construct the domain for vectors by creating an average vector for
the target at moving at 60\%, 80\% and when the observations are deemed flawed.
These vectors are then used in conjuction with observations themselves to create
the probability table.\nl


\fix{}{tables need to be updated to better illustrate the possibility of bad
observations}

The domain for distance is divided into 5 intervals, with the first and the last
intervals being the biggest. This ensures that we can have improved precision
for the remaining 3 intervals. As such the probability table for \texttt{SD\_0}
and \texttt{SD\_2} can be seen in see \autoref{SDtable}.

\begin{center}
\begin{table}[H]
\begin{tabular}{|l|l|l|l|}
\hline
\multicolumn{4}{|l|}{P(SD\_0|Vector\_1)} \\ \hline
Vector\_0     & (0.033; -0.029)-80\% & (0.012; -0.014)-Bad observations &
(0.020,-0.018)-60\% \\\hline 
[0 - 60]      & 3/22 = 0.14    & 0    & 0.11 \\ \hline 
[60 - 75]    & 9/22 = 0.41 & 0.33 & 0.44 \\ \hline
[75 - 90]   & 0.33 & 0.33 & 0.22 \\ \hline
[90 - 105]   & 0    & 0.11 & 0.17 \\ \hline
[105 - 255]   & 0.34 & 0.23 & 0.06   \\ \hline
\end{tabular}
\caption{The probability table for sensing the distance given the vector.}
\label{SDtable}
\end{table}
\end{center}

\texttt{SD\_1} is a special case, since it is dependant on
both \texttt{Vector\_1} and \texttt{Vector\_2}, this can be seen in
\autoref{SDtable2}.

\begin{table}[H]
\centering
\begin{tabular}{|l|l|l|l|l|l|l|l|l|l|}
\hline
\multicolumn{10}{|l|}{P(SD\_1|Vector\_1, Vector\_2)} \\\hline 
Vector\_2 & \multicolumn{3}{l|}{} & \multicolumn{3}{l|}{} &
\multicolumn{3}{l|}{}\\\hline 
 &       &       &       &       &       &       &       &       &       \\
\hline &       &       &       &       &       &       &       &       &       \\ \hline
 &       &       &       &       &       &       &       &       &       \\ \hline
 &       &       &       &       &       &       &       &       &       \\ \hline
 &       &       &       &       &       &       &       &       &       \\ \hline
 &       &       &       &       &       &       &       &       &       \\ \hline
\end{tabular}
\caption{The probability table for sensing the distance given two vectors.}
\label{SDtable2}
\end{table}

% Vector\_0     & (0.028; -0.013) & (-0.005; -0.028) & (0.028,-0.028)  \\\hline   
% [0 - 60]      & 0    & 0    & 2    \\ \hline 
% [60 - 75]    & 3    & 3    & 8    \\ \hline
% [75 - 90]   & 3    & 3    & 4    \\ \hline
% [90 - 105]   & 0    & 1    & 3    \\ \hline
% [105 - 255]   & 3    & 2    & 1    \\ \hline

The domain for angle is likewise determined by how far the turret can turn,
which is 360 degrees. It is of note that the angle is set as 0 at the start of
program, which results in a negative angle if it turns towards the target at the
start, instead of following it. This means that domain for sensing the angle is
adjusted slight to -30 to 330 degrees. Like distance, \autoref{SAtable} is
the probability for \texttt{SA\_0} and \texttt{SA\_2}.
\begin{center}
\begin{table}[H]
\label{SAtable}
\begin{tabular}{|l|l|l|l|}
\hline
\multicolumn{4}{|l|}{P(SA\_0|Vector\_1)} \\ \hline
Vector\_0     & (0.033; -0.029)-80\% & (0.012; -0.014)-Bad observations &
(0.020,-0.018)-60\% \\\hline 
[-30 - 0]   & 5/22 = 0.23 & 0    & 3/20 = 0.15   \\ \hline 
[0 - 30]    & 8/22 = 0.36 & 8/30 = 0.27 & 9/20 = 0.45 \\ \hline
[30 - 60]   & 8/22 = 0.36 & 10/30 = 0.33 & 8/20 = 0.40 \\ \hline
[60 - 90]   & 1/22 = 5 & 5/30 = 0.17 & 0    \\ \hline
[90- 330]   & 0    & 7/30 = 0.23   & 0    \\ \hline
\end{tabular}
\caption{The probability table for sensing the angle given the vector}
\end{table}
\end{center}

The probability table for \texttt{SA\_1} is given as:

\begin{table}[H]
\centering
\begin{tabular}{|l|l|l|l|l|l|l|l|l|l|}
\hline
\multicolumn{10}{|l|}{P(SD\_1|Vector\_1, Vector\_2)} \\\hline 
Vector\_2 & \multicolumn{3}{l|}{} & \multicolumn{3}{l|}{} &
\multicolumn{3}{l|}{}\\\hline 
 &       &       &       &       &       &       &       &       &       \\
\hline &       &       &       &       &       &       &       &       &       \\ \hline
 &       &       &       &       &       &       &       &       &       \\ \hline
 &       &       &       &       &       &       &       &       &       \\ \hline
 &       &       &       &       &       &       &       &       &       \\ \hline
 &       &       &       &       &       &       &       &       &       \\ \hline
\end{tabular}
\caption{The probability table for sensing the angle given two vectors.}
\label{SDtable2}
\end{table}


% The domain for time is considered the difference in time between the first
% observation and subsequent observations, see \autoref{ST_table}. This means
% that at most we will need to consider time for it takes for the target to
% travel from one end of the shooting range to other at the slowest speed.
% \begin{center}
% \begin{table}[H]
% \label{ST_table}
% \begin{tabular}{|l|l|l|l|}
% \hline
% \multicolumn{4}{|l|}{P(ST\_0|Vector\_0)} \\ \hline
% Vector\_0     & (-0.15,-0.10) & (-0.15,-0.10)    & (-0.15,-0.10)  \\\hline   
% [0 - 1500]      &     &     &     \\ \hline 
% [1501 - 3000]    &     &     &     \\ \hline
% [3001 - 4500]   &     &     &     \\ \hline
% \end{tabular}
% \caption{The probability table for sensing time given a vector}
% \end{table}
% \end{center}
% 
% The domains for velocity is set to be three seperate speeds, 30cm/s and 42cm/s,
% which correspond to the target moving at 40\% and 60\% of its maximum speed,
% see \autoref{Velo_table}.
% \begin{center}
% \begin{table}[H]
% \label{Velo_table}
% \begin{tabular}{|l|l|l|l|}
% \hline
% \multicolumn{4}{|l|}{P(Velocity)} \\ \hline
% 30cm/s   & 0.50   \\\hline   
% 42cm/s   & 0.50   \\ \hline
% \end{tabular}
% \caption{The probability table for velocity of the target.}
% \end{table}
% \end{center}

The vectors are used to provide both the direction as well as the speed of the
target, this value is show in cm/millisecond along the x and y axis. The
vectors are found by taking the average of the vectors which are clumped
together. Three different vectors are found, an average for the target going at
80\%, an average 60\% and finally a vector that is faulty.

\fix{This means that the vectors domain consist of 3 values,}{might
need to  be updated} see \autoref{Vector0_table}. 

\begin{center}
\begin{table}[H]
\label{Vector0_table}
\begin{tabular}{|l|l|}
\hline
P(Vector\_1) &\\ \hline
(0.028; -0.013) & 0.25   \\ \hline 
(-0.005; -0.028) & 0.25  \\ \hline
(0.028,-0.028)   & 0.50  \\ \hline
\end{tabular}
\caption{The probability table for a vector\_1}
\end{table}
\end{center}

The probability for \texttt{Vector\_2} is given as the following conditional
probability table: \fix{}{needs to be updated}
\begin{center}
\begin{table}[H]
\label{SAtable}
\begin{tabular}{|l|l|l|l|}
\hline
\multicolumn{4}{|l|}{P(SA\_0|Vector\_1)} \\ \hline
Vector\_0     & (0.028; -0.013) & (-0.005; -0.028) & (0.028,-0.028)  \\\hline   
(0.028; -0.013)   & 0 & 0    & 0    \\ \hline 
(-0.005; -0.028)    & 0 & 0    & 0    \\ \hline
(0.028,-0.028)   & 0    & 0    & 0   \\ \hline
\end{tabular}
\caption{The probability table for sensing the angle given the vector}
\end{table}
\end{center}

% Since a bayesian network has been constructed we can abuse the property of
% inference to reduce the size of our probability table. Instead of a joint
% probability-table, it is sufficient instead to just look at the tables we are
% interested in. Since our evidence will be inserted at our S-node and the
% value we are looking for is contained in the node Vector it is thus enough to
% provide the conditional probability table P(Vector\_0|S\_0,S\_1). This reduces
% the overall statespace of our model

\subsection{Using the Network}

In order to use the model we need to insert evidence, ``e'', unto the sensor
nodes of network. This information is then able to flow to both of the vector
nodes through the connections mentioned in \autoref{LabelBN}.

\figx{BNEvidence}{The network with evidence on its sensor nodes.}

We can look at \texttt{Vector\_2} and find the most likely outcome of it, and
see if it is likely to be a bad vector.


