\section{Domains and Probability}\label{MID}
The purpose of the model is to predict \texttt{Vector\_2} when given information
about \texttt{Vector\_1}. In order to accomplish this, we need to model how the
two vectors relate to each other, and how they are affected by information about the
targets location. The first step to this approach is to define what this
information's domain is, that is, how the data is represented and used.
There are however limitations we need to consider in relation to the
implementation of this model. The model is constructed using Hugin Lite
\citep{Hugin} which allows it to be constructed easily, but it limits the total
number of states to 50 and the number of cases to 500. For our model, an example
of a case, is a specific value in \texttt{SD\_0} given a specific value in
\texttt{Vector\_1}. This means, that in order to represent our vectors and
observations of the target, we need to simplify this information in order to
fit the model. While this decreases the accuracy of the model, it is necessary
for implementing it given the limitations. For example, if we want to model the
entire domain for the distance, we would need a state for each possible value
the domain contains. To work around these limits we have divided the domains
for the observations and vectors into intervals.\nl

Most of the domains are easy to determine, for instance the domains for each of
the sensors are determined by what they can output. As such the domain for
distance has values between 0 and 255cm, as this is what the ultrasonic sensor
can sense. This domain is divided into 5 intervals, with the first and the last
intervals being the biggest. This ensures that we can have improved precision
for the remaining 3 intervals. The domain for angle is determined by how far the
turret can turn, which is 360 degrees. It should be noted that the angle is set
as 0 at the start of the program, which results in a negative angle if it turns
towards the target in the beginning, instead of following it. This means the
domain for sensing the angle is adjusted to be between -30 and 330 degrees. The
domains for the angles and distances are shown below in \autoref{distAngleDom1}. 

\begin{table}[H]
\centering
\begin{tabular}{l|l|l}
State & Angle (deg) & Distance (cm) \\ \hline
1     & -30 - 0     & 0 - 60        \\
2     & 0 - 30      & 60 - 75       \\
3     & 30 - 60     & 75 - 90       \\
4     & 60 - 90     & 90 - 105      \\
5     & 90 - 330    & 105 - 255     
\end{tabular}
\caption{Domains for angles and distances respectively.}
\label{distAngleDom1}
\end{table}

The states for the vectors consist of the perceived speed of the target, that
is, how far along the x and y-axis it has moved per millisecond. These vectors
were determined as an average of vectors which were within a 40\% margin of
error of the expected speed. They are simply denoted by how fast
the target is moving, \textbf{41cm/s}, \textbf{57cm/s} and when it is a 'bad
vector', referred to as \textbf{BV}. A 'bad vector' is a vector that is
different from the length of the expected vector by more than 40\%. Due to
time constraints the margin has been set at 40\% as the difference between the
vectors acquired from the tests with the target running at 41cm/s, the speed of
the target is consistently lower than what we expected. If an additional
vector is added that is not perpendicular to the turret, we would
need to denote them by their actual vector for clarity. The vectors are limited
to 3 states, as 1 additional vector will result in 98 additional cases, due to
dependencies between the other nodes and the vectors.
The domain for the vectors are presented in \autoref{vecDom1}.

\begin{table}[H]
\centering
\begin{tabular}{l|l}
State & Vector \\ \hline
1     & 41cm/s \\
2     & 57cm/s \\
3     & BV
\end{tabular}
\caption{Domain for the vectors.}
\label{vecDom1}
\end{table}

Using this simplified approach means the model lose some precision, but it
allows us to work around the limit of 500 cases. Instead of having only a few
nodes with a continuous set of values, we can consider additional variables,
which can make up for the loss of accuracy. It also allows us to group the
values that seem unlikely to be used often into big intervals.

\subsection{Constructing the Probability Tables}

For each speed we want to implement in the model, we need a dataset which is
based on the sensor readings from the turret at that speed, see
\autoref{AppendixMIData}. This allows us to create a series of vectors which we
can use to fill the conditional probability tables. This is done by taking each
observation, such as distance or angle, and comparing it to the expected vector.
For instance, we make an observation at time 0 in the 57cm/s set a distance of
94, we then look at the corresponding V1 and see that it is classified as an accepted
vector for 57cm/s. We can then note this in the conditional probability table
for \texttt{SD\_0|Vector\_1}. After this is done for both sets of tests, we can
construct the table which can be seen in \autoref{SDtable1}. From the 22
observations in the 57cm/s set, 8 of those distances are in the interval [60-75]
see \autoref{SDtable1}. While it seems there is little difference in the
probability distribution, this can most likely be attributed to sample size and
the size of the intervals.

\begin{table}[H]
\centering
\begin{tabular}{|l|l|l|l|}
\hline
\multicolumn{4}{|l|}{P(SD\_0|Vector\_1)} \\ \hline
Vector\_1     & 41cm/s & 57cm/s & BV \\\hline 
[0 - 60]      & 0       & 3/22 = 0.14 & 3/42 = 0.07\\ \hline 
[60 - 75]    & 7/16 = 0.43 & 8/22 = 0.36 & 22/42 = 0.52\\ \hline
[75 - 90]   & 5/16 = 0.31 & 8/22 = 0.36 & 10/42 = 0.24\\ \hline
[90 - 105]   & 2/16 = 0.13 & 2/22 = 0.9 & 6/42 = 0.15\\ \hline
[105 - 255]   & 2/16 = 0.13& 1/22 = 0.5 & 1/42 = 0.02 \\ \hline
\end{tabular}
\caption{The conditional probability table for sensing the distance given the
vector.}
\label{SDtable1}
\end{table}
  
\texttt{SD\_1} is a special case, since it is dependent on both
\texttt{Vector\_1} and \texttt{Vector\_2}, this can be seen in
\autoref{SDtable21}. However, there is a problem with the conditional
probability table. Due to design choices, there is no possibility of a
vector changing from 41cm/s to 57cm/s and vice versa. This is a problem since we
need each column to sum to 1, which means that the table cannot be
used. A way to work around this would be to add very small chance that such
a change can occur, but we have not done this presently.

\begin{table}[H]
\centering
\begin{tabular}{|l|p{1.2cm}|p{1.2cm}|p{1.2cm}|p{1.2cm}|p{1.2cm}|p{1.2cm}|p{1.2cm}|p{1.2cm}|p{1.2cm}|}
\hline
\multicolumn{10}{|l|}{P(SD\_1|Vector\_1, Vector\_2)} \\\hline
Vector\_2 & \multicolumn{3}{l|}{41cm/s} & \multicolumn{3}{l|}{57cm/s} &
\multicolumn{3}{l|}{BV}\\\hline
Vector\_1 & 41cm/s      &  57cm/s     & BV     & 41cm/s      & 57cm/s     
& BV & 41cm/s & 57cm/s & BV \\\hline
[0-60]& 0 & 0 & 0 & 0 & 0 & 2/6 = 0.33& 0 & 2/15 = 0.13 & 1/12 = 0.08 \\\hline
[60-75]& 1/3 = 0.33 & 0 & 5/12 = 0.42 & 0 & 2/6 = 0.33 & 1/6 = 0.17 & 1/6 = 0.17
& 9/15 = 0.6 & 6/12 = 0.5\\ \hline
[75-90] & 1/3 = 0.33 & 0 & 4/12 = 0.33 & 0 & 2/6 = 0.33 & 1/6 = 0.17 & 3/6 =
0.5 & 3/15 = 0.2 &3/12 = 0.25\\\hline
[90-105] & 0 & 0 & 1/12 = 0.08 & 0 & 1/6 = 0.17 & 2/6 = 0.33 & 1/6 = 0.17
& 1/15 = 0.07 & 2/12 = 0.17\\\hline
[105-255] & 1/3 = 0.34 & 0 & 2/12 = 0.17& 0 & 1/6 = 0.17 & 0 & 1/6 = 0.16 & 0
&0\\\hline
\end{tabular}
\caption{The conditional probability table for sensing the distance given two
vectors.}
\label{SDtable21}
\end{table}

As mentioned in \autoref{ModelHMM}, the network can be extended indefinitely,
as such the tables have been generalized such that \texttt{SD\_0} and
\texttt{SD\_2} are the same. The same is applied to \texttt{SA\_0} and
\texttt{SA\_2}, these tables can be seen in \autoref{AppProbTabs}. The probability
distribution of the vectors is seen below in \autoref{Vector0Table1}, while the
likelihood of a bad vector is very high, it is still less than half of the
vectors.

\begin{table}[H]
\centering
\begin{tabular}{|l|l|}
\hline
P(Vector\_1) &\\ \hline
41cm/s & 8/40 = 0.20   \\ \hline 
57cm/s & 11/40 = 0.28  \\ \hline
BV & 21/40 = 0.52  \\ \hline
\end{tabular}
\caption{The probability table for a Vector\_1}
\label{Vector0Table1}
\end{table}


\texttt{Vector\_2} is given as the following conditional probability table, see
\autoref{vector2tab1}. The table shows the likelihood of the second vector given
the first vector. It can be seen that when the first vector is generated it is
very likely that the second vector will be a bad vector, regardless of how fast
the target is moving.

\begin{table}[H]
\centering
\begin{tabular}{|l|l|l|l|}
\hline
\multicolumn{4}{|l|}{P(Vector\_2|Vector\_1)} \\ \hline
 Vector\_1 & 41cm/s    & 57cm/s    & BV    \\ \hline
 41cm/s    & 1/3 = 0.33 & 0     & 4/10 = 0.4    \\ \hline
 57cm/s    & 0          & 2/7 = 0.29 & 2/10 = 0.2  \\ \hline
 BV        & 2/3 = 0.67 & 5/7 = 0.71 & 4/10 = 0.4  \\ \hline
\end{tabular}
\caption{The conditional probability table of Vector\_2 given Vector\_1}
\label{vector2tab1}
\end{table}


\subsection{Using the Network}
In order to present the model in action, we can create a concrete example of the
model determining the direction/speed of the target.\nl

The turret starts its operation by looking straight ahead while waiting for the
target to appear. The target enters the turret's field of vision, and the turret
tracks the target. During the tracking, the turret observes the target at 3
locations, which in this case is an angle/distance pair: (9,104), (20,76) and
(46,65).\nl

This information is used as evidence for the model in Hugin, which allows
us to determine a probability for \texttt{Vector\_1} and \texttt{Vector\_2}
based on our test-data, which is presented in \autoref{AppendixMIData}. For
\texttt{Vector\_1}, given our 3 locations, this results in the following
probabilities:

\begin{itemize}
  \item 57cm/s - 78.57\% chance.
  \item 41cm/s - 14.86\% chance.
  \item Bad Observation - 6.57\% chance.
\end{itemize}

In this case, the model would conclude that \texttt{Vector\_1} would be most
likely to have the target moving at a speed of 57cm/s. Now, given the same
information, the model would assign \texttt{Vector\_2} the following probabilities:

\begin{itemize}
  \item 57cm/s - 24.84\% chance.
  \item 41cm/s - 55.00\% chance.
  \item Bad Observation - 20.16\% chance.
\end{itemize}

The probability for \texttt{Vector\_2} indicates that the most likely conclusion
is that something went wrong with the data. This does not mean that the target's
behavior is unknown, but that the gathered data does not correlate to previous
experience. In case the two vectors do not reach the same conclusion, the
resulting speed will be based on the following reasoning:

\begin{table}[H]
\centering
\begin{tabular}{|l|l|}
\hline
Case                                                                                 & Conclusion \\ \hline
Both vectors agree                                                                   & Vector\_2  \\ \hline
Both vectors disagree                                                             
& Vector\_2  \\ \hline \begin{tabular}[c]{@{}l@{}}First vector is bad,\\ second vector is good\end{tabular} & Vector\_2  \\ \hline
\begin{tabular}[c]{@{}l@{}}First vector is good,\\ second vector is bad\end{tabular} & Vector\_1  \\ \hline
\end{tabular}
\caption{Speed conclusion based on vector\_1 and vector\_2.}
\end{table}

While this model may not always output the correct result, with enough previous
data to base the probabilities on, some of the cases of disagreement between the
vectors would be eliminated.



%This information is then able to flow to both of the vector
% nodes through the connections mentioned in \autoref{LabelBN}.
% 
% \figx[1.15]{BNEvidence}{The network with evidence on its sensor nodes.}
% 
% We can look at \texttt{Vector\_2} and find the most likely outcome of it, and
% see if it is likely to be a bad vector.


