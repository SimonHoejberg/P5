\section{Domains and Probability}\label{MID}
The purpose of the model is to predict \texttt{Vector\_2} when given information
about \texttt{Vector\_1}. In order to do this, we need to model how the two
vectors relate to each other, and how they are affected by information about the
targets location. The first step to this approach is to define what this
information's domains is, that is, how the data is represented and used.
There are however limitations we need to consider in relation to the
implementation of this model. The model is constructed using Hugin Lite
\citep{Hugin} which allows it to be constructed easily, but it limits the total
number of states to 50 and the number of cases to 500. This means, that in order
to represent our vectors and observations of the target, we need to simplify
this information in order to fit the model. While this decreases the accuracy of
the model, it is necessary for implementing it given the limitations. For our
model, an example of a case, is a specific value in \texttt{SD\_0} given a
specific value in \texttt{Vector\_1}. For example, if we want to model the
entire domain for the distance, we would need a state for each possible value
the domain contains. To work around these limits we have divided the domains
for the observations and vectors into intervals. These domains for the angles
and distances are shown below in \autoref{distAngleDom}, and the domain for
\texttt{Vector\_1} and \texttt{Vector\_2} are presented in \autoref{vecDom}.
Both the angles and vectors are limited to 5 different states. The angles have
states representing the values between -30 degrees and 330 degrees, and the
distances represent values between 0cm and 255cm.

\begin{table}[H]
\centering
\label{distAngleDom}
\begin{tabular}{l|l|l}
State & Angle (deg) & Distance (cm) \\ \hline
1     & -30 - 0     & 0 - 60        \\
2     & 0 - 30      & 60 - 75       \\
3     & 30 - 60     & 75 - 90       \\
4     & 60 - 90     & 90 - 105      \\
5     & 90 - 330    & 105 - 255     
\end{tabular}
\caption{Domains for angles and distances respectively.}
\end{table}

The states for the vectors consist of the percieved speed of the target, that
is, at what power threshold the target is moving at. This information is
represented as either 60\%, 80\% or 'bad observation'.

\begin{table}[H]
\centering
\label{vecDom}
\begin{tabular}{l|l}
State & Vector (\%power) \\ \hline
1     & 60\%             \\
2     & 80\%             \\
3     & Bad Observation
\end{tabular}
\caption{Domain for \texttt{Vector\_1} and \texttt{Vector\_2}.}
\end{table}

Using this simplified approach  means the model lose some precision, but it allows us to
work around the limit of 500 cases. Instead of having only a few nodes with a
continuous set of values, we can consider additional variables, which can
make up for the loss of accuracy. It also allows us group the values
that seem unlikely to be used often into big intervals.


\subsection{Constructing the Probability Tables}

Most of the domains are easy to determine, for instance the domains for each of
the sensors are determined by what they can output. As such the domain for
distance has values between 0 and 255cm, as this is what the ultrasound sensor
can sense. The domain for the vectors is determined by gathering data about the
vectors and sensor observations by outputting the observed values, see
\autoref{AppendixMIData}. This data is used to construct the
probability tables, first by constructing the vectors between each
observation and the one after, and then normalizing by dividing with the
difference in time between the observations. These vectors are then used to
construct the domain for vectors by creating an average vector for the target
at moving at 60\%, 80\% and when the observations are deemed flawed. These
vectors are then used in conjuction with observations themselves to create the
probability table.\nl

The domain for distance is divided into 5 intervals, with the first and the last
intervals being the biggest. This ensures that we can have improved precision
for the remaining 3 intervals. As such the probability table for \texttt{SD\_0}
and \texttt{SD\_2} can be seen in see \autoref{SDtable}.

\begin{center}
\begin{table}[H]
\begin{tabular}{|l|l|l|l|}
\hline
\multicolumn{4}{|l|}{P(SD\_0|Vector\_1)} \\ \hline
Vector\_0     & (0.033; -0.029)-80\% & (0.012; -0.014)-Bad observations &
(0.020,-0.018)-60\% \\\hline 
[0 - 60]      & 3/22 = 0.14 & 3/27 = 0.11 & 0 \\ \hline 
[60 - 75]    & 9/22 = 0.41 & 12/27 = 0.44 & 14/24 = 0.58 \\ \hline
[75 - 90]   & 5/22 = 0.23 & 8/27 = 0.30 & 6/24 = 0.25 \\ \hline
[90 - 105]   & 4/22 = 0.18 & 1/27 = 0.04 & 3/24 = 0.13 \\ \hline
[105 - 255]   & 1/22 =0.04 & 3/22 = 0.11 & 1/24 = 0.04   \\ \hline
\end{tabular}
\caption{The probability table for sensing the distance given the vector.}
\label{SDtable}
\end{table}
\end{center}

\texttt{SD\_1} is a special case, since it is dependent on
both \texttt{Vector\_1} and \texttt{Vector\_2}, this can be seen in
\autoref{SDtable2}.

\begin{table}[H]
\centering
\begin{tabular}{|l|p{1.2cm}|p{1.2cm}|l|p{1.2cm}|p{1.2cm}|p{1.2cm}|l|p{1.2cm}|p{1.2cm}|}
\hline
\multicolumn{10}{|l|}{P(SD\_1|Vector\_1, Vector\_2)} \\\hline 
Vector\_2 & \multicolumn{3}{l|}{80\%} & \multicolumn{3}{l|}{Bad} &
\multicolumn{3}{l|}{60\%}\\\hline 
Vector\_1 & 80\%      &  Bad     & 60\%      & 80\%      & Bad      & 60\%     
& 80\% & Bad & 60\% \\\hline
[0-60]&1/12 = 0.08 & 1/3 = 0.33 & 0      & 2/12 = 0.17 & 1/6 = 0.17 & 0 & 0  
& 0 & 0 \\ \hline
[60-75]&4/12 = 0.33 & 1/3 = 0.33 & 1      & 7/12 = 0.58 & 3/6 = 0.5 & 0      & 1     
& 7/15 = 0.47 & 5/9 = 0.56\\ \hline
[75-90] & 4/12 = 0.33 & 0 & 0 & 2/12 = 0.17 & 1/6 = 0.17 & 2/3 = 0.67 & 0 
& 5/15 = 0.33 &1/9 = 0.11\\\hline
[90-105] & 2/12 = 0.17 & 1/3 = 0.33 & 0 & 1/12 = 0.08 & 0 & 0 & 0 
& 2/15 = 0.07  & 3/9 = 0.33\\\hline
[105-255] & 1/12 = 0.08 & 0 & 0 & 0 & 1/6 = 0.17 & 1/3 = 0.33 & 0 & 1/15 = 0.07     
&0\\\hline
\end{tabular}
\caption{The probability table for sensing the distance given two vectors.}
\label{SDtable2}
\end{table}


% Vector\_0     & (0.028; -0.013) & (-0.005; -0.028) & (0.028,-0.028)  \\\hline   
% [0 - 60]      & 0    & 0    & 2    \\ \hline 
% [60 - 75]    & 3    & 3    & 8    \\ \hline
% [75 - 90]   & 3    & 3    & 4    \\ \hline
% [90 - 105]   & 0    & 1    & 3    \\ \hline
% [105 - 255]   & 3    & 2    & 1    \\ \hline

The domain for angle is likewise determined by how far the turret can turn,
which is 360 degrees. It is of note that the angle is set as 0 at the start of
program, which results in a negative angle if it turns towards the target at the
start, instead of following it. This means that domain for sensing the angle is
adjusted slight to -30 to 330 degrees. Like distance, \autoref{SAtable} is
the probability for \texttt{SA\_0} and \texttt{SA\_2}.
\begin{center}
\begin{table}[H]
\label{SAtable}
\begin{tabular}{|l|l|l|l|}
\hline
\multicolumn{4}{|l|}{P(SA\_0|Vector\_1)} \\ \hline
Vector\_0     & (0.033; -0.029)-80\% & (0.012; -0.014)-Bad observations &
(0.020,-0.018)-60\% \\\hline 
[-30 - 0]   & 5/22 = 0.23 & 0    & 3/20 = 0.15   \\ \hline 
[0 - 30]    & 8/22 = 0.36 & 8/30 = 0.27 & 9/20 = 0.45 \\ \hline
[30 - 60]   & 8/22 = 0.36 & 10/30 = 0.33 & 8/20 = 0.40 \\ \hline
[60 - 90]   & 1/22 = 0.05 & 5/30 = 0.17 & 0    \\ \hline
[90- 330]   & 0    & 7/30 = 0.23   & 0    \\ \hline
\end{tabular}
\caption{The probability table for sensing the angle given the vector}
\end{table}
\end{center}

The probability table for \texttt{SA\_1} is given as:

\begin{table}[H]
\centering
%{|l|l|l|l|l|l|l|l|l|l|}
\begin{tabular}{|l|p{1.2cm}|p{1.2cm}|p{1.2cm}|p{1.2cm}|p{1.2cm}|p{1.2cm}|p{1.2cm}|p{1.2cm}|p{1.2cm}|}
\hline
\multicolumn{10}{|l|}{P(SA\_1|Vector\_1, Vector\_2)} \\\hline 
Vector\_1 & \multicolumn{3}{l|}{80\%} & \multicolumn{3}{l|}{Bad} &
\multicolumn{3}{l|}{60\%}\\\hline 
Vector\_2 & 80\% & Bad & 60\% & 80\% & Bad & 60\% & 80\% & Bad & 60\% \\
\hline
[-30 - 0] & 5/12 = 0.42  & 0/3 = 0 & 0/1 = 0 & 0/12 = 0 & 0/6 = 0 & 0/3 = 0 &
0/1 = 0 & 0/15 = 0 & 0/9 = 0 \\ \hline
[0 - 30] & 3/12 = 0.25 & 2/3 = 0.67 & 1/1 = 1 & 4/12 = 0.33 & 0/6 = 0 & 0/3 = 0
& 0/1 = 0 & 6/15 = 0.4 & 2/9 = 0.22 \\ \hline
[30 - 60] & 4/12 = 0.33 & 1/3 = 0.33 & 0/1 = 0 & 6/12 = 0.5 & 2/6 = 0.33 & 1/3 =
0.33 & 1/1 = 1 & 7/15 = 0.47 & 4/9 = 0.44 \\ \hline
[60 - 90] & 0/12 = 0 & 0/3 = 0 & 0/1 = 0 & 1/12 = 0.08 & 0/6 = 0 & 2/3 = 0.67 &
0/1 = 0 & 2/15 = 0.13 & 3/9 = 0.33 \\ \hline
[90 - 330] & 0/12 = 0 & 0/3 = 0 & 0/1 = 0 & 1/12 = 0.08 & 4/6 = 0.67 & 0/3 = 0 &
0/1 = 0 & 0/15 = 0 & 0/9 = 0 \\ \hline
\end{tabular}
\caption{The probability table for sensing the angle given two vectors.}
\label{SDtable2}
\end{table}


% The domain for time is considered the difference in time between the first
% observation and subsequent observations, see \autoref{ST_table}. This means
% that at most we will need to consider time for it takes for the target to
% travel from one end of the shooting range to other at the slowest speed.
% \begin{center}
% \begin{table}[H]
% \label{ST_table}
% \begin{tabular}{|l|l|l|l|}
% \hline
% \multicolumn{4}{|l|}{P(ST\_0|Vector\_0)} \\ \hline
% Vector\_0     & (-0.15,-0.10) & (-0.15,-0.10)    & (-0.15,-0.10)  \\\hline   
% [0 - 1500]      &     &     &     \\ \hline 
% [1501 - 3000]    &     &     &     \\ \hline
% [3001 - 4500]   &     &     &     \\ \hline
% \end{tabular}
% \caption{The probability table for sensing time given a vector}
% \end{table}
% \end{center}
% 
% The domains for velocity is set to be three seperate speeds, 30cm/s and 42cm/s,
% which correspond to the target moving at 40\% and 60\% of its maximum speed,
% see \autoref{Velo_table}.
% \begin{center}
% \begin{table}[H]
% \label{Velo_table}
% \begin{tabular}{|l|l|l|l|}
% \hline
% \multicolumn{4}{|l|}{P(Velocity)} \\ \hline
% 30cm/s   & 0.50   \\\hline   
% 42cm/s   & 0.50   \\ \hline
% \end{tabular}
% \caption{The probability table for velocity of the target.}
% \end{table}
% \end{center}

The vectors are used to provide both the direction as well as the speed of the
target, this value is show in cm/millisecond along the x and y axis. The
vectors are found by taking the average of the vectors which are clumped
together. Three different vectors are found, an average for the target going at
80\%, an average 60\% and finally a vector that is faulty.

This means that the vectors domain consist of 3 values, a vector for moving
at 80\%, 60\% and a bad vector, see \autoref{Vector0Table}. 

\begin{center}
\begin{table}[H]
\begin{tabular}{|l|l|}
\hline
P(Vector\_1) &\\ \hline
(0.033; -0.029) & 13/33 = 0.39   \\ \hline 
(0.012; -0.014) & 7/33 = 0.22  \\ \hline
(60\%)   & 13/33 = 0.39  \\ \hline
\end{tabular}
\caption{The probability table for a Vector\_1}
\label{Vector0Table}
\end{table}
\end{center}

The probability for \texttt{Vector\_2} is given as the following conditional
probability table: 
\begin{table}[]
\centering
\begin{tabular}{|l|l|l|l|}
\hline
\multicolumn{4}{|l|}{P(Vector\_2|Vector\_1)} \\ \hline
 Vector\_1 & 80    & bad    & 60    \\ \hline
 80    & 4/8 = 0.5    & 1/8 = 0.13    & 0    \\ \hline
 bad   & 4/8 = 0.5    & 2/8 = 0.25    & 1/4 = 0.25  \\ \hline
 60    & 0            & 5/8 = 0.62    & 3/4 = 0.75  \\ \hline
\end{tabular}
\caption{The conditional probability table of Vector\_2 given Vector\_1}
\label{vector2tab} 
\end{table}


% \begin{center}
% \begin{table}[H]
% \label{SAtable}
% \begin{tabular}{|l|l|l|l|}
% \hline
% \multicolumn{4}{|l|}{P(SA\_0|Vector\_1)} \\ \hline
% Vector\_0     & (0.033; -0.029) 80\% & (0.012; -0.014) Bad observation & 60\%  \\\hline 
% (-30:0)   & 0.23 & 0    & 0    \\ \hline 
% (0; 30)    & 0.36 & 0.27 & 0.15    \\ \hline
% (30,60)   & 0.36    & 0.33 & 0.45   \\ \hline
% (60:90)   & 0.045	& 0.17 & 0.4 	\\ \hline
% (90:330)  & 0	& 	0.23		& 0 		\\ \hline
% \end{tabular}
% \caption{The probability table for sensing the angle given the vector}
% \end{table}
% \end{center}

% Since a bayesian network has been constructed we can abuse the property of
% inference to reduce the size of our probability table. Instead of a joint
% probability-table, it is sufficient instead to just look at the tables we are
% interested in. Since our evidence will be inserted at our S-node and the
% value we are looking for is contained in the node Vector it is thus enough to
% provide the conditional probability table P(Vector\_0|S\_0,S\_1). This reduces
% the overall statespace of our model

\subsection{Using the Network}

In order to use the model we need to insert evidence, ``e'', unto the sensor
nodes of network. This information is then able to flow to both of the vector
nodes through the connections mentioned in \autoref{LabelBN}.

\figx{BNEvidence}{The network with evidence on its sensor nodes.}

We can look at \texttt{Vector\_2} and find the most likely outcome of it, and
see if it is likely to be a bad vector.


