\section{Domains and Probability}

The model is designed to be used to predict \texttt{Vector\_1}, to get an
accurate read on \texttt{Vector\_1} we need to model how the vectors relate to
the observations and each other, as well as what their domains are. There are
however limitations we need to consider. The model is constructed using Hugin
Lite \citep{Hugin}, this allows the model to be constructed easily but limits the total number of nodes to 50 and the number of cases to 500, an
example of a case is a specific \texttt{S\_0} given a specific
\texttt{Vector\_0}. This limitation causes problems for how precise the model
can be. This can be especially problematic when we would like to use a variable
to represent the distance between the target and turret. To work around these
limits we have:

\begin{itemize}
  \item Partitioned the domain into intervals.
  \item Removed the intervals that were deemed very unlikely to ever be used.
\end{itemize}

Partitioning the domain into intervals means the model loses some precision, but
it allows us to work around the limit of 500 cases. This means that instead of
having only a few nodes with a continuous set of variables we can consider
additional variables, which can potentially make up for the loss of
accuracy.
By removing certain intervals, such as distance values below 50 cm, we can 
reduce the number of cases.\nl

Most of the domains are easy to determine. The domains for each of the sensors
are determined by what they can output. For instance, the domain for distance
needs to between 0 and 255, as this is what the ultrasound sensor can sense. The
domain for angle is likewise determined by how far the turret can turn, 180
degrees. The domain for time is the difference in time between the
first observation and subsequent observations. By determining the domain
this way, we can reduce the available domain to a more manageable size. \fix{The
domains for velocity is set to be three seperate speeds, 30cm/s, 36cm/s and
42cm/s}{Is velocity used as a node, and are these the correct speeds?}. The
domain for the vectors is a bit more difficult to determine. \fix{In order to
determine this domain we have conducted a tests where the vectors and the data
used were measured.}{Perform tests to determine vectors and their likely hoods
as well as all the other chances} By determining the vectors this way, along
with gathering the other data, get a rough estimate of the domain and construct
the probability tables for the other nodes.

\subsection{Constructing the Probability Tables}


% -velocity - maaske fjernes
% -vector - This variable contains the actual vector for the target at time S\_0.t
% . This value directly influences our sensor readings as can be seen in the edges
% going from the vector nodes to two corresponding sensor observations(Vector\_0
% to S\_0 and S\_1). This feature contains two variables, x\_speed and y\_speed
% detailing the direction and speed of the target on each axis. In order to
% calculate a speed, it needs two observations over time which is why a vector
% node is linked to two observations
% 
% 
% The S node contains all our sensor input information, and each S node
% corresponds to a observation at a specific point in time. The S node contains
% the 3 variables given to us by the NXT in order to calculate the vectors we
% need. The values contained herein are as follows:
% \begin{itemize}
% \item distance - This variable contains the value from our ultrasonic sensor reading 
% \item angle - This value is given from the NXT, calculated from the angle of the motor powering the horizontal platform. 
% \item time - the NXT also provides a timestamp to the observation. 
% \end{itemize}
% The distance and angle variable can be used to calculate a position relative to our turret. These variables can then be used in combination with the time of the observation to compare two vectors in order to calculate a trajectory and speed.
% The network is constructed as a HMM. 
% 
% 
% Worth noting is that the observations does not occur on fixed intervals, but
% are instead provided once the NXT has found good data. This creates an
% unpredictable behavior regarding when the observations occur, and thus we need
% to contain a time-variable in our observation feature



% -Problems with the model:
% tjek - problemer med hugin og deres max number of states
% problemer med precision da der bliver lavet grove tiln�rmelser
% begr�nset antal af vektorer


%\section{Partitioning the Domains}



\subsection{inference}
Since a bayesian network has been constructed we can abuse the property of
inference to reduce the size of our probability table. Instead of a joint
probability-table, it is sufficient instead to just look at the tables we are
interested in. Since our evidence will be inserted at our S-node and the
value we are looking for is contained in the node Vector it is thus enough to
provide the conditional probability table P(Vector\_0|S\_0,S\_1). This reduces
the overall statespace of our model

\subsection{Using the Network}



The domains
limitations on the domains


-Problems with the model:
problemer med hugin og deres max number of states
problemer med precision da der bliver lavet grove tilnaermelser
begraenset antal af vektorer


% \subsection{features}
% -velocity - m�ske fjernes
% -vector - This variable contains the actual vector for the target at time
% S_0.t . This value directly influences our sensor readings as can be seen in
% the edges going from the vector nodes to two corresponding sensor
% observations(Vector_0 to S_0 and S_1). This feature contains two variables,
% x_speed and y_speed detailing the direction and speed of the target on each
% axis. In order to calculate a speed, it needs two observations over time
% which is why a vector node is linked to two observations