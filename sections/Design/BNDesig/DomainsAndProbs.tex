\section{Domains and Probability}

The model is designed to predict \texttt{Vector\_1}, and to do this we need to
model how the vectors relate to the observations, other vectors and the
velocity, as well as what their domains are. There are however limitations we
need to consider. The model is constructed using Hugin Lite \citep{Hugin}, this
allows the model to be constructed easily but limits the total number of nodes
to 50 and the number of cases to 500. An example of a case is a specific
value in \texttt{SD\_0} given a specific value in \texttt{Vector\_0}. This
limitation decreases how precise the model can be, it can be especially problematic when we want to
have variables that represent the distance between the target and turret as
well as the current angle of the turret. To work around these limits we have
divided the domains into intervals. This means the model lose some precision,
but it allows us to work around the limit of 500 cases. Instead of having
only a few nodes with a continuous set of values, we can consider additional
variables, which can  make up for the loss of accuracy. It also allows us group
the values that seem unlikely to be used often into big intervals.

\subsection{Constructing the Probability Tables}

Most of the domains are easy to determine, for instance the domains for each of
the sensors are determined by what they can output. As such the domain for
distance has values between 0 and 255, as this is what the ultrasound sensor
can sense. It is much harder to determine the domain for the vectors, as such we
have gathered data about the vectors and sensor observations. This data is not
only used to determine the domain for the vectors, it is also used to determine
the probability tables.\nl

The domain for distance is divided into 7 intervals, with the first and the last
intervals being the biggest, while the remaining 5 divide the rest of the
distance evenly between them, see \autoref{SDtable}.

\begin{center}
\begin{table}[H]
\label{SDtable}
\begin{tabular}{|l|l|l|l|}
\hline
\multicolumn{4}{|l|}{P(SD\_0|Vector\_0)} \\ \hline
Vector\_0     & (-0.15,-0.10) & (-0.15,-0.10)    & (-0.15,-0.10)  \\\hline   
[0 - 75]      &     &     &     \\ \hline 
[76 - 100]    &     &     &     \\ \hline
[101 - 125]   &     &     &     \\ \hline
[126 - 150]   &     &     &     \\ \hline
[151 - 175]   &     &     &     \\ \hline
[176 - 200]   &     &     &     \\ \hline
[201 - 255]   &     &     &     \\ \hline
\end{tabular}
\caption{The probability table for sensing the distance given the angle}
\end{table}
\end{center}

The domain for angle is likewise determined by how far the turret can turn,
which is 180 degrees. It is of note that the angle is set as 0 at the start of
program, in practice this means that domain for sensing the angle is from -15 to
165 degrees. \fix{}{explain -15 degrees is due to camera spotting and moving
too quickly}. 

\begin{center}
\begin{table}[H]
\label{SA_table}
\begin{tabular}{|l|l|l|l|}
\hline
\multicolumn{4}{|l|}{P(SA\_0|Vector\_0)} \\ \hline
Vector\_0     & (-0.15,-0.10) & (-0.15,-0.10)    & (-0.15,-0.10)  \\\hline   
[-15 - 15]      &     &     &     \\ \hline 
[16 - 45]    &     &     &     \\ \hline
[46 - 75]   &     &     &     \\ \hline
[76 - 105]   &     &     &     \\ \hline
[106 - 135]   &     &     &     \\ \hline
[136 - 165]   &     &     &     \\ \hline
\end{tabular}
\caption{The probability table for sensing the angle given the vector}
\end{table}
\end{center}

The domain for time is considered the difference in time between the first
observation and subsequent observations, see \autoref{ST_table}. This means
that at most we will need to consider time for it takes for the target to
travel from one end of the shooting range to other at the slowest speed.

\begin{center}
\begin{table}[H]
\label{ST_table}
\begin{tabular}{|l|l|l|l|}
\hline
\multicolumn{4}{|l|}{P(ST\_0|Vector\_0)} \\ \hline
Vector\_0     & (-0.15,-0.10) & (-0.15,-0.10)    & (-0.15,-0.10)  \\\hline   
[0 - 1500]      &     &     &     \\ \hline 
[1501 - 3000]    &     &     &     \\ \hline
[3001 - 4500]   &     &     &     \\ \hline
\end{tabular}
\caption{The probability table for sensing time given a vector}
\end{table}
\end{center}

The domains for velocity is set to be three seperate speeds, 30cm/s and 42cm/s,
which correspond to the target moving at 40\% and 60\% of its maximum speed,
see \autoref{Velo_table}.

\begin{center}
\begin{table}[H]
\label{Velo_table}
\begin{tabular}{|l|l|l|l|}
\hline
\multicolumn{4}{|l|}{P(Velocity)} \\ \hline
30cm/s   & 0.50   \\\hline   
42cm/s   & 0.50   \\ \hline
\end{tabular}
\caption{The probability table for velocity of the target.}
\end{table}
\end{center}

The domain for the vectors are determined by 
\fix{}{determine vectors}

\begin{center}
\begin{table}[H]
\label{Vector0_table}
\begin{tabular}{|l|l|l|l|}
\hline
\multicolumn{4}{|l|}{P(Vector\_0|Velocity)} \\ \hline
Vector\_0     & (-0.15,-0.10) & (-0.15,-0.10)    & (-0.15,-0.10)  \\\hline   
[0 - 1500]      &     &     &     \\ \hline 
[1501 - 3000]    &     &     &     \\ \hline
[3001 - 4500]   &     &     &     \\ \hline
\end{tabular}
\caption{The probability table for a vector given the velocity}
\end{table}
\end{center}

\begin{center}
\begin{table}[H]
\label{Vector1_table}
\begin{tabular}{|l|l|l|l|}
\hline
\multicolumn{4}{|l|}{P(Vector\_1|Vector\_0,Velocity)} \\ \hline
Vector\_0     & (-0.15,-0.10) & (-0.15,-0.10)    & (-0.15,-0.10)  \\\hline   
[0 - 1500]      &     &     &     \\ \hline 
[1501 - 3000]    &     &     &     \\ \hline
[3001 - 4500]   &     &     &     \\ \hline
\end{tabular}
\caption{The probability table for a vector given the velocity and the previous
vector}
\end{table}
\end{center}


% Perform tests to determine vectors and their likely
% hoods as well as all the other chances.
% 
% By determining the vectors this way, along with gathering the other data, get a
% rough estimate of the domain and construct the probability tables for the other
% nodes.



\subsection{inference}





Since a bayesian network has been constructed we can abuse the property of
inference to reduce the size of our probability table. Instead of a joint
probability-table, it is sufficient instead to just look at the tables we are
interested in. Since our evidence will be inserted at our S-node and the
value we are looking for is contained in the node Vector it is thus enough to
provide the conditional probability table P(Vector\_0|S\_0,S\_1). This reduces
the overall statespace of our model

\subsection{Using the Network}






