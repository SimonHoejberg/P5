\section{Domains and Probability}

The model is designed to be used to predict \texttt{Vector\_1}, to get an
accurate read on \texttt{Vector\_1} we need to model how the vectors relate to
the observations and each other, as well as what their domains are. There are
however limitations we need to consider. The model is constructed using Hugin
Lite \citep{Hugin}, this allows the model to be constructed easily but limits
the total number of nodes to 50 and the number of cases to 500, an example of a
case is a specific \texttt{S\_0} given a specific \texttt{Vector\_0}. This
limitation causes problems for how precise the model can be. This can be
especially problematic when we would like to use a variable to represent the
distance between the target and turret. To work around these limits we have
divided the domains into intervals. This means the model loses some precision,
but it allows us to work around the limit of 500 cases. This means that instead
of having only a few nodes with a continuous set of values we can consider
additional variables, which can make up for the loss of accuracy. It also allows
us group the values that seem unlikely to be used often into big intervals.

\subsection{Constructing the Probability Tables}

Most of the domains are easy to determine, for instance the domains for each of
the sensors are determined by what they can output. As such the domain for
distance has values between 0 and 255, as this is what the ultrasound sensor
can sense. These values can then be divided into 7 intervals, with the first and
the last intervals being the biggest, while the remaining 5 divide the rest of
the distance evenly between them, see \autoref{SD_table} \fix{}{insert probs}.

\begin{center}
\begin{table}[]
\label{SD_table}
\begin{tabular}{|l|l|l|l|}
\hline
\multicolumn{4}{|l|}{P(SD\_0|Vector\_0)} \\ \hline
Vector\_0     & (-0.15,-0.10) & (-0.15,-0.10)    & (-0.15,-0.10)  \\\hline   
[0 - 75]      &     &     &     \\ \hline 
[76 - 100]    &     &     &     \\ \hline
[101 - 125]   &     &     &     \\ \hline
[126 - 150]   &     &     &     \\ \hline
[151 - 175]   &     &     &     \\ \hline
[176 - 200]   &     &     &     \\ \hline
[201 - 255]   &     &     &     \\ \hline
\end{tabular}
\caption{The probability table for sensing the distance given the angle}
\end{table}
\end{center}

The domain for angle is likewise determined by how far the turret can turn,
which is 180 degrees. It is of note that the angle is set as 0 at the start of
program, in practice this means that domain for sensing the angle is from -15 to
165 degrees. \fix{}{explain -15 degrees is due to camera spotting and moving
too quickly}. 

\begin{center}
\begin{table}[]
\label{SA_table}
\begin{tabular}{|l|l|l|l|}
\hline
\multicolumn{4}{|l|}{P(SA\_0|Vector\_0)} \\ \hline
Vector\_0     & (-0.15,-0.10) & (-0.15,-0.10)    & (-0.15,-0.10)  \\\hline   
[-15 - 15]      &     &     &     \\ \hline 
[16 - 45]    &     &     &     \\ \hline
[46 - 75]   &     &     &     \\ \hline
[76 - 105]   &     &     &     \\ \hline
[106 - 135]   &     &     &     \\ \hline
[136 - 165]   &     &     &     \\ \hline
\end{tabular}
\caption{The probability table for sensing the angle given the vector}
\end{table}
\end{center}

The domain for time is the difference in time between the first observation and
subsequent observations. By determining the domain this way, we can reduce the
available domain to a manageable size, as it will at most be time it takes to
travel from one end of the shooting range to the others.\fix{}{explain the 4.5
limit to time}
\begin{center}
\begin{table}[]
\label{ST_table}
\begin{tabular}{|l|l|l|l|}
\hline
\multicolumn{4}{|l|}{P(ST\_0|Vector\_0)} \\ \hline
Vector\_0     & (-0.15,-0.10) & (-0.15,-0.10)    & (-0.15,-0.10)  \\\hline   
[0 - 1500]      &     &     &     \\ \hline 
[1501 - 3000]    &     &     &     \\ \hline
[3001 - 4500]   &     &     &     \\ \hline
\end{tabular}
\caption{The probability table for sensing time given a vector}
\end{table}
\end{center}

\fix{The domains for velocity is set to be three seperate speeds, 30cm/s, 36cm/s
and 42cm/s}{Is velocity used as a node, and are these the correct speeds?}.

The domain for the vectors is a bit more difficult to determine. \fix{In order
to determine this domain we have con ducted a tests where the vectors and the
data used were measured.}{Perform tests to determine vectors and their likely
hoods as well as all the other chances} By determining the vectors this way,
along with gathering the other data, get a rough estimate of the domain and
construct the probability tables for the other nodes.



\subsection{inference}
Since a bayesian network has been constructed we can abuse the property of
inference to reduce the size of our probability table. Instead of a joint
probability-table, it is sufficient instead to just look at the tables we are
interested in. Since our evidence will be inserted at our S-node and the
value we are looking for is contained in the node Vector it is thus enough to
provide the conditional probability table P(Vector\_0|S\_0,S\_1). This reduces
the overall statespace of our model

\subsection{Using the Network}





