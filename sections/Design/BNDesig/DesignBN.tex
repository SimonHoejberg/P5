\chapter{Model Design}

The purpose of the belief network is to predict the directional vector of the
target. This allows us to make a general prediction of the direction and speed
the target is travelling at. This can then be used to hit the target at a
specified point in time. \fix{This chapter will be used to describe the network
and explain the variables and their domain.}{}

\fix{}{Do we need to talk about creating our own version of
hugin? --> i would say it is outside the scope of the project}
\fix{}{Do we need to discuss a simpler model? --> maybe able to run on NXT but
must be very simple}

\section{Network Structure}
\fix{The network}{change to non-simple so we can better explain }, see
\autoref{SimplifiedBN}, displays a simplified version of the network. In this
network the sensor observations, \texttt{S\_0} to \texttt{S\_2}, are simplified
representations of the 3 sensor observations distance, angle and time. Worth
noting is that the observations does not occur on fixed intervals, but are
instead provided once the NXT has found acceptable data. This creates an
unpredictable behavior regarding when the observations occur, and thus we need
to contain a time-variable in our observation feature. Since the actual model
receives time in the observations it cannot be considered a true HMM, since in
an HMM time is considered as an implicit part of the variables, in the sense
that \texttt{S\_0} would be taken at time 0 while \texttt{S\_2} would be at
time 2.

\figx[0.8]{SimplifiedBN}{The model for the belief network}

The nodes are, as mentioned in \autoref{LabelBN}, variables and the edges are
the dependencies. The nodes \texttt{Vector\_0} and \texttt{Vector\_1} are used
to represent possible directional vectors the target can have. The edges moving
from \texttt{Vector\_0} to \texttt{S\_0} and \texttt{S\_1} represent both of
these nodes are dependant on \texttt{Vector\_0}. Since the sensor nodes are
comprised of distance, angle and time, we need two of such sets in order to reliably
determine a vector. \texttt{Vector\_1} has edges to \texttt{S\_1},
\texttt{S\_2}, this represent that the we use the previous observation to
determine a new direction vector. The edge going from \texttt{Vector\_0} to
\texttt{Vector\_1} represents that the previous direction vector can be compared
the new vector to the old. This vector can then be used in conjunction with the
last observation set to find the targets position at a given time.
