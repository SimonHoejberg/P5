\chapter{Model Design}

The purpose of the belief network is to predict the directional vector of the
target. This allows us to make a general prediction of the direction and speed
the target is travelling at. This can then be used to hit the target at a
specified point in time. \fix{This chapter will be used to describe the network
and explain the variables and their domain.}{}

\fix{}{Do we need to talk about creating our own version of
hugin? --> i would say it is outside the scope of the project}
\fix{}{Do we need to discuss a simpler model? --> maybe able to run on NXT but
must be very simple}

\section{Network Structure}
The network, see \autoref{BNDone}, displays a the final version of the network.
In this network the nodes \texttt{SA\_0-2} and \texttt{SD\_0-2} represent
observations of distance and angle. The numbers next to the nodes represent the
time at which they happen.

%  Worth noting is that the observations does
% not occur on fixed intervals, but are instead provided once the NXT has found
% acceptable data. This creates an unpredictable behavior regarding when the
% observations occur, and thus we need to contain a time-variable in our
% observation feature. Since the actual model receives time in the observations
% it cannot be considered a true HMM, since in an HMM time is considered as an
% implicit part of the variables, in the sense that \texttt{S\_0} would be taken
% at time 0 while \texttt{S\_2} would be at time 2.

\figx[0.8]{BNDone}{The model for the belief network}

The nodes are, as mentioned in \autoref{LabelBN}, variables and the edges are
the dependencies. The nodes \texttt{Vector\_1} and \texttt{Vector\_2} are used
to represent possible directional vectors the target can have. The edges moving
from \texttt{Vector\_1} to \texttt{SA\_0}, \texttt{SD\_0}, \texttt{SA\_1} and
\texttt{SD\_1} represent that all of these nodes are dependant on
\texttt{Vector\_0}. 
Since the sensor nodes are comprised of distance and angle, we need two
of such sets in order to reliably determine a vector. \texttt{Vector\_2} has
edges to \texttt{SA\_1}, \texttt{SD\_1}, \texttt{SA\_2} and
\texttt{SD\_2}, this represent that the we use the
previous observation to determine a new direction vector. The edge going from
 \texttt{Vector\_1} to \texttt{Vector\_2} represents that the previous direction
vector can be compared the new vector to the old. This vector can then be used
in conjunction with the last observation set to find the targets position at a
given time.