\chapter{Model Design}\label{BNModel}

The purpose of the belief network is to predict the vector of the
target. This allows us to make a general prediction of the direction and speed
the target is travelling at. This can then be used to hit the target at a
specified point in time. This chapter will be used to describe the network,
define the domains and construct the conditional probability tables for each of
the variables.
%\fix{}{Do we need to talk about creating our own version of
%hugin? --> i would say it is outside the scope of the project}
%\fix{}{Do we need to discuss a simpler model? --> maybe able to run on NXT but
%must be very simple}

\section{Network Structure}
\autoref{BNDone} displays the final version of the network. In this network the
nodes \texttt{SA\_0-2} represent angle observations while \texttt{SD\_0-2}
represent the distance observations.

\figx[0.8]{BNDone}{The model for the belief network}

Time is represented implicitly in the model by numbering the nodes 0-2, as such
\texttt{SD\_1} is taken at time 1. The actual time interval between each of
these observations is irrelevant, but can be used to provide a consistent data.
The interaction between the the observations and how they are used by the
network can be seen in \autoref{mwr}.

\figx[0.8]{mwr}{An example of how the observations are translated and inserted
into the nodes}

The network represents how the observations, angle and distance, are dependent
on the vector of the target. The nodes are, as mentioned in \autoref{LabelBN},
variables and the edges represent the dependencies. The nodes
\texttt{Vector\_1} and \texttt{Vector\_2} are used to represent possible
vectors the target can have. The edges moving from \texttt{Vector\_1} to
\texttt{SA\_0}, \texttt{SD\_0}, \texttt{SA\_1} and \texttt{SD\_1} represent
that all of these nodes are dependent on \texttt{Vector\_0}. Since the sensor
nodes are comprised of distance and angle, we need two of such sets in order to
reliably determine a vector. \texttt{Vector\_2} has edges to \texttt{SA\_1},
\texttt{SD\_1}, \texttt{SA\_2} and \texttt{SD\_2}, this represent that the we
use the previous observation to determine a new direction vector. The edge
going from \texttt{Vector\_1} to \texttt{Vector\_2} represents that new vector
is dependent on the old, in the sense that the vector should not change wildly,
unless a faulty observation has occurred.

\subsection{Filling in the tables}
For each speed we want to implement in the model, we need a dataset, based on
the sensor readings from turret. In order to gather datasets a test was
performed. The test was conducted such that the observations were grouped into a
set: each set consists of 3 sensor readings, the first two observations of a set
are used to construct a vector V1 and the last two observations are used to
construct a vector V2. Using this method we end up with a set of vectors based
upon sensor observations, which means we know the values that resulted in the
vector, which will come in handy in the next step.
Next step is to group the resulting vectors according to whether or not it was
deemed a good vector or a bad vector. After going through the vectors for each
speed, we end up with two sorted list of vectors, one containing all the good
vectors and one containing the faulty ones. Next step is plotting the 
measurements that resulted in a given vector into a table, based on the sensor
readings - the way this was accomplished was to look at a given vector and then
reference the sensor readings. For example, we wanted to know the probability of
a vector given an angle - so what we did was make a table with two columns -
good and bad indicating the quality of the vector. We then referenced the
specific vector and then plotted in the two angle values used to calculate the
given vector. Doing this for each vector, we end up with a probability table.
The same method is used to construct a probability table for distance.

 
 %  Worth noting is that the observations does
% not occur on fixed intervals, but are instead provided once the NXT has found
% acceptable data. This creates an unpredictable behavior regarding when the
% observations occur, and thus we need to contain a time-variable in our
% observation feature. Since the actual model receives time in the observations
% it cannot be considered a true HMM, since in an HMM time is considered as an
% implicit part of the variables, in the sense that \texttt{S\_0} would be taken
% at time 0 while \texttt{S\_2} would be at time 2.
