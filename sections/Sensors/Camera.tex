\section{Camera}\label{SensorTheory}
For the purpose of identifying and tracking a target, the turret will
make use of a specialized camera, which is capable of identifying coloured
objects. In order to use the camera, the user specifies a colour of choice,
which will then be searched for. The colours are set using a program called
``NXTCamView'''. When the camera registers an object of this colour, it will
build a datalogical representation of the object consisting of a number of
rectangles. Such a representation can be seen in \autoref{TwoBlobs2.PNG}.

\figx[0.70]{TwoBlobs2.PNG}{Visual representation of the output from the camera.}

For each identified rectangle, the camera outputs the coordinates for that
rectangle's sides, top, bottom and colour. 

Below are the technical specifications \citep[p. 1]{NXTCam} for the camera.
\begin{itemize}
  \item Tracks up to 8 targets
  \item Analyzes 30 frames per second
  \item Uses a resolution of 88x144 pixels
  \item Uses a standard NXT sensor port
  \item Resulting data consists of: number, color and coordinates of these
  objects.
\end{itemize}

\section{Camera Sensor Test}\label{CamTest}%Will most likely need a better
Testing the camera aims at answering two main questions. Namely
what colours can be easily recognized, and at what distance can the target be
identified. As such, this subsection will consist of a test setup, the resulting
data and the conclusions which can be drawn from this data.

\subsection{Camera Test - Setup}\label{CamTestSetup}
The setup for testing the camera consists of placing the turret at one end of a
long table, pointing towards a target at the other end. The table is marked with
a number of lines each representing a distance of 25cm from the turret's
location. The target for this test is a rectangular cardboard cutout, which is
used to hold a sheet of coloured paper which is folded around it. Behind the
coloured paper is a dark background which is used in some, but not all
tests.
This setup can be seen in \autoref{SensorCamSetupV2}.

\figx[0.15]{SensorCamSetupV2}{Setup for the camera test (With Background).}

\subsection{Camera Test Results - Colours}\label{CamTestColours}
In order to determine what colour can best be detected, the target is wrapped
with a piece of coloured paper, and is then placed at a distance of 2m from the
turret. The camera is then used to search for the specified colour. As explained
in \autoref{SensorTheory} this data consists of rectangular segments of the
identified colour, and these segments sizes and colours. The quality of the
individual colours is determined by looking at two factors. The first factor is
the consistency of the recorded data, which is determined by the variance in the
data. The second factor is the cameras ability to keep recognizing the target
at different distances.\nl

The first test investigates the consistency of the recorded data. To do this we
look at three main points. The maximum and minimum measured areas, the difference
between these points, and the average measured area. In addition each colour is
given a rating based on the amount of frames where a colour was recognized.
These ratings are: Ok (~70-100\%), bad (~40-69\%), very bad (~20-39\%) and Null
(No data). It should also be noted that the results from this test have been
observed through the NXTCam calibration software, and that the data points have
been manually observed and written down. The test was done twice, once with a
background of a contrasting colour, and one without. The results from the first
test can be seen in \autoref{table:CamTest1}.

\dataTable{SensorCamTest1}{Data from test 1. (With background)}{CamTest1}

From the data presented in \autoref{table:CamTest1} we can conclude the
following:
\begin{itemize}
  \item The blue and yellow colours can not be recognized
  \item The green colour has the highest variation of 163 pixels
  \item The green colour has a low amount of frames with recognition
  \item The red colour has the lowest variation of 19 pixels
  \item Of the 'ok' colours, orange and red, orange has the largest measured
  average area.
\end{itemize}

Based on these conclusions we can determine that the best colours would be
either red or orange. In order to verify this assumption the test is repeated
without the contrasting background. The results from the second test can be seen
in \autoref{table:CamTest2}.

\dataTable{SensorCamTest2}{Data from test 2. (Without background)}{CamTest2}

From the data presented in \autoref{table:CamTest2} we can conclude the
following:
\begin{itemize}
  \item For some colours the contrasting background results in worse data
  \item Given no contrasting background 'yellow' can be seen, but the data is
  'bad'
  \item Given no contrasting background 'red' can no longer be recognized
  \item Given no contrasting background 'orange' has no variance in data
\end{itemize}

\fix{Based on these conclusions we can determine that orange would be colour of
choice, as the camera is capable of recognizing the colour without any variance
in the recorded data.}{But we use red???}

\subsection{Camera Test Results - Distance}\label{CamTestDistance}
Based on the choice made in \autoref{CamTestColours} to use orange as the colour
of the target, this section will be used to further determine how well this
colour can be used at different distances. For this purpose, the target will be
placed at various distances from the target, as noted in \autoref{CamTestSetup}.
The measurements will use the same ratings 'ok', 'bad', 'very bad' and 'null',
as described in \autoref{CamTestColours}. The resulting data from this test can be
seen in \autoref{table:CamTest3}.

\dataTable{SensorCamTest3}{Quality of data at different distances.}{CamTest3}

The data presented in \autoref{table:CamTest3} represents the camera's ability
to recognize the target at different distances, but during the testing a major
flaw was identified. It was found that the the measured colour of the target
changes based on its distance from the turret and the light levels at its
current location. In effect this means that no single colour can be used to
recognize the target at all distances, but instead multiple colours need to be
recognized based on the distance to the target. This problem has two identified
solutions, either the camera is programmed to recognize a number of different
shades of orange, or the turret will be limited to only operate in well and
consistently lit areas.