\section{Sensor Theory}\label{SensorTheory}
For the purpose of sensing its environment, the turret will need two key
features: the ability to identify a target, and the ability to measure the
distance to said target. The identification process will be used to find and
track a target, and the distance measurement will be used to accurately use the
formula for projectile motion described in \autoref{ProjMotion} to hit the
target.

\subsection{Colour Camera}\label{SensTheoCam}
For the purpose of identifying and tracking a target, the developed turret will
make use of a specialized camera, which is capable of identifying coloured
objects. In order to use the camera, the user specifies a colour of choice,
which will then be searched for. When the camera registers an object of this
colour, it will build a datalogical representation of the object consisting of a
number of rectangles. Such a representation can be seen in
\autoref{TwoBlobs2.PNG}. 

\figx[0.70]{TwoBlobs2.PNG}{Visual representation of what the camera senses.}

For each identified rectangle, the camera will output the coordinates for that
rectangle's sides, top, bottom and colour. 

Below are the technical specifications\citep[p. 1]{NXTCam} for the camera.
\begin{itemize}
  \item Tracks up to 8 targets
  \item Analyzes at 30 frames per second
  \item Uses a resolution of 88x144 pixels
  \item Uses a standard NXT sensor port
  \item Resulting data consists of: number of objects, color of objects and
  bounding box coordinates of these objects.
\end{itemize}

\subsection{Ultrasonic Sensor}\label{SensTheoUltra}
For the purpose of measuring the distance to the chosen target, the turret will
make use of two ultrasonic sensors. These sensors work by producing sound waves
from one of two frontal holes, which is then bounced back by an object and
received by the second hole. An image of this sensor can be seen in \autoref{}.
The delay between producing and receiving the sound wave is used to determine
the distance to a given object.

\figx[0.2]{UltrasoundPic.jpg}{The two ultrasonic sensors and the camera mounted
on the turret.}

The described sensor is a standardized ultrasonic sensor from Lego NXT. The
sensor's technical specifications\citep{Ultrasonic_sensor} can be found below.
\begin{itemize}
  \item Has a max range of 2.5m
  \item Has a precision of +/- 3cm
  \item Looses precision after 1.5m
  \item Has a 25ms delay between readings
  \item Works best with large, hard-sized objects
  \item Each measurement outputs a single integer
\end{itemize}

% The product description says that it has a maximum range of 2.5m with
% a precision of +/- 3cm \citep{Ultrasonic_sensor}\KT. In actuality the product
% that was used begins loose precision after anything further away than 1.5m.
% 
% There is a delay of 25ms at how often it can be used to determine the distance
% to an object. Anything faster than that and the results become very imprecise
% \Source.

% \section{Infrared Sensor}
% The infrare sensor works on the same basic prinicple as the ultrasonic sensor,
% insted of it using sound it uses infrared light, and mesures the intensitivit of
% the light reflected by the object infront of the sensor, if there no reflected
% light are then no object is infront of the sensor and if the amount of reflected
% light is high then the object is close to the sesnor \Source.

% En infrarød sensor virker
% på samme princip som ultra sonic sensor, her er det så i stedet for at måle hvor lang tid der går, så gør den brug af intentiviten af
% lyset der bliver kast tilbage til at udregner hvr langt væk et objekt er. Hvs
% der ingen lys på den bestemmet bølgelængde som der bliver opfanget af lys
% sensren inde i den infrarøde sensor så er der ingen objekt foran den, mens hvis
% der er et objekt meget tæt på kommer der meget lys tlbage \Soure. 





%http://www.education.rec.ri.cmu.edu/previews/nxt_products/robotics_eng_vol_1/preview/content/reference/helpers/ultrasonic.htm

%http://www.techno-stuff.com/irrd.htm

%http://www.mindsensors.com/ev3-and-nxt/112-high-precision-medium-range-infrared-distance-sensor-for-nxt-or-ev3

%http://education.rec.ri.cmu.edu/content/electronics/boe/ir_sensor/1.html



