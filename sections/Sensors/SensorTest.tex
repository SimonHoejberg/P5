\section{Sensor Testing}
The two sensors used on the turret both have datasheets with technical
documentation, but in order to verify this information, and to gain a
deeper understanding of how to use these sensors, this section will be used to
test their individual features. The first test in \autoref{UltraTest} will be
used to determine the accuracy and limitations of the ultrasonic sensor. The second
test in \autoref{CamTest} will be used to determine how the camera can best be
used to identify a target, and how the resulting data can be accurately used. 

% First the
% testing methodology will be explained for each test, and then this methodology
% will be used to devise tests for the two sensors. 
% 
% In this chapter the different sensors will be tested, the reasoning for testing
% the sensors is that we need to find out how precise the sensors are and how far
% they can measure. There are two sensor which needs to be tested, one is an
% ultrasonic sensor which is used for measuring the distance. The other sensor
% which needs to be tested is a camera for tracking specific colours. We will then
% use the results from these test to utilize the full potential of the sensors. 
 
% To find out which sensors are the optimal
% sensors for the project, this chapter will be used to test the different sensors. The testing of the
% sensors will also be used to find out how precise they are.\nl
% % 
% First the testing methodology will be explained, then this methodology
% will be used to devise tests for the different sensors.

% \subsection{Testing Methodology}
% The testing will be divided into to two parts, one part is testing the two
% distance sensors and the second part is testing the camera. 

\subsection{Ultrasonic Sensor Test}\label{UltraTest}
The test of the ultrasonic sensor was divided into 4 different experiments, the
difference between the experiments are the size of the object being tracked and
whether the sensor is shielded or not. The different object sizes was tested
with a shielded sensor and a non shielded sensor. \nl

\graphL{Actual distance [cm]}{Measured distance
[cm]}{shieldedBoxStanding,shieldedBoxLandscape}{Box Standing, Box
Landscape}{Results with a shielded sensor and a box h.
14.2cm and w. 9cm}{shieldedTestBox}

\graphL{Actual distance [cm]}{Measured distance
[cm]}{unshieldedBoxStanding,unshieldedBoxLandscape}{Box Standing, Box
Landscape}{Results with a unshielded sensor and a box h.
14.2cm and w. 9cm}{unshieldedTestBox}

\graphL{Actual distance [cm]}{Measured distance
[cm]}{shieldedPaperStanding,shieldedPaperLandscape}{Paper Standing, Paper
Landscape}{Results with a shielded sensor and a A4 Paper}{shieldedTestPaper}

\graphL{Actual distance [cm]}{Measured distance
[cm]}{unshieldedPaperStanding,unshieldedPaperLandscape}{Paper Standing, Paper
Landscape}{Results with a unshielded sensor and a A4 Paper}{unshieldedTestPaper}


As seen in \autoref{graph:shieldedTestBox}, \autoref{graph:unshieldedTestBox},
\autoref{graph:unshieldedTestPaper} and \autoref{graph:shieldedTestPaper} the
sensor was able to determine the distance with better precision when it is
shielded and the object was in the landscape position. \nl

It can then be concluded that the shielded sensor performed a lot better than
the unshielded, it can also be concluded that the placement of the object can
have a big impact. This is due to the fact that the sensor does not have a very
large field of view. Therefore, the object should be wider rather than taller. 

% \subsection{Infrared}
% The test of the infrared sensor was not able to be conducted since there were
% some issues with the different sensors. The first sensor did not work at all.
% Since it was an infrared sensor, a camera was used to see if the infrared
% LED lid up. Since it did not light we concluded that the sensor was broken. \nl
% 
% When trying to test the other sensor, first it did not give a reading so the
% same test was conducted using a camera to see if the sensor was broken. It was
% concluded that the sensor was not broken. It was then discovered that it was a
% problem with the software, where the address for the reading the sensor on the
% I2C port was changed. This meant that it was not possible to get a proper
% reading. The readings were either 0 or a reading above 60.000 depending on which
% port was used. \nl
% 
% Another issue with using the infrared sensor is the range. The medium range
% infrared sensor only has a range from 10 to 80 cm. Therefore this sensor could
% become a very limiting factor in regards to how far the target can be placed
% from the cannon. 

\subsection{Camera Sensor Test}\label{CamTest}%Will most likely need a better
Testing the camera will consist of trying to answer two main questions. Namely
what colours can be easily recognized, and at what distance can the target be
identified. As such, this subsection will consist of a test setup, the resulting
data and the conclusions which can be drawn from this data.

\subsubsection{Camera Test - Setup}
The setup for testing the camera consists of placing the turret at one end of a
long table, pointing towards a target at the other end. The table is marked with
a number of lines each representing a distance of 25cm from the turret's
location. The target for this test is a rectangular cardboard cutout, which is
used to hold a sheet of coloured paper which is folded around it. Behind the
coloured paper is a dark background which is used in some, but not all tests.
This setup can be seen in \autoref{SensorCamSetup.JPG}.

\figx[0.13]{SensorCamSetup.JPG}{Setup for the camera test (With Background).}

\subsubsection{Camera Test Results - Colours}
In order to determine what colour can best be detected, the target wrapped with
a piece of coloured paper, and is then placed at a distance of 2m from the
turret. The camera is then used to search for the specified colour. As explained
in \autoref{SensTheoCam} this data consists of rectangular segments of the
identified colour, and these segments sizes and colours. The quality of the
individual colours is determined by looking at two factors. The first factor is
the consistancy of the recorded data, which is determined by the variance in the
data. The second factor is the cameras ability to correctly   


 The results from this test can be seen in
\autoref{SensorCamTest1}. 

\dataTable{SensorCamTest1}{test}{test}














several parts, these parts will heavily rely
on the technical specifications of the camera. Therefore the first part of this
section will be on the technical specifications of the camera and the needed
software.\nl

The camera runs at 30 fps and with a resolution of 88 x 144. This could
potentialy become an issue due to the fact that it is a very low resolution.
Although this is a could be a problem, it depends on how large the
object/objects being tracked are.
The camera also has the ability to track up to 8 different objects
\fix{\Source}{Kan vi refere til de tekniske specfikationer i user guide}. \nl

The setup of the camera is quite simple and is done using a computer, to setup
which color it is suposed to track. The process of choosing the color is done
using some software that comes with the camera, here a colormap is made which
tells the preprocessor on the camera which colors to track. 

There are three different test that can be done, the first to 
determine which colour the camera best sees, the second to see how
far the camera can track an object and the third to determine if lighting up the
object changed the cameras abillity to track the colour \/ object.\nl

The first test is to determine what colour the camera best can see, to do this
a square of paper with the same size will be used to make sure that the only
diffence is the colour of the paper. The setup of the test is that the camera
should stay at the same place so that the lighting condition stays the same,
then the paper square is place at a certain distance to the camera. The result
of each test is how many blobs of the tracking was the recordnised and how
consisten the size of the coloured blob is.\nl

The second test, to test how far the camera can track objects is to do this the
turret it setup in the same manner as the test before, here the object which is
the paper is moved further and further away, optional could the colours of the
paper be changed to see how well the different colours can be seen from a
distance. The results of this test is the distance that the camera can see each
colour and again how consisten it is.\nl

The third test, to test weather lighting up the object with an flashlight
improves how well the camera sees the object, again this test needs to bed setup
in the same manner at the two other test with the turret in the same place
thoughout the test and then the test pocedes as test 2 with observering how far
away it can see the object and as a result determine if the flashlight is
nessery and what colour works best with it.

