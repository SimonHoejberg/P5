\chapter{Introduction}
Computers have become an integral part of modern society, not just as personal
computers or servers, but also in the most minute of electronic devices
like kitchen appliances and radiator thermostats. This entire class of minute
computers surrounds us in much greater numbers than ever before, although their
presence is not as noticeable as for example a personal computer. These
computers are integrated into a large part of all electronic devices. This class
of computer systems is also known as embedded systems.\nl


%Focuserer vi p� at optimizere softwaren?
This projecs focuses on embedded systems and ends up with a physical product,
which qualifies as an embedded system. This presents some issues as
developing for embedded systems impose hardware limitations in the form of a
limited amount of memory and processing power. Therefore, this project will be
focused on designing, implementing and optimizing a software solution for an
embedded hardware platform. Furthermore, as the project group is partaking in
an introductory course to machine intelligence, this project will include
aspects relating to implementation of machine intelligence models.
% The theme of this semester is Embedded systems. This means that the focus of
% the project will be working with a set of constraints that must be balanced.
% Furthermore the project should include techniques from the machine intelligence
% course and the real-time systems course.

\section{Initial Problem}\label{InitialProblem}
During the initial semester introduction, a recurring project concept was
proposed by the semester coordinator Ren\'e Hansen. This proposal took the
form of the following challenge:

\begin{center}
\begin{minipage}{0.8\linewidth}
\textit{Design and construct an autonomous turret such that it is capable of
hitting a moving target.}
\end{minipage}
\end{center}

Based on the desire to learn from the experience, as well as the interesting
topic and possible challenges, this proposal was chosen as topic for this
semester project. To further define the challenge, Ren\'e added that the target
should:
\begin{center}
\begin{minipage}{0.8\linewidth}
\textit{Move at a velocity such that the turret will be incapable
of hitting it without determining an offset to shoot with.}
\end{minipage}
\end{center}

To elaborate this means that if the turret were to simply fire without an
offset, the target would simply move out of the way. This creates the
constraint, that the turret should be able to calculate where the target will be
when the projectile reaches it, and be able to use this information to
accurately hit the target while in motion. This challenge results in the initial
problem:

\begin{center}
\begin{minipage}{0.8\linewidth}
\textit{How can an autonomous turret be designed and constructed such that it is
capable of hitting a moving target.}
\end{minipage}
\end{center}

Based on the namesake of this project 'Turret Systems 101', from here on out the
product will be refered to as \namep.
