\chapter{Introduction}
Computers have become an integral part of modern society, not just as personal
computers or server stations, but also in the most minute of electronic devices
like kitchen appliances and radiator thermostats. This entire class of minute
computers surrounding us in much greater numbers, although their presence is a
lot more incognito. These computers are built into larger electronic devices.
This class of computer systems is also known as Embedded Systems.\nl

The theme of this semester is Embedded systems. This means that the focus of
the project will be working with a set of constraints that must be balanced.
Furthermore the project should include techniques from the machine intelligence
course and the real-time systems course.

% Computers are an integral part of modern society, they are used in everything
% from assemblyline factories to kitchen appliances. More and more they are being
% used in daily life, without even being noticed. A computer now covers much more
% than just a desktop, laptop, tablet or a server. There is an entire class of
% computers surrounding us in much greater numbers, although their presence is a
% lot more incognito. These computers are built into larger electronic devices.
% This class of computer systems is also known as Embedded Systems.

\section{Initial Problem}
A recurring project proposal idea is a challenge proposed by the semester coordinator Ren\'e
Hansen. The challenge is:

\begin{center}
\begin{minipage}{0.8\linewidth}
\textit{Design and construct an autonomous turret such that it is capable of
hitting a moving target.}
\end{minipage}
\end{center}

Based on the desire to learn from the experience, as well as the interesting
topic and possible challenges, this proposal was chosen as topic for this
semester project. To further define the challenge, Ren\'e added that the target
should:
\begin{center}
\begin{minipage}{0.8\linewidth}
\textit{Move at a velocity such that the turret will be incapable
of hitting it without determining an offset to shoot with.}
\end{minipage}
\end{center}

To elaborate this means that if the turret were to simply fire without an
offset, the target would simply move out of the way. This creates the
constraint, that the turret should be able to calculate where the target will be
when the projectile reaches it, and be able to use this information to
accurately hit the target while in motion. This challenge results in the initial
problem:
% Based on desire to learn from the experience, as well as the reason ``shooting
% stuff is fun'', it was decided to work with this project. This produces the
% question:\nl

% One of the project proposals was a challenge by the semester coordinator Ren\'e
% Hansen to build an autonomous turret that is capable of hitting a moving target.
% Based on desire to learn from the experience, as well as the reason ``shooting
% stuff is fun'', it was decided to work with this project. This produces the
% question:\nl
\begin{center}
\begin{minipage}{0.8\linewidth}
\textit{How can an autonomous turret be designed and constructed such that it is
capable of hitting a target in motion without the use of user-input.}
\end{minipage}
\end{center}

% \textit{Is it possible to construct an embedded system that is able to fire and
% hit a target in motion without user-input?}
