\chapter{Data for the Belief Network}\label{AppendixMIData}
The following is the collection of 2 sets of 10 tests made by \name. Each test
produced three readings of the x- and y- coordinate, the time, the angle and the
distance to the target. \autoref{table:Data60} is the set taken at a speed
41cm/s while \autoref{table:Data80} is taken at a speed of 57cm/s. The sets were
collected by having the target move perpendicular to the turret.
\dataTable{MIData60}{The data from where the target has a speed of
41cm/s}{Data60} \dataTable{MIData80}{The data from where the target has a speed
of 57cm/s}{Data80}

When analysed these observations provide the vectors which can be seen in
\autoref{table:Vec80} and \autoref{table:Vec60}. This is done by creating V1 as
a vector representing the difference between the second and first observation,
in addition it is normalized such that it appears as in the form cm/ms. V2 is
likewise created as a representation of the difference between the third and
second observation, also normalized by time.

\dataTable{Vec80}{The resulting vectors based on the data from the set
at 57cm/s}{Vec80}
\dataTable{Vec60}{The resulting vectors based on the data from the set
at 41cm/s}{Vec60}
