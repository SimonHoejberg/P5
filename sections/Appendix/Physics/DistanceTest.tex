\chapter{Distance Test}\label{AppendixDistTest}
During development of the autonomous turret, tests have been made in order to
determine how far the turret can shoot its projectiles. This appendix will be
used to present these tests and its resulting conclusions.

\section{Test Setup}
In this test the turret is placed next to a wall which have been marked with
distance indicators. The turret is facing right, and all tests are video
recorded. This setup can be seen in \autoref{DistanceTestSetup.JPG}.
\figx[0.1]{DistanceTestSetup.JPG}{Experiment setup for testing projectile
distance.} In \autoref{DistanceTestSetup.JPG} the turret's muzzle is placed exactly 1 meter
from the distance markings. As such, each full line on the markers represent a
distance of 1 meter. A paper has been placed in the background in order to
identify the recorded test when evaluating the videos. 

\subsection{Test Execution}
In each test the turret starts in its resting position at an angle of ~5
degrees. The turret then angles itself to the given position and fires a single
projectile. This procedure is video recorded in order to accurate determine
exactly where the projectile landed. This is done a number of times for each
given angle. Video analysis is done using the program Tracker, which is
developed as a part of the Open Source Physics program.

\section{Resulting Data}
The resulting data from these tests consists of measured angles and their
respective distance. This is complemented with the average distance for each
angle and the calculated variance.
\dataTable{DistanceTestData}{Angle/Distance realationship}{AngDistRel} 

For each measured angle this gives the following average distance:
\dataTable{DistanceTestDataAVG}{Average Angle/Distance
realationship}{AngDistRelAVG}

\subsection{Theorem Comparison}
The data from these tests have been gathered imperically, but in order to
determine their quality, or to identify any unknown variables, it is necessary
to compare them to the theoretical results. This can be determined by using
\autoref{FinEq2} which was described in \autoref{CalcDist}:
\begin{equation}\label{FinEq2}
x\ =\ \frac{v_0*cos(a)}{g}* \left(
v_0*sin(a)+\sqrt{v_0^2*sin(a)^2+2*g*(h+r*sin(a))}\right)+r*cos(a)
\end{equation}

This results in the following data:
\dataTable{DistanceTestTheoretical}{Theoretical Angle/Distance
realationship}{AngDistTheo}

\section{Conclusion}
It appears that the theoretical data is off from the test data. We need to
figure out why!
