\chapter{Distance Test}\label{AppendixDistTest}
During development of the autonomous turret, a test was made in order to
determine how far the turret can shoot its projectiles. In addition, video
recordings of the tests will be used to determine the muzzle velocity of the
projectile. This appendix will be used to present this test and the resulting
conclusions.

\section{Test Setup}
In this test the turret is placed next to a wall which have been marked with
distance indicators. The turret is facing right, and all tests are video
recorded. This setup can be seen in \autoref{DistanceTestSetup.JPG}.
\figx[0.1]{DistanceTestSetup.JPG}{Experiment setup for testing projectile
distance.} In \autoref{DistanceTestSetup.JPG} the turret's muzzle is placed exactly 1 meter
from the distance markings. As such, each full line on the markers represent a
distance of 1 meter. A paper has been placed in the background in order to
identify the recorded test when evaluating the videos. 

\subsection{Test Execution}
In each test the turret starts in its resting position at an angle of ~5
degrees. The turret then angles itself to the given position and fires a single
projectile. This procedure is video recorded in order to accurate determine
exactly where the projectile landed. This is done a number of times for each
given angle. Video analysis is done using the program Tracker, which is
developed as a part of the Open Source Physics program \cite{Tracker}.

\section{Resulting Data}
The resulting data from these tests consists of measured angles and their
respective distance. This is complemented with the average distance for each
angle and the calculated variance.
\dataTable{DistanceTestData}{Angle/Distance realationship}{AngDistRel} 

For each measured angle this gives the following average distance:
\dataTable{DistanceTestDataAVG}{Average Angle/Distance
realationship}{AngDistRelAVG}

\subsection{Theorem Comparison}
The data from these tests have been gathered imperically, but in order to
determine their quality, or to identify any unknown variables, it is necessary
to compare them to the theoretical results. This can be determined by using
\autoref{FinEq2} which was described in \autoref{CalcDist}:
\begin{equation}\label{FinEq2}
x\ =\ \frac{v_0*cos(a)}{g}* \left(
v_0*sin(a)+\sqrt{v_0^2*sin(a)^2+2*g*(h+r*sin(a))}\right)+r*cos(a)
\end{equation}

This results in the following data:
\dataTable{DistanceTestTheoretical}{Theoretical Angle/Distance
realationship}{AngDistTheo}
From this data we can see a large difference between the theoretical distance
and the measured distance. While the differences exist, a pattern can still be
recognized, where the maximum distance is recorded at 45 degrees, and the
distance depends on the angle of fire. 

\section{Conclusion}
The results from this test indicates that there is a difference between the
theoretical method for determining the angle, and the actual resulting angle. As
we have been unable to identify the problem, we have determined to use a
different approach for determining the angle, which is presented below in
\autoref{aCalc}. In addition, by analyzing the video recordings of the tests, it
was found that the muzzle speed of the projectile was 4.82m/s.

\subsection{Angle Calculation}\label{aCalc}
In order to determine which angle to fire at in order to hit the target, we have
chosen an approach were we map the data from this test into a function. This is
done using the trendline determined from the sets of gathered data. This results
in the following formula:

\begin{equation}\label{angleCalc}
Angle\ =\ abs(0.2573 * distance - 22.85)
\end{equation} 

While this approach is theoretically more inaccurate, it should be
accurate as long as the distance to the target does not exceed the projectiles
maximum range. In addition, as the turret does not need to shoot a target over a
longer distance, this approach should be satisfactory for this project.
