\chapter{Accuracy Test}\label{AppendixAccTest}
During development of the turret, a test was made in order to determine the
accuracy of shots fired by the turret. That is, given a number of shots at the
same location, what is the chance that the turret hits the specified location.

\section{Test Setup}
In this test the turret is placed next to a wall which have been marked with
distance indicators. In essence this test setup is the same as in
\autoref{AppendixDistTest}, but the area in which the projectiles land is
covered in flour. This flour is used to indicate exactly where the projectile
landed.

\subsection{Test Execution}
At the beginning of a test, the turret is set to angle itself to the designated
angle. The turret is manually set to fire a single projectile, and the location
which the projectile hit is then noted down. This is repeated 5 times for each
angle, and for 6 different angles. The whole test has been repeated once, such
that the resulting data is split into two different sets.

\section{Resulting Data}
The resulting data consists of a number of tests. Each test results in an angle
coupled with all the respective distances. Each test is also complemented with
width between the middle of the turret, and the point the projectile hit.

\subsection{Set 1}
The data from the first data set can be seen below in \autoref{table:FTest1}. 
\dataTable{FlourTest1}{Angle/Distance relationship coupled with spread.}{FTest1} 

\subsection{Set 2}
The data from the second data set can be seen below in \autoref{table:FTest2}. 
\dataTable{FlourTest2}{Angle/Distance relationship coupled with
spread.}{FTest2}

\section{Data Interpretation: Width Spread}
From the data it can be seen that the distance the projectile has to travel
decreases the chances of accurately hitting the target.The average spread for
each set is shown below in \autoref{table:AVGSpread}.

\dataTable{AVGSpread}{Average spread recorded in the two data sets.}{AVGSpread}

While it is not a linear increase in spread, the locations further away had
considerably more spread that those which were closer. 

\section{Data Interpretation: Length Spread}
Another problem with the accuracy of the turret lies in the fact that there is a
variety in the distance the projectile will travel at a given angle. This is shown below in
\autoref{AngleVsDistSpread}, where the average spread in distance increases the
farther the projectile should be fired. In addition it has also been noted that
the turret has a chance to misfire. When this happens, the projectile will not
follow its usual arch, and will travel a considerately shorter distance.

\dataTable{AngleVsDistSpread}{Average spread recorded in the two data
sets.}{AngleVsDistSpread}

The data shown in \autoref{AngleVsDistSpread} was calculated by looking at all
of the recorded distances.
As stated, the turret has a chance to misfire. 


\section{Conclusion}
It appears that the theoretical data is off from the test data. We need to
figure out why!
