\chapter{Spread Test}\label{AppendixAccTest}
After developing the design for the turret, a test was made in order to
determine how accurate the turret at hitting a target location. That is,
given a number of shots at the same location, what is the chance that the turret
hits the specified location. This test differs from the one in
\autoref{AppendixDistTest}, as this test focuses on the chance to hit, instead
of how far the turret can shoot.

\section{Test Setup}
In essence this test setup is the same as in \autoref{AppendixDistTest} where
the turret fires along a wall, but in this test the area in which the
projectiles land is covered in flour, which is used to indicate exactly
where the projectile landed.

\subsection{Test Execution}
At the beginning of a test, the turret is set to angle itself to the designated
angle. A button is then pressed in order to manually fire a single projectile.
This could introduce errors as there is an interaction with the turret. The
location, in which the projectile hit, is clearly indicated in the flour. This
is repeated 5 times for each angle, and for 6 different angles. The test
was done twice, such that the resulting data is split into two different sets.

\section{Resulting Data}
The resulting data consists of a number of tests. Each test resulted in an
angle coupled with the respective distances. Each test is also complemented
with a width between the middle of the turret, and the location the projectile
hit.
This is done in order to show the spread in two dimensions.

\subsection{Set 1}
The data from the first set can be seen below in \autoref{table:FTest1}. 
\dataTable{FlourTest1}{Angle/Distance relationship coupled with spread.}{FTest1} 

\subsection{Set 2}
The data from the second set can be seen below in \autoref{table:FTest2}. 
\dataTable{FlourTest2}{Angle/Distance relationship coupled with
spread.}{FTest2}

\section{Data Interpretation: Width Spread}
From the data it can be seen that the distance the projectile has to travel
decreases the chances of accurately hitting the target. The average spread for
each set is shown below in \autoref{table:AVGSpread}.

\dataTable{AVGSpread}{Average spread recorded in the two data sets.}{AVGSpread}

While it is not a linear increase in spread, the locations further away had
considerably more spread than those which were closer. Assuming we
ignore the data from the misfired shots, we get the following average width-spreads. 

\begin{table}[H]
\centering
\begin{tabular}{|l|l|l|l|l|l|l|}
\hline
Angle (deg) & 4     & 13    & 22    & 31    & 40    & 49    \\ \hline
Spread (cm) & 3.357 & 1.583 & 2.917 & 3.861 & 6.607 & 3.583 \\ \hline
\end{tabular}
\caption{Average width-spread for the different angles.}
\end{table}

\section{Data Interpretation: Length Spread}
Another problem with the accuracy of the turret lies in the fact that there is a
variety in the distance the projectile will travel at a given angle. This is shown below in
\autoref{table:AngleVsDistSpread}, where the average spread in distance
increases the further the projectile is fired. In addition it has also been
noted that the turret has a chance to misfire. When this happens, the projectile will not
follow its usual arch, and will travel a considerately shorter distance. This
misfire chance was found to be 15\% on average.

\dataTable{AngleVsDistSpread}{Average spread recorded in the two data
sets.}{AngleVsDistSpread}

The data shown in \autoref{table:AngleVsDistSpread} was calculated by looking at
all of the recorded distances. Assuming we ignore data from misfired shots, the
average length-spread for the different angles are:

\begin{table}[H]
\centering
\caption{Average length-spread for the different angles.}
\begin{tabular}{|l|l|l|l|l|l|l|}
\hline
Angle (deg) & 4 & 13 & 22   & 31   & 40 & 49 \\ \hline
Spread (cm) & 4 & 5  & 13.5 & 12.5 & 20 & 9  \\ \hline
\end{tabular}
\end{table}


\section{Conclusion}
By looking at this test we can conclude the following:
\begin{itemize}
  \item The turret has an average chance to misfire of 15\%.
  \item If we disregard the misfires, the turret has a average width-spread of
  3.357cm at 4 degrees, and up to 6.607cm at 40 degrees.
  \item If we disregard the misfires, the turret has an average length-spread of
  4cm at 4 degrees, and up to 20cm at 40 degrees.
  
\end{itemize}

This test has shown a large difference between results for the
same angle. While this does pose a problem for the turret, we have been unable
to identify which factors causes this behaviour. As there is no clear
correlation between the angle of fire and the spread, we assume that the problem
lies with the firing mechanic itself. 
