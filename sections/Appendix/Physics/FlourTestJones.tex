\chapter{Hit-Accuracy Test}\label{AppendixAccTest}
After developing the design for the turret, a test was made in order to
determine how accurate the turret could shoot at a target location. That is,
given a number of shots at the same location, what is the chance that the turret
hits the specified location. This test differs from the one in
\autoref{AppendixDistTest}, as this test focuses on the chance to hit, instead
of how far the turret can shoot.

\section{Test Setup}
In essence this test setup is the same as in \autoref{AppendixDistTest} where
the turret fires along a wall, but in this test the area in which the
projectiles land is covered in flour, which is used to indicate exactly
where the projectile landed.

\subsection{Test Execution}
At the beginning of a test, the turret is set to angle itself to the designated
angle. The button is then manually pressed on order to set the turret to fire a
single projectile. The location, in which the projectile hit, is clearly
indicated in the flour. This is repeated 5 times for each angle, and for 6
different angles. The whole test has been done twice, such that the resulting
data is split into two different sets.

\section{Resulting Data}
The resulting data consists of a number of tests. Each test results in an angle
coupled with all the respective distances. Each test is also complemented with
width between the middle of the turret, and the point the projectile hit. This
is done in order to show the difference in width and length between hit
locations.

\subsection{Set 1}
The data from the first data set can be seen below in \autoref{table:FTest1}. 
\dataTable{FlourTest1}{Angle/Distance relationship coupled with spread.}{FTest1} 

\subsection{Set 2}
The data from the second data set can be seen below in \autoref{table:FTest2}. 
\dataTable{FlourTest2}{Angle/Distance relationship coupled with
spread.}{FTest2}

\section{Data Interpretation: Width Spread}
From the data it can be seen that the distance the projectile has to travel
decreases the chances of accurately hitting the target. The average spread for
each set is shown below in \autoref{table:AVGSpread}.

\dataTable{AVGSpread}{Average spread recorded in the two data sets.}{AVGSpread}

While it is not a linear increase in spread, the locations further away had
considerably more spread that those which were closer. 

\section{Data Interpretation: Length Spread}
Another problem with the accuracy of the turret lies in the fact that there is a
variety in the distance the projectile will travel at a given angle. This is shown below in
\autoref{table:AngleVsDistSpread}, where the average spread in distance
increases the farther the projectile is fired. In addition it has also been noted that
the turret has a chance to misfire. When this happens, the projectile will not
follow its usual arch, and will travel a considerately shorter distance. This
misfire chance was found to be 3.3\% in the first set, and 23.3\% in the second
set.

\dataTable{AngleVsDistSpread}{Average spread recorded in the two data
sets.}{AngleVsDistSpread}

The data shown in \autoref{table:AngleVsDistSpread} was calculated by looking at
all of the recorded distances. 

\section{Conclusion}
By looking at this test we can conclude the following:
\begin{itemize}
  \item The turret has an average chance to misfire of 13.3\%.
  \item In the recorded lengths, the hit location has a spread of between 9cm
  and 107cm.
  \item In the recorded widths, the hit location has a spread of between 0.25cm
  and 8.55cm.
\end{itemize}

This test has shown a large possible difference between points for the
same angle. While this does pose a problem for the turret, we have been unable
to identify which factors causes this behaviour. As there is no clear
correlation between the angle of fire and the spread, we assume that the problem
lies with the fireing mechanic itself. 
