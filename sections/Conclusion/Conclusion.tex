\chapter{Conclusion}\label{FinalConclusion}
This chapter will be used to conclude on the success of the project, based on
the requirements in \autoref{FeatAndReq} and the ideal scenario in \autoref{ProblemDomain},
and the test of the turret's accuracy presented in \autoref{TurretAcc}. \nl

Throughout this project the group has been designing and developing an embedded
system in the form of an autonomous turret, which is capable of hitting a target
in motion. The physical design of the turret was chosen to be made using LEGO,
and likewise the hardware and software for the turret has been based on the LEGO
Mindstorms NXT.\\
To make an initial conclusion on the success of the project, there are some
parts of the requirements which have not been met while others have.
As we have been unable to satisfy all needed requirements, we deem that \name
is not a satisfactory product, but given better sensors and more time for
optimizing the program, it would be able to satisfy all the set requirements. 
The requirements, and their level of completion, can be
seen below in \autoref{RequirementTable}.

\begin{table}[H]
\centering
\begin{tabular}{|p{\textwidth/2}|p{\textwidth/2}|}
\hline
Feature/Requirement & Evaluation\\ \hline
 The system needs to use the NXT as the embedded system. & This
 requirement was met.\\
 \hline 
 The system must be an intelligent agent. & This requirement was not met
 since there is no belief network implemented.\\ \hline
 Since the turret should monitor an area and be able to hit a moving target,
it needs to be able to turn. If it is not able to turn, the line of sight as well
as the timing in order to hit a target will be severely limited. In regards
to how far it should be able to turn, a minimum is set to 90 degrees, as
this will provide a decently sized area in front of the turret.
& This requirement was met, since the turret is able to turn 360 degrees. \\
\hline 
In order to perceive the target, the turret will need to make use of sensors.
& This requirement was met since the turret uses sensors to gather data. \\
\hline The turret needs to be able to shoot in order to hit a target. This has to
be done within a range of 200cm as it is the max capability of the sensors
see \autoref{SensorTest}.
& This requirements was met as the turret is able to shoot approximately 250cm.
\\ \hline
The target needs to move fast enough to require a firing offset, the
target also needs to move on an even plane and with a constant speed.
These two additional requirements are needed to ensure that there are as
few factors as possible that are not accounted for in the scenario.
& This requirement was met. \\ \hline
The turret needs to be able to calculate the firing offset to ensure a high
accuracy. It is not enough to hit the target sometimes, the turret should
be able to hit it reliably. It should be able to hit a moving target at least
80\% of the time.
& This requirement was not met since the turret only has a hit
accuracy of 12.5\%, see \autoref{TurretAcc}. However, the hit accuracy was
higher when the target was 50cm away and moving at a speed of 40\%.\\ \hline
\end{tabular}
\caption{Requirements}
\label{RequirementTable}
\end{table}
\section{Final Conclusion}
Based on the partial completion of the requirements, we conclude that, while
\name is not a satisfactory product, it is still partially successful. We
have been able to build and implement an autonomous turret, which is capable of
hitting a moving target, albeit with a very low chance to hit.\nl

In addition, while we have been unsuccessful in implementing a belief network,
we conclude, that enough of the network has been modeled, in order to allow us
to be able to implement it, given more time for further development.\nl

Furthermore, while the turret is designed to scan in 360 degrees and handle
targets in all possible directions, the turret is limited to waiting for a
target to appear, instead of scanning an area and searching for it. While this
is not satisfactory, this is a limitation imposed by the quality of the sensors.
Thus given better sensors and time for further development, the turret could be
able to realize the ideal operational scenario. \nl

