\chapter{Conclusion}\label{FinalConclusion}
This chapter will be used to conclude on the success of the project, based on
the requirements in \autoref{FeatAndReq} and the ideal scenario in \autoref{ProblemDomain},
and the test of the turret's accuracy presented in \autoref{TurretAcc}. \nl

Throughout this project the group has been designing and developing an embedded
system in the form of an automatic turret, which is capable of hitting a target
in motion. The physical design of the turret was chosen to be made with LEGO,
and likewise the hardware and software for the turret has been based on the LEGO
NXT platform.\\
To make an initial conclusion on the succes of the project, there are some
points from the requirements which have not been met while other points are.
As we have been unable to satisfy all needed requirements, we deem that \name
is not a satisfactory product, but given better sensors and more time for
calibration, it would be able to satisfy all the set requirements. 
The requirements, and their level of completion, can be
seen below in \autoref{RequirementTable}.

\begin{table}[H]
\centering
\begin{tabular}{|p{\textwidth/2}|p{\textwidth/2}|}
\hline
Requirements & Fulfilled\\ \hline
 The system needs to use the Lego NXT as the embedded system. & This
 requirement was met.\\
 \hline 
 The system must be an intelligent agent. & This requirement was not met
 since there is no Machine Intelligence implemented.\\ \hline
 Since the turret should monitor an area and be able to hit a moving target,
it needs to be able to turn. If it is not able to turn, line of sight as well
as the timing in order to hit a target will be severely limited. In regards
to how far it should be able to turn, a minimum is set to 90 degrees, as
this will provide a decent sized area in front of the turret. & This requirement
was met, since the turret is able to turn 360 degrees. \\ \hline
In order to perceive the target without receiving information from an
outside source, the turret will need to be able to sense the target. Based
on the results from the sensor tests, see \autoref{SensorTest}, the sensors will
be able to see approximately 200 cm. & This requirement was only partially met,
this is due to the fact that the quality of the measurements was of varying
quality. \\ \hline
 The turret needs to be able to shoot in order to hit a target. This has to
be done within a range of 200 cm as it is the max capability of the sensors. &
This requirements was met as the turret is able to shoot approcimately 2.5 m. \\
\hline
 In addition to the target moving fast enough to require a firing offset, the
target also needs to move on an even plane and with a constant speed. & This
requirement was met. \\ \hline
The turret needs to be able to calculate the firing offset to ensure a high
accuracy. It is not enough to hit the target sometimes, the turret should
be able to hit it reliably. It should be able to hit a moving target at least
80\% of the time. & This requirement was not met since the turret only has a hit
accuracy 12.5\%, see \autoref{TurretAcc}. However, the hit accuracy was higher
when the target was 50 cm away. This was due to the fact that the offset was not
properly calculated or the data from the sensor was bad. \\ \hline
\end{tabular}
\caption{Requirements}
\label{RequirementTable}
\end{table}
\section{Final Conclusion}
Based on the partial completion of the requirements, we conclude that, while
\name is not a satisfactory product, it is still partially succesful, as we have
been able to build and implement an autonomous turret, which is capable of
hitting a moving target, albeit with a very low chance to hit.\nl

In addition, while we have been unsuccesful in implementing a machine
intelligence model, we conclude, that enough of the groundwork has been done, in
order to allow us to be able to implement it, given more time for future
development.\nl

Furthermore, in the current implementation, the turret is limited to waiting for
a target to appear, instead of scanning an area and searching for it. While this
is not satisfactory, this is a limitation imposed by the quality of the sensors,
and as such, given better sensors and more time for future development, the
turret could be made able to realize the ideal operational scenario.
