\chapter{Discussion}\label{projDiscus}
This chapter will be used to examine the project as a whole, and try to identify
how the project could be improved. This will be done by reflecting upon the
unsatisfactory parts of the developed product and the project process, and
suggesting improvements for potential future development.

\section{Sensor Quality}\label{DSQ}
A major problem was the quality of the sensor input\ldots

\section{Quantity of Data}
As discussed above in \autoref{DSQ}, the quality of the received data was not
satisfactory\ldots

\section{Implementation of Machine Intelligence}
Due to problems with the management of the project, the machine intelligence has
not been implemented.

\section{Communication Using Bluetooth}
As a part of the machine intelligence, we would have liked to implement
communication via bluetooth such that calculations could be made on a PC instead
of the NXT. But we were able not 

\section{Should have refactored the code}
As a group we decided to use an agile software engineering paradigm. This means that
refactoring should be a recurring event in our development strategy. Ideally, after each iteration, the code should be refactored. Although we did refactor our code from time to time, there was no schedule to it at all, resulting in a lack of refactoring. What we should have done is dedicate a team to refactor the code after each iteration, ensuring that the refactoring occurs regularly. This would also be more in line with the Unified Process way of thinking, making the refactoring part of the development process, instead of just doing it when feel it is necessary.

\section{Put more thought into picking an OS/Firmware, language to write in}

\section{Scheduling}
As we have no control over the scheduling of the different threads on the NXT\ldots

\section{Less Multithreading}
During evaluation of the

(HINT: Track og GetDataAndCalculate bør ikke køre samtidigt. Den ene starter
  den anden, og holder den kun åben i et tidsinterval.)

\section{Real camera and custom software}
(HINT: Rigtigt kamera med vores egen software.)
