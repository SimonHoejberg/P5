\chapter{Discussion}\label{projDiscus}
This chapter will be used to examine the project as a whole, and try to identify
how the project could be improved. This will be done by reflecting upon the
unsatisfactory parts of the developed product and the project process, and
suggesting improvements for potential future development.

\section{Sensor Quality}\label{DSQ}
One of the major problems encountered throughout the project, was the
general quality of the sensors. From the start of the project, we have been
limited in what the sensors could output, and as the project went along, the
sensor quality started to degrade. Originally the ultrasonic distance sensors
could measure up to 250cm and do this in a cone in front of it. At the end
of the project the sensor would struggle to observe an object 150cm away, and
any cone of vision had disappeared. In much the same way, at the start of the
project the camera could easily identify a coloured object at a distance of just
over 2m, while at the end of the project, it would still be able to identify the
object, but not as clearly as before.\nl

The distance sensor's user guide states that it works best on large objects with
hard surfaces, which is one of the reasons why our target is a large red metal
can. Considering that the target should be ideal, that the sensor's
abilities to identify it has degraded over time and that the program used
is a simple piece of software designed to test the sensor, we are drawn to the
conclusion, that the quality of the sensors are too low, and that they are
considerably worse now, that they were at the start of the project.\nl

While the camera has not decreased much in quality over the span of the project,
the quality of the sensor as a whole has not been satisfactory. While the sensor
is easy to use, the quality of the output is lower that what we originally
expected. The first problem lies in the fact that the resolution of the camera
is too low, which results in problems when it comes to accurately identifying
objects, and telling objects apart from the background. In addition, we have
found that the camera needs a bright light source in order to be able to
identify the target at a distance of 2m. This has turned out to be a big
problem, as it has limited the turret severely and has reduced our ability to
accurately test the turret, as many tests have been unsuccesful due to the
camera not being able to identify the target.\nl

For another project, it could be possible to use a cheap camera sensor with a
higher resolution, and do the colour recognition ourselves. While the NXTCam has
a dedicated microcontroller for this process, it could be possible to simplify
the colour recognition enough to be able to implement it on the NXT. An example
of this could be that the NXTCam has the ability to observe 8 different blobs of
colour at the same time. If this was reduced to one, which we believe would
still be acceptable for this project, it could be possible to reduce the
processing power required, an as such be able to implement it on the NXT.\nl

While the ultrasonic distance sensors have not been satisfactory, we have been
unable to identify any alternative sensors which would be both easy to integrate
into the NXT and which are capable of identifying targets at the desired range
of up to 2m.

\section{Quantity of Data}
As discussed above in \autoref{DSQ}, the quality of the received data was not
satisfactory, which has led to problems with gathering the required information
in order to calculate the targets future position. The current implementation of
the \name software requires the turret to observe the target at at least 2
different points, but preferrably at a much higher amount.
The fact that we have only two points to calculate the future position from,
creates a lot of problems when it comes to drawing an accurate conclusion. This
is because two points results in only a single speed and directional vector, on
which no further analysis can be done in order to determine the accuracy of the
data.\nl

Given a larger amount of data, it would be possible to determine which data to
use, and to analyze this data in order to get an average direction and speed.
This way, any misreadings from the sensors would even out, and a more accurate
conclusion could be drawn.\nl

From our observations, the reason for the low amount of data mostly comes down
to the poor quality of the sensors, but it is possible that the current
implementation could be optimized to work under these conditions. In the current
implementation, the ultrasonic sensors are the limiting factor when it comes to
gathering data, as a data point can only be made when the sensor observes the
target.
A problem with this implementation could be that the distance sensors are only
activated once the camera returns information about the target being in its
vision. Considering that we have no control over the scheduling of the NXT, it
could be possible that this dependency between the two sensors causes the
distance sensor to somtimes miss its window of opportunity to observe the
target.
In an alternate implementation, we could try a solution where the distance
sensors are always on, such that the commands to turn them on/off would not
interfere with its ability to locate the target in time. In this implementation,
we could simply sort the data such that only the useful data is saved, while
circumventing the possible problems of scheduling the on/off commands to the
sensors.

\section{Implementation of Machine Intelligence}
Due to problems with the management of the project, the machine intelligence has
not been implemented. Several problems were identified during our construction
of the intelligent agent. First and foremost our scheduling of the design
activity was inadequate. Which means that at first we didn't really plan the MI
resulting in us starting the design of the intelligent agent way too late. This
complicated other things in the project for example the serial communication,
which resulted in us not having time to get the USB connection to work. Another
consequence of our late planning, was that the MI-agent was built around our NXT
code, and not in conjunction with. This resulted in some oddities and
simplifications regarding our design of the network.


\section{MI-Model Design}
One of the inconsistencies of our belief network is the probability tables that
we are operating on, which are based on a limited amount of learning data. That
means that we have certain states where the probabilities are quite uncertain. A
simple solution for this would have been more training data, as that would
increase the precision of our probabilities. Another factor that could have
increased our precision is modeling more possible states on our features,
although the hardware did restrict us.

\section{Serial Communication to the NXT}
When trying to implement the machine intelligence, we found that an complex
machine intelligence model would require more processing power and memory than
what was available on the NXT. As such, as a part of the machine intelligence,
we would have liked to implement communication via either bluetooth or USB, such
that the NXT could gather data, which it then sent to a PC, which would compute
all complex calculations, and then simply send the results to the NXT.

\subsection{Bluetooth}
In an attempt to implement serial communication to the NXT, we tried to
establish a bluetooth connection between the NXT and a PC. This implementation
never succeded as none of the bluetooth enabled PCs in the group were able to
establish a conncetion to the NXT.\nl
One of the likely reasons we never succeded to implement bluetooth
communications are beacuse we did not devote enough time to research and test
methods for doing so. In an alternate project, it would be a good idea to focus
on this functionality early on, as is alleviates some of the problems we
encountered during the project. One such problem is, that all transfer of files
between the NXT and the PC was done using a USB cable. This posed a problem, as
the turret rotates during its operation, which means that the cable can not
always be connected. This resulted in a lot of annoyances, as we needed a
connection to read the NXT screen, upload new software and download data logs,
which could all be done using a bluetooth connection.

\subsection{USB}
While we were able to use a USB connection to share files between the NXT and a
PC, we were never able to share a continious stream of information between the
two. We tried to implement a communication system by using the NXT's mailbox
system, which is basically a set of variables or strings, which can be read
through a serial connection. Our attempt was done using MonoBrick
\cite{MonoBrick}, which is a library for C\# which implements methods for
reading and writing to the NXT's mailbox. Our attempt to implement a serial
connection consisted of having 3 group members try to use their individual
understanding and implementation of MonoBrick, to establish a connection to the
NXT mailbox. While we succeded in writing data from a PC to the NXT, we were
unable to read data from the NXT.

\section{Code Refactoring}
As a group we decided to use an agile software engineering paradigm. This means
that refactoring should be a recurring event in our development strategy.
Ideally, after each iteration, the code should be refactored. Although we did
refactor our code from time to time, there was no schedule to it at all,
resulting in a lack of refactoring. What we should have done is dedicate a team
to refactor the code after each iteration, ensuring that the refactoring occurs
regularly. This would also be more in line with the Unified Process way of
thinking, making the refactoring part of the development process, instead of
just doing it when feel it is necessary.

\section{Programming language}
For programming the NXT we went with the C-like language NXC, if however we were
to make a similar project, it would be beneficial to choose another language
along with better firmware. This would be the case since the NXC language is not
that well documented, and we did not observe any larger community behind
the language. This meant that finding solutions to weird issues was not that
easy. Therefore, choosing another language, as for example LeJos \cite{LeJos},
which has better documentation and better functionality. \nl

LeJos also has the advantage of being object oriented, which is highly
advantageous when several persons are working on the same program. Furthermore,
the ability to create objects gives the option to divide the responsibilities of
different methods.\nl

Another issue with using NXC, was we did not have access to a debugger. This
would not be a problem when using LeJos, which can be run in the JVM on the
computer. The only problem with doing it like this, is the ability to get sensor
input and send outputs to the motors. Thus, this part would still have been run
on the NXT itself, which is not that big of an issue.\\
To conclude, using a different programming language, along with different
firmware, could have helped us in developing the program, along with having
better opportunities for involving all the group members in the process.


\subsection{Scheduling}
Another aspect of choosing which language to write in and the associated
firmware is scheduling. Since the firmware from NXC is not very well documented
we do not know how the scheduler works, thus we do not have any control over how
the task are scheduled. If we had chosen another language we would have had more
options in terms of scheduling. For example if we had chosen LeJos it would have
been possible to create periodic tasks, this has the advantage of giving more
control of the flow in the program. The LeJos also has support for choosing when
threads are started, this in terms gives us the option to have a lot more
control over the scheduler \cite{lejosTimer}.


\section{Less Multithreading}

During evaluation of the final software implementation, we noticed a potential
problem with the way our two main tasks \textc{Track} and
\textc{GetDataAndCalculate} are interleaved. In the current implementation, both
of these tasks are set to always be running concurrently, and as such, they
both have to be scheduled, even when \textc{GetDataAndCalculate} is simply doing
busy wait, while it is waiting for a signal from \textc{Track}.\nl

This problem could have been solved by implementing the use of mutexes in order
to set one task to sleep while it waits for the other to finish. An alternate
solution whould be to have \textc{Track} running for the entirety of the program
execution, and then when it observed the target, it would start another task
called \textc{GetData}, which whould only collect one set of data. This would
allow the gathering of data to complete without stopping the tracking. When it
has enough data it will start the \textc{GetDataAndCalculate} task, such that
the turret will calculate the future position and shoot at the target.\nl

While we are uncertain that this is a real problem, we can conclude that having
a task busy-wait will take up some amount of processing resources. As this is
the case, if the \name software were to be developed further, this would
definitely be changed.
