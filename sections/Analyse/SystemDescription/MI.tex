\section{Machine Intelligence}
 
-------What is an agent (S1.1 p4) --------

An agent is something that acts in an environment.
An agent acts intelligent when:
- what it does is appropriate for its circumstances and its goals
- it is flexible to changing environments and changing goals
- it learns from experience
- it makes appropriate choices given its perceptual and computational
limitation. It typically cannot observe the state of the world and has finite
memory and limited time.

-------something about states---------
??? - maybe or maybe not

--------Knowledge base (s1.3)---------
AI is about reasoning
An agent is a coupling of perception, reasoning and acting.

Its actions depend on:
- prior knowledge about the agent and the environment
- history of interaction with the environment, which is composed of:
+ observation of the current environents
+ past experiences of previous actions and observations or other data, from
which it can learn
- goals that it must try to achieve or preferences over states of the world
- Abilites which are the primitive actions it is capable of carrying out
(consider using fig1.3 p11)

Problems do you usually not have clear solutions
To solve a problem a designer must:
- flesh out the task and determine what constitutes a solution
- represent the problem in a language with which a computer can reason
- use the computer to compute an output, which is an answer presented to a user
or a sequence
(Consider using fig1.4 p12)

A knowledge base is the representation of all of the knowledge stored on the
agent. A representation should be:
- rich enough to express the knowledge needed to solve the problem
- as close to the problem as possible
- amenable to efficient computation
- able to be acquired from people, data and past experience

Questions needed to be considered when given a problem:
- what is a solution, and how good must a solution be?
- how can a problem be represented?
- how can the agent compute an output that can be interpreted as a solution to
the problem? what about worst and average cases?

Much work in AI is motivated by commonsense reasoning

there are 4 common classes of solutions, they are not exclusive:
- optimal solutions
- satisficing solution --> satisfying + sufficient
- approximate optimal solutions
- probable solution

 -------- Dimensions of complexity ---------
Modularity:
flat - modular - hierachical

Representation scheme:
States - features - relations -propositition

Planning Horizon:
non-planning - finite horizon - indefinite horizon - infinite horizon

Uncertainty:
- Sensing uncertainty:
+ Fully observable
+ Partially observable

- Effect uncertainty:
+ Deterministic
+ Stochastic

Preference:
- goals are either achivement goals or maintenance goals
- complex preferences, ordinal preference, cardinal preference

Number of agents:
- single agent
- multiple agents

Learning:
- Knowledge is given
- knowledge is learned

Computational limits:
- Perfect rationality
- bounded rationality
