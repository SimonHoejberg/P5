\section{Intelligent Agents}
 
As \name is an autonomous turret and needs to be able to hit a moving target.
It needs to have some intelligence in order to determine how to position itself
to hit the target. The turret can be considered intelligent when its decisions
are explainable by computations. How intelligent it is, can as
\citep[ch.1.1,p.4]{MIBook} states, be judged by:

\begin{enumerate}
  
  \item \say{What it does is appropriate to its circumstances and goals.}
  \item \say{It is flexible to changing environments and changing goals.}
  \item \say{It learns from experience.}
  \item \say{It makes appropriate choices given its perceptual and
  computational limits.}
\end{enumerate}
%See MI book, chapter 1.1 p.4

In order to be intelligent \name needs to be able to represent its
knowledge in a suitable way that allows it to work it into a suitable solution. 

\subsection{Knowledge Representation}\label{KR}
%See MI book chapter 1.4 p.12

\name needs to be able to represent the problem of hitting a
moving target as a suitable representation. This representation can then be
computed into a suitable output which in turn can be interpreted as
a solution, whether or not the target is actually hit. This can be seen in
\autoref{MIRep}.

\figxs[0.8]{MIRep}{An illustration of how a problem is converted into a
solution}{\citep[ch. 1.4 p.12]{MIBook}}

The representation is a piece of knowledge from the representation
scheme. The representation scheme is used to make sure the individual
representations are detailed and close to the problem, such that the necessary
knowledge can be used to solve said problem \fix{\citep[ch.1.4.1]{MIBook}}{Not
the right ch?}.
Examples of knowledge that \name has is: the targets speed and how much the turret has
turned. When the turret has a sufficient amount of knowledge, it should be used
in a way that makes it possible to compute an output. Whether this output is a
solution or not, depends on what a solution is. There are 4 different categories
of solutions:

\begin{itemize}
  \item \textbf{Optimal solution} - A solution which has been deemed optimal
  through some measurement of quality.
  \item \textbf{Satisficing solution} - A solution that is both deemed
  satisfying by the end user and sufficient in terms of overall requirements. 
  \item \textbf{Approximately optimal solution} - A solution that is close in
  quality to the optimal solution. While not being perfect, this solution is
  deemed to be sufficient.
  \item \textbf{Probable solution} - A solution that is likely to be a solution
  but is not always guaranteed to be.
\end{itemize}

As \name is shooting on a moving target, the solution needs to be able
to  reliably hit the target. \name is also a real-time system, as such there are
certain limits to how much time can be spent searching and waiting for the
optimal shot.

\subsection{Design Space}
The design space of an intelligent agent can be considered a series of 8
factors, which are used to give an estimation of the system \citep[ch.
1.5]{MIBook}\KT. These factors are: modularity, representation scheme, planning
horizon, uncertainty, preference, number of agents, learning and computational
limits.

\subsubsection{Modularity}
Modularity describes how easily the system can be divided into modules, as well
as how easy it is to understand these modules separately. Modularity has three
different levels, flat, modular and hierachical. Flat means that everything is
contained within a single module. Modular means that the system can be divided
into several modules that can be understood separately from the system.
Hierachical means that the modules can recursively have additional modules
inside them \citep[ch1.5.1]{MIBook}. \name as an autonomous turret needs to
have several functions, for example it needs to be able to turn, aim and determine the position of a target.
This means that \name should be a modular system because, while it is a
relatively simple system, it has distinct parts which can each be considred
modules of their own.

\subsubsection{Representation Scheme}
As mentioned in \autoref{KR} the representation scheme is used to house the
various knowledge the agent could need to perform. This knowledge is usually
represented in terms of states. A state is a specific combination of
the knowledge \citep[ch1.5.2]{MIBook}. For example a turret with the ability
to turn 180 degrees and tilt its cannon 45 degrees would have $180 * 45$ states. As an example, a
specific state would have the turret turned 90 degrees and the cannon tilted at an angle of 42
degrees. This leads to alot of different states as the amount of possible values
increases. Instead of reasoning using states it is possible to combine the
relevant values into features. This means that the state is determined by the
value of the features, such as a feature, Tilt, could cover how tilted the
cannon is and have the values covering 0 to 45.

\subsubsection{Planning Horizon}
The planning horizon for an intelligent agent governs how far ahead it plans
when it comes to the consequences of its actions. For example a child might
only do something if there is an immediate reward, whereas an adult might have
the foresight to do something for a reward thats some time off. There exist the
following types of planning horizons \citep[ch1.5.3]{MIBook}:
\begin{itemize}
	\item \textbf{Non-planning} - The agent does not consider the future when
	making decisions, or time is not involved as a factor.
	\item \textbf{Finite horizon} - The agent considers the future for a fixed
	finite set of steps. 
	\item \textbf{Indefinite horizon} - The agent looks ahead for finite but not
	fixed number of steps.
	\item \textbf{Infinite horizon} - The agent plans on running forever.	
\end{itemize}

Considering \name is an autonomous turret, non-planning can already be ruled
out as that would that would result in a naive approach for the turret where it
would fire when it sees a target. Given the limited time a target is in range,
the turret can be considered having a finite horizon as the number of steps it
needs are limited.

\subsubsection{Uncertainty}

There are two parts to uncertainty. The first part is sensing uncertainty while
the second is effect uncertainty. Sensing uncertainty concerns what the agent
can determine based on its observations. There are two possibilities concerning
what it can sense \citep[ch.1.5.4]{MIBook}:
\begin{itemize}
  \item \textbf{Fully observable} - The agent can determine the current state
  based on observations.
  \item \textbf{Partially observable} - The agent can not directly observe the
  current state. This can be caused by noisy observations or when multiple
  states result in the same observations.
\end{itemize}

In order have knowledge of the target, \name will need to observe since the
properties of the target are not known by \name beforehand. This results in
uncertainty as we not only need to predict the future position of the target,
the sensors may provide noisy observations. We can thus already determine that
\name will only be able to partially observe the state.\nl

Effect uncertainty concerns how accurately the agent can determine the resulting
state based on an action in its current state. There are two possibilities
\citep[ch.1.5.4]{MIBook}:
\begin{itemize}
  \item \textbf{Deterministic} - The resulting state can be determined based on
  an action and the current state.
  \item \textbf{Stochastic} - The resulting state is the result of a probability
  distribution.
\end{itemize}

\name can not directly observe the current state, it is only able
to observe some of it through its sensors, as such the system is stochastic.
Although the system is stochastic, it can be modelled as a deterministic system, where
the actions can be be detemined by what it can observe.
 
\subsubsection{Preferences}
For an agent to be able to make good decisions it is necessary to provide an
idea of the goals we are trying to accomplish. Furthermore there are certain
preferences concerning the trade-off between different solutions
\citep[ch.1.5.5]{MIBook}.

\begin{itemize}
  \item \textbf{Goals} - There are two types of goals, achievement goals and
  maintenance goals. Maintenance goals must be achieved in all the states it
  visits. Achievement goals should hold true for the final state. 
  \item \textbf{Complex preferences} - There are two types of preferences,
  ordinal preferences and cardinal preferences. Ordinal preferences describe
  preferences where only the order of the preferences indicate importance.
  Cardinal preferences are those that are prioritized according to their value,
  such that more value is better.
\end{itemize}

For this project, first and foremost an achievement goal is needed in order to
describe the goal for the agent. This is simply the combined state of features
that allows us to hit a target in motion. 

\subsubsection{Number of Agents}
Depending on what an agent needs to accomplish, it might need to take the
reasoning of other agents into account. This leads to two types of reasoning
\citep[ch.1.5.6]{MIBook}:
\begin{itemize}
  \item \textbf{Single agent reasoning} - The agent does not take other
  agents into account and assumes they are part of the environment.
  \item \textbf{Multiple agent reasoning} - The agent takes the reasoning of
  other agents into account.
\end{itemize}

In regards to \namep, it needs to take the reasoning of the target into
consideration in order to actually hit it.

% Reasoning with multiple agents can be hard. This is due to the fact that an
% agent must take into account the other agents goals and preferences and reason
% strategically about these. It can also be advantageous to ignore the other
% agents if their behavior is not impacted by what the agent does.
% \namep uses only a single agent due to the fact that their only is a single
% agent so it does not need to take any other agents into account.

\subsubsection{Learning}

There are two ways of handling knowledge: knowledge can given in form of a
model or the agent can be given data to learn from \citep[ch.1.5.7]{MIBook}. 
\begin{itemize}
  \item \textbf{Knowledge is given} - The agent possesses a model which the
designer thinks is suitable for the task.
  \item \textbf{Knowledge is learned} - The agent can use data or past
  experiences to find a model that fits the data.
\end{itemize}

\name should make use of a prebuilt model that can accurately help it decide
what to do and calculate the various parts needed for projectile motion, see
\autoref{ProjMotion}. It is limited how useful learning is in this project, as
it is limited how much time and data we can provide for it to train with.

\subsubsection{Computational Limits}
An agent may not always be able to compute the best action to a problem quickly
enough, either due to memory limitation or just the sheer size of the problem.
This means that the agent may need to trade off the time it needs to compute a
solution with how good the solution is. This limit determines whether the agent
has \citep[ch.1.5.8]{MIBook}:

\begin{itemize}
  \item \textbf{Perfect rationality} - The agent tries to find the best action
  without taking resources into account.
  \item \textbf{Bounded rationality} - The agent decides the best action given
  the computational limits.
\end{itemize}

\name has bounded rationality due to the fact that the NXT provides limited
resources, as well as the limited time the target is in range. This means that
in some situations, like deciding when to shoot, it will have to determine the
best course of action at the moment instead of spending a lot of time computing
the best solution. In addition to just the calculations, \name also needs to
move into a proper position, and the shot itself has a certain travel time. All
of this leads to a system which uses bounded rationality.

\subsection{Belief Network}

The agent is only able to partially observe the states, it will thus need to be
able to reason under uncertainty. A belief network is a way to do this, it is an acyclic
directed graph consisting of random variables \citep[ch.6.3]{MIBook}. These
variables and their domains are usually derived from the features in the representation scheme, although
some random variables can be created to act as intermidiate variables. The
variables that make up the belief network are connected via either a
\textbf{serial}, \textbf{convergent} or \textbf{divergent} connection. These
connections dictate how information is able flow through the network given
evidence on a node.
\begin{itemize}
  \item A \textbf{serial} connection is a set of nodes connected in a sequence.
  These connections allow information to flow as long as the middle node is not
  blocked by evidence. If evidence is present on the middle node, it is not
  possible to say anything new about first node given more evidence further down
  the sequence. An example of a serial connection can be seen in
  \autoref{Serial2}.
 \figx[0.8]{Serial2}{An example of a serial connection}
  \item A \textbf{convergent} connection consists of a child node with two or
  more parents. In this connection the absence of evidence on the child node, or
  one of its descendants, makes it impossible to transmit information between
  the parent nodes. If however evidence is present on the child node, or one of
  its descendants, it is possible to transmit evidence between the
  parent nodes. An example of a convergent connection can be seen in
 \autoref{Convergent2}.
 \figx[0.8]{Convergent2}{An example of convergent connection}
  \item A \textbf{divergent} connection is a parent node connected to two or
  more children. In this kind of connection it is only possible to transmit
  information between the children if the parent node has evidence. An
  example of a divergent connection can be seen in \autoref{Divergent2}.
 \figx[0.8]{Divergent2}{An example of a divergent connection}
\end{itemize}

The independence assumption \citep[p.240]{MIBook} says:

\begin{center}
\begin{minipage}{0.8\linewidth}
\textit{Each random variable is conditionally independent of its non-descendants
given its parents.}
\end{minipage}
\end{center}

This means that if the node has few parents, the number of variables it can
directly affected it are few. It also means that the fewer parents each node
has the fewer probabilities need to be specified.

\subsection{Hidden Markov Model}

A belief network can be made to model a dynamic system by using a \textbf{Markov
chain} to represent a feature at different points in time. By adding
observations to the Markov chain, it becomes a \textbf{hidden Markov
model (HMM)}. If more than one state maps to the same observation it
becomes partial and if the same state maps to different observations at
different times it becomes noisy \citep[ch.6.5.2]{MIBook}. A Markov chain and by
extension an HMM, can be extendeded indefinitely, meaning that a small number
of parameters can provide an infinite network\citep[ch.6.5.1]{MIBook}. For
example: A man observes an airplane taking off northbound therefore adding
evidence at $O_0$, then at $O_1$ he observes that the airplane has turned
eastward, at $O_2$ he observes that it has turned southward and at $O_3$ he
sees that it continues south. This allows him to guess the position of the
plane at $P_4$, see \autoref{HMM}.
 \figx[0.9]{HMM}{A belief network as a hidden Markov model}



% \subsection{Conclusion}
%  All of this has an influence on how the belief network for \name should be
%  structured, as the nodes in it is the velocity of the target, the position of
%  the target, the distance to it and the angle of fire. The information we get is
%  from the sensors which means that the evidene should be on the distance and
%  angle node. We want the evidence to flow in such a way that it is possible to
%  transmit evidence to the velocity node.\nl


% 
%  The reason for using a belief network is that it makes it possible to model the
%  world with the data that \name gets from it's sensors. This model can be
%  relatively compact Which is an advantage since the NXT does not have a lot of
%  processing power. Another reason for Using a belief network is that the \name
%  is only able to observe the world partially since there is some noise on the
%  sensors, which means that it needs to be able to reason under uncertainty.
%  A belief network is also useful since \name needs to be able to plan
%  ahead when taking actions, since it should be able to consider the movement of
%  the target before shooting.
