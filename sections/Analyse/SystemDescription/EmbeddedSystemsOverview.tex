\chapter{System Description}\label{SystemDescription}
As a part of this semester project the developed products must take the form of
embedded systems, which makes use of the techiques taught in the 'Real Time
Systems (RTS)' course, or the 'Machine Intelligence (MI)' course. In order to
implement these techniques we will need to gain a deeper understanding of these
concepts. As such, this chapter will be used to describe what an embedded system
is, how real-time systems incorporate deadlines and schedules and examine what
elements from machine intelligence could be used by an autonomous turret.

\section{Embedded Systems}
As explained by the embedded systems manufacturer BarrGroup
\citep{Barr_Embedded}, an embedded system is: \nl

\begin{center}
\begin{minipage}{0.8\linewidth}
\textit{A combination of computer hardware and software, and perhaps additional
mechanical or other parts, designed to perform a dedicated function. In some
cases, embedded systems are part of a larger system or product, as in the case
of an antilock braking system in a car.} \fix{}{How do we explict state that
this is not our text	}
\end{minipage}
\end{center}
% An embedded system is a device with limited resources both in terms of computing
% power and how much memory it has available. They are used everywhere to do small
% task and jobs. \fix{It is difficult to give an exact definition of what an
% embedded system is, but a crude definition\citep[ch.1.1]{vahid1999embedded}
% is}{We need a better definition}:\nl
% 
% \textit{An  embedded  system  is  nearly  any computing system other than a desktop,
% laptop, or  mainframe  computer.}\nl

This means that embedded systems cover alot of different types of computers, for example
the ABS-brakes in a modern car, the cash registers in retailstores, washing machines, etc.
The requirements can therefore differ alot between systems, although they do
share some characteristics \citep[ch.1.1]{vahid1999embedded}:

\begin{enumerate}
  \item \textbf{Single-functioned}:Most embedded systems only have a single purpose, running the same program over and over again. An example could be a washing machine. The computer inside it will always have the same function(control the washing machine), and although it might have a variety of different programs to execute(wool, synthetics, different temperatures) it is still considered single-functioned. On the other hand a desktop pc allows the user to install a variety of different programs, and does not have a universal single purpose.
  \item \textbf{Tightly constrained}: All systems have some limits, but
  embedded systems are much more tightly constrained when it comes to cost,
  size, performance and power. These are to be understood as a compromise between the different aspects, for example it could be impossible to make the embedded system smaller without making it more expensive. 
  \item \textbf{Reactive and real-time}: If an embedded system has to react and respond to external stimuli it is considered a real-time system. We will further expand upon what this means in \autoref{rts}.
\end{enumerate}
