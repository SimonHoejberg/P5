\chapter{System Description}
This chapter examines embedded systems and their limits,
real-time systems concerning how scheduling and deadlines work and machine
intelligence.
In this chapter we will provide a description of the different concepts within the
three system categories
This semesters requirements of implementing a embedded system that must use techniques
from either the RTS or MI-course dictates that we need an understanding of these concepts,
in order to fulfill them. So in this chapter an overview of the concepts of these categories
is provided and in conclusion, certain requirements can be penned.

\fix{}{Add more text and better description of what will be done}


\section{Embedded Systems}
An embedded system is a device with limited resources both in terms of computing
power and how much memory it has available. They are used everywhere to do small task and
jobs. It is difficult to give an exact definition of what an embedded system is,
but a crude definition\citep[ch.1.1]{vahid1999embedded} is:\nl

\textit{An  embedded  system  is  nearly  any computing system other than a desktop,
laptop, or  mainframe  computer.}\nl

This means that embedded systems cover alot of different types of computers, for example
the ABS-brakes in a modern car, the cash registers in retailstores, washing machines, etc.
The requirements can therefore differ alot between systems, although they do
share some characteristics \Source:

\begin{enumerate}
  \item \textbf{Single-functioned}:  Most embedded systems only have a single
  function which is executed repeatedly.
  \item \textbf{Tightly constrained}: All systems have some limits, but embedded
  systems due to their nature are much more tightly constrained when it comes to
  cost, size, performance and power. These constraints are linked in the sense
  that an increase in one area directly affects the other areas, see
  \autoref{ESConstraints}.
  \item \textbf{Reactive and realtime}: A lot of embedded systems also has to
  react and respond to certain input from the physical world, for example via sensors.
  This is what creates the concept of real-time systems, since the response may have a deadline
  based from the real world. Then in fact it is a real-time system, but embedded systems and
  RTS have a lot of overlap.
\end{enumerate}

\figxs{ESConstraints}{How the constraints depend on each
other}{\Source}

\fix{write about microcontroller?}{Depending on whether or not it is necessary
it may be dropped.}
