\chapter{System Description}\label{SystemDescription}
As a part of this semester's requirements, the developed product must take the
form of an embedded systems. In addition, the system must also make use of
techniques relating to the fields of real-time systems or machine intelligence.
As such, this project will be focused on using techniques relating to machine
intelligence, while cursory touching upon the subject of real-time systems. In
order to implement these techniques, we will need to gain a deeper understanding
of these concepts. As such, this chapter will be used to describe what an
embedded system is, which aspects from real-time systems are important for
embedded systems and which elements from machine intelligence can be
used by an autonomous turret.

\section{Embedded Systems}
As explained by the embedded systems manufacturer BarrGroup
\citep{Barr_Embedded}, an embedded system can be defined as: \nl

\begin{center}
\begin{minipage}{0.8\linewidth}
\say{A combination of computer hardware and software, and perhaps additional
mechanical or other parts, designed to perform a dedicated function. In some
cases, embedded systems are part of a larger system or product, as in the case
of an antilock braking system in a car.}
\end{minipage}
\end{center}
% An embedded system is a device with limited resources both in terms of computing
% power and how much memory it has available. They are used everywhere to do small
% task and jobs. \fix{It is difficult to give an exact definition of what an
% embedded system is, but a crude definition\citep[ch.1.1]{vahid1999embedded}
% is}{We need a better definition}:\nl
% 
% \textit{An  embedded  system  is  nearly  any computing system other than a desktop,
% laptop, or  mainframe  computer.}\nl

This means that embedded systems cover a lot of different types of systems, for
example the antilock braking system (ABS) in a modern car and washing machines.
The requirements can therefore differ between systems, although they do share
some characteristics \citep[ch.1.1]{vahid1999embedded}:

\begin{enumerate}
  \item \textbf{Single-functioned}: Most embedded systems only have a single
  purpose, running the same program over and over again. An example could be a
  washing machine. The computer inside it will always have the same
  function (control the washing machine), and although it might have a variety
  of different programs to execute (wool, synthetics, different temperatures),
  it is still considered single-functioned.
  \item \textbf{Tightly constrained}: All systems have some limits, but embedded
  systems are much more tightly constrained when it comes to cost, size,
  performance and power. These are to be understood as a compromise between the
  different aspects, for example it could be impossible to make the embedded
  system smaller without making it more expensive.
  \item \textbf{Reactive and real-time}: Many embedded systems have to react to
  some kind of external stimuli. If an embedded system has to take into account
  this external dynamic information, it is considered a real-time system. We
  will further elaborate upon what this means in \autoref{rts}.
\end{enumerate}
