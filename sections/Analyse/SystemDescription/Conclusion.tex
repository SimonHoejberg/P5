\section{Conclusion}\label{EmbConc}
Now that the basic concepts of each aspect of the system have been explained, a summary will now be concluded in
relation to the project.
\subsection{Embedded}
The system needs to be an embedded system, which requires taking into consideration the
contraints imposed by such a system. By using a hardware platform such as Arduino or Lego NXT
this would be sufficient, since the platforms restrict on the basis of computing power,
size and memory. Cost isn't a factor at this point since the university provides the platform,
and the project isn't meant to be mass-produced.
\subsubsection{Real-Time Systems}
Although this project group will focus on the Machine Intelligence, the project still has
some inherit RTS elements, such as deadlines and scheduling of task
\subsubsection{Machine Intelligence}

\subsection{Conclusion}
\fix{}{conclusion inside of conclusion}
Relative to an embedded system, the turret will be
\begin{itemize}
  \item Single-functioned - Our turret will be constructed with its only purpose of hitting a target in motion
  \item Tightly-constrained - When working on a hardware platform, the platform imposes constraints in the form of
  size, cost(production and design), power and memory. The system would also have to adhere to these constraints since it
  needs to be implemented on a hardware platform
  \item Reactive - The turret senses an area and if a certain input is received it will react on it and try to fire a projectile
  in order to hit a target. Hence the system is reactionary
\end{itemize}
% In terms of RTS