\section{Conclusion}\label{EmbConc}

Throughout this chapter the basic concepts of each aspect of the system have
been explained. As such, this section will be used to conclude on how these
different aspects can be used in relation to the \name system.

\begin{itemize}
  \item The constraints for this semester requires the developed product to be
  an embedded system, which implies that we need to take the hardware
  constraints into consideration and design it for this specific task. This
  requirement is fulfilled by using a hardware platform such as Arduino or Lego
  NXT, since the platforms are restricted on the basis of computing power, size
  and memory.
  \item Since it needs to be an intelligent agent, it needs to have a model
  which can be used to solve the problem. Since the agent can only partially
  observe the state it must be able to reason under uncertainty. To do this the
  agent can use a belief network to try and predict the future based on
  observations.
  \item \name can be considered a firm real-time system, as once certain
  deadlines have passed, for instance when firing, the information becomes
  obsolete and is no longer useful, as the target will have moved on. 
\end{itemize}  

With the system analysed, it is now possible to determine how we ensure that
the turret will actually hit the target.


% \subsubsection{Machine Intelligence}
% 
% \subsection{Conclusion}
% \fix{}{conclusion inside of conclusion}
% Relative to an embedded system, the turret will be
% \begin{itemize}
%   \item Single-functioned - Our turret will be constructed with its only purpose of hitting a target in motion
%   \item Tightly-constrained - When working on a hardware platform, the platform imposes constraints in the form of
%   size, cost(production and design), power and memory. The system would also have to adhere to these constraints since it
%   needs to be implemented on a hardware platform
%   \item Reactive - The turret senses an area and if a certain input is received it will react on it and try to fire a projectile
%   in order to hit a target. Hence the system is reactionary
% \end{itemize}
% % In terms of RTS
