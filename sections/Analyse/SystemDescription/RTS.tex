\section{Real-Time Systems}
Although the project group does not have any members following the Real-Time
Systems (RTS) course, the project still incorporates certain elements from RTS.
As such, this section will provide a quick overview of the elements which will
be used in relation to the project.
In order to have a proper foundation for the core concepts in the field of
real-time systems, we will give our own definition of what we percieve real-time
systems to be:

\begin{center}
\begin{minipage}{0.8\linewidth}
\textit{A real time system is any information processing system, which needs to
react within a finite time period to externally generated inputs in order meet
its deadlines.}
\end{minipage}
\end{center}

For this project we consider a real-time system to be categorized by looking at
four different elements: timeliness,, deadlines, tasks and scheduling. As such,
these will be further explained throughout this section.

\subsection{Timeliness}
% Timeliness is defined as being a function of the total system, for example a
% missile guidance system requires the input data within a few milliseconds. This
% means that for all real time systems the concept of timeliness is relative to
% the project - which makes sense since the only thing to care about is whether
% or not the deadlines are met \Source.

In the context of real-time systems in this project, timeliness should be
understood as an operation being able to complete it tasks ahead of the deadline. As such the
degree of timeliness depends on the deadlines in a given system, for example a
missile guidance system could require its input data within a few milliseconds.
This means that for any given real time systems the concept of timeliness is
relative to its functionality - which makes sense since we always care about
whether or not the deadlines are met.


\subsection{Deadlines}
% As mentioned in our RTS definition, real time systems are concerned about
% deadlines in a system. The real time system gets classified depending on the
% consequences of not meeting a deadline. If the result of a failed deadline is
% catastrophic, the system gets classified as a hard real time system - examples
% of such are control systems for airplanes, nuclear power plants, etc \Source.
% If an occasional deadline is missed, or the system doesn't shut down if a
% deadline is missed, it might be a soft real time system, for example
% video-conferencing software the result of a missed deadline might be dropped
% frames but nothing critical \Source.

As mentioned in our RTS definition, real time systems are concerned about
deadlines in a system. As such, we can classify these systems depending on the
consequences of not meeting a deadline. If the result of a failed deadline is
catastrophic, we choose to classify such a system as a 'hard' real time
system. These could for example include things such as control system for
airplanes or nuclear power plants where it is vitally important that the
system can react in a timely fashion. If an occasional deadline can be safely
missed, or the system doesn't shut down if a deadline is missed, we choose to
classify it as a 'soft' real time system. These systems could for example be 
video-conferencing software where the result of a missed deadline could be
dropped frames but nothing critical which would prevent the operation of the
system.


\subsection{Tasks}
For the purpose of this project, we consider a task to be a subset of a given
program which can be executed on a thread. For example a task in an ABS-brake
system could be to use a method named ReadVelocity() to determine the speed of
the vehicle. In order to construct a schedule for a given system, the time to
process the different tasks is needed. By having this overview of
all the different tasks and their needed execution time, it is possible to
accurately schedule a given program.

% We consider a task to be a subset of a given program. For example a task in an
% ABS-brake system could be ReadVelocity(). In order to construct a schedule first the time to
% process all the different tasks is needed. This provides a overview of all the
% different tasks needed to complete the function and their respective runtime
% \Source.

\subsection{Scheduling}
To ensure that the deadlines of the different tasks are \textbf{always} met,
scheduling is needed to organize their execution. By explicitly stating a
schedule for the system to follow, a scheduling analysis can be conducted in
order to ensure that the schedule does in fact always hold true and no deadlines
are missed \Source. Scheduling is one of the fundamental concepts of RTS, but
for scheduling to work, knowledge is needed about how long tasks take to finish
\Source.

% To ensure that the deadlines \textbf{always} are met, scheduling is needed.
% By explicitly stating a schedule for the system to follow, a scheduling analysis
% can be conducted in order to ensure that the schedule does in fact always hold
% true and no deadlines are missed \Source. Scheduling is one of the
% fundamental concepts of RTS. But for scheduling to work, knowledge is needed
% about how long tasks take to finish \Source. 


% \subsection{An example}
% An example of a real time system is the ABS-brakes on a car

% At its most basic level, real time systems are systems where the timing of
% specific parts are a vital factor. Consider the example of an ABS-brake - the
% system needs an input i.e. a sensor input from the wheels. This information is
% then used to calculate the correct amount of pressure to apply to the brakes in
% order to optimize the braking length of the vehicle, that is the output. The
% time it takes from receiving the input till it produces the output is critical.
% This is due to the fact that the brakes needs to be on instantly. So in fact
% when talking about real time systems, the only concern is whether or not the
% deadlines are met.