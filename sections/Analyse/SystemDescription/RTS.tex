\section{Real-Time Systems}
Although the project group doesn't have any members following the RTS course,
the project still incorporates certain elements from RTS. In this section a
quick overview of the elements will be described in order to provide a basic
understanding in relation to the project.

\subsection{Definition}
In order to give a proper foundation of the core concepts in the field of RTS, a
proper definition will be given.
The Oxford Dictionary of Computing \citep{daintith2008dictionary} defines a
real-time system as follows \say{Any system in which the time at which output is
produced is significant. This is usually because the input corresponds to some
movement in the physical world, and the output has to relate to that same
movement. The lag from input time to output time must be sufficiently small for
acceptable timeliness. Timeliness is a function of the total system: missile
guidance requires output within a few milliseconds of input whereas scheduling
of steamships requires responses measured in days. Real-time systems are usually
considered to be those in which the response time is of order milliseconds;
interactive systems are those with response times of order seconds and batch
systems are those with response times of hours or days. Real-time systems may be
subdivided into hard and soft, depending on the severity of failure to meet a
deadline for output.}

\subsection{Introduction}
From the definition it is given that \say{Any system in which the time at
which output is produced is significant}. At its most basic, RTS systems are
just systems where the timing of specific parts are a vital factor. Consider
the example of an ABS-brake again - the system needs an input i.e. a sensor
input from the wheels. This information is then used to calculate the correct
amount of pressure to apply to the brakes in order to optimize  the braking
length of the vehicle, that is the output. The time it takes from receiving 
the input till it produces the output is critical. This is due to the fact that
the brakes needs to be on instantly. So in fact when talking about
realtime-systems, the only concern is whether or not the deadlines are met.

\subsection{Timeliness}
The definition mentions the concept of timeliness as the following:
\say{Timeliness is a function of the total system: missile guidance requires
output within a few milliseconds of input whereas scheduling of steamships requires responses
measured in days}. This means that for all realtime-systems the concept of
timeliness is relative to the project - which makes sense since the only thing
to care about is whether or not the deadlines are met.

\subsection{Deadlines}
As mentioned in the given definition, realtime systems are concerned about
deadlines in a system.
The realtime system gets classified depending on the consequences of not
meeting a deadline. If the result of a failed deadline is catastrophic, the
system gets classified as a hard real-time system - examples of such are control
systems for airplanes, nuclear power plants, etc.
If an occasional deadline is missed, or the system doesn't shut down if a
deadline is missed, it might be a soft realtime system, for example
video-conferencing software the result of a missed deadline might be droped
frames but nothing critical.

\subsection{Tasks}
A task is a subset of the program. For example a task in the ABS-brake system
could be ReadVelocity(). In order to construct a schedule first the time to
process all the different tasks is needed. This provides a overview of all the
different tasks needed to complete the function and their respective runtime.

\subsection{Scheduling}
To ensure that the deadlines \textbf{always} holds true, scheduling is needed.
By explicitly stating a schedule for the system to follow, a scheduling analysis
can be conducted in order to ensure that the schedule does in fact always hold
true and no deadlines are missed. Scheduling is one of the
fundamental concepts of RTS. But for scheduling to work, knowledge is needed
about how long tasks takes. 
