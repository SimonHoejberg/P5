\chapter{Platform Choice}\label{PlatformC}
The purpose of this chapter will be to analyse the different hardware platforms available
and ultimately select one for the project. The platforms to be examined are Lego
NXT Mindstorms and Arduino. These are chosen due to the fact that these
platforms are offered by the university and that they are suited for an embedded system.
First the platforms will be described briefly, then they will be
compared based on a set of criteria. In conclusion a platform for the project
will be chosen.

\section{Evaluation Criteria}\label{EvalCrit}
In order to compare the two platforms we have to establish a set of criteria
which which can be used to evaluate their individual strengths. The criteria
will be based on the ease of use, availability of components and the limitations
of the hardware. The comparison of NXT and Arduino are based on the following
criteria:

\begin{itemize}
  \item Building - The turret needs to be physically constructed, thus an
  important factor is the need for a simple construction process, which can
  allow for more focus on the software aspect.
  %\item Integration of Sensors - As mentioned before, the group doesn't want to
  %spend too much time on the physical aspect of construction, so the platform
  % should try to minimize the time spent on things like soldering and integration of electrical
  %components and sensors.
  \item I/O Ports - The number of available I/O Ports needs to be able to
  support several sensors and motors.
  \item Processing power - The processing power needs to be taken into account
  since it has an effect on how complex the developed program can be.
  \item Availability and Integration - This criteria refers to the ease of
  acquiring and implementing new components for the given platform, e.g. buying
  a specialized sensor module and implementing it using existing software
  libraries.
\end{itemize}

\section{Lego NXT Mindstorms}
The Lego NXT Mindstorms is the successor to the first robotics-platform released
by Lego the Robotics Invention System \citep{NXTHistory}. Although the NXT
product-line contains the NXT 1.0, 2.0 and the newer EV3 brick, in this project
the EV3 will be disregarded since this is not being offered by the university.
Regarding the NXT 1.0 and 2.0 brick, they are in fact identical from a hardware
perspective and so we will consider them the same in the scope of this project
\citep{NXTversions}\nl

The Lego NXT Mindstorms kit consists of the NXT brick and a set of
compatible sensors and actuators. The NXT brick \citep[P.70]{NXTStats} consists
of:

\begin{itemize}
  \item ARM 32-bit processor running at 48MHz with 256KB Flash storage and 64KB RAM
  \item 8-bit AVR running at 8MHz with 4KB Flash and 512B RAM
  \item 4 input ports compatible with the I2C standard and 3 output ports
\end{itemize}

The Lego NXT comes with the NXT-G programming environment which is a visual
programming language for use with the NXT platform \cite{NXTG}. There exist a
variety of unofficial programming languages for the NXT, such as leJOS(Java) and
RobotC(C) which makes it possible to write programs using the more popular programming
languages \citep{NXTProgrammingLanguage}.\nl
When using the LEGO NXT platform, one can make use of a multitude of 1st or 3rd
party peripherals. These specialized peripherals come programmed and assembled -
all the user has to do is handle the data received by the NXT brick.
Another feature of the platform is that the NXT can be directly combined with
Lego Technics, or just lego in general, to use as a building platform. This
makes construction simple, as it is not necessesary to manifacture your own
mechanical parts.

\section{Arduino}
The Arduino platform is developed as a tool for fast and effecient prototyping.
By design, the platform is designed to be used effeciently by people without
much knowledge about programming and electronics. The Arduino lineup consists of
a number of different boards, each with their own hardware specifications. As
such, a model with sufficient memory, processing power and I/O ports.
The arduino comes with its own programming language and IDE
\citep{ArduinoIntro}. For the sake of comparison we will use the Arduino Uno,
since this is the board being offered by the university.\nl

The Arduino Uno has the following technical specifications\citep{UNOSpecs}:

\begin{itemize}
  \item ATmega328P processor 16 MHz clockfrequency
  \item 32 KB Flash storage
  \item 2 KB SRAM
\end{itemize}

It should also be noted that as the Arduino platform is not limited to work with
specific hardware, it has access to a large variety of 3rd party peripherals
\citep{ArduinoComponents}. In addition, due to the popularity of the platform,
there appears to be a large dedicated community, which provides a large amount
of libraries to use with the modules.

\section{Combination of platforms}\label{CombPlat}
Instead of using only a single platform, it is possible to combine them. There
are two ways to achieve this. One option is to connect the platforms using the
I2C protocol. This can be achieved relatively simple using a breadboard, a
resistor and 2 cables for each platform \citep{ArduinoNXT}. The advantage of
this approach is the fact that we can combine the two platforms and their
respective modules as we please. This setup comes with the drawback that it can
be very restrctive in relation to the turret design, as everything needs to be
connected by cables to a breadboard. This design could be useful for a
prototype, but the design limitations could be too limiting on the turrets
mobility to be used in the final implementation. In addition, this approach will
also require more effort on our side, to ensure proper compatibility between the
platforms. The second option is to use the arduino platform with modules made
for the NXT. This can be achieved by using a special shield, that
extends the arduino with ports that are compatible with the NXT modules. This
would allow us to have the customizability of the arduino while still being able
to use the proprietary sensors from the NXT. However the NXT shield is limited
as it only supports up to two motors \citep{NXTShield}. Another issue with using
the arduino is the ability to build it into the product, as it is not designed
to be built into Lego the same way the NXT is.\nl

\section{Platform comparison}
Based on the criteria defined in \autoref{EvalCrit}, this section will be used
to evaluate the two prototyping platforms, and finally conclude which platform
will be used for the \name project. This evaluation is based on a combination of
the group's personal experiences and technical details of the individual system.
As such, the chosen platform is not only chosen based on objectivly comparable
factors, but also on the group members preferences and experiences in regards to
the two platforms.

% Based on a set of criteria the two prototyping platforms will be evaluated, and
% in conclusion a platform will be selected:

\begin{enumerate}
  \item Building - When it comes to the ease of constructing the physical
  aspects of the turret, we regard the NXT as being the better platform. This
  choice was made based on the fact that using the NXT gives access to the
  entirety of the Lego Technics system, which makes it easy to construct a
  technical platform. Using the Arduino platform could be difficult, as it could
  require us to manually manufacture the individual parts of the turret, e.g.
  designing a functional shaft for the cannon to rotate about or constructing a
  fireing mechanism.
  While it is possible to integrate an Arduino into the LEGO platform, we find
  that it is considerably easier to use the NXT platform, as it is specifically
  designed for that purpose. The NXT also has the advantage that the various
  pieces can be combined in a lot of ways, allowing change in the design without
  necessarily needing new parts.

  \item I/O ports - When it comes to availability and flexibility in I/O
  Arduino is a vastly superior platform, since it not only far surpasses the NXT
  in number of available ports, it also allows flexibility in whether the ports
  are input or output and it also has both analog and digital ports.
  
  \item Processing power - When comparing the technical specifications of the
  two platforms, the NXT is superior in terms of both clockspeed and memory.
  While the NXT has 48MHz clock frequency, 256KB Flash storage and 64KB RAM, a
  standard Arduino UNO has a clock frequency of 16 MHz, 32 KB Flash storage and
  2 KB SRAM.
  
  \item Availability and Integration - If we limit the peripherals used by the
  NXT to those specifically designed for the platform, the Arduino has access
  to a far wider selection of peripherals than the NXT. The Lego NXT
  peripherals are built and configured to be used with a Lego NXT brick, thus
  resulting in a much simpler and streamlined process, being almost
  plug-and-play. Based on personal experience, the Arduino often requires more
  work in regards to integrating the modules with the board. Also, as the NXT
  sensors are designed to be used with LEGO, they are easy to mount on the
  turret. In addition, while it is possible to use NXT-based sensors with the
  Arduino platform, this would result in potential design restrictions and
  problems as described in \autoref{CombPlat}. Thus considering that most of
  the modules needed for the project, such as sensors, are available for both
  platforms, the ease of implementing and mounting the sensors still favors the
  NXT.
\end{enumerate}

Since the university can provide the platforms, as well as some of the
peripherals, the cost aspect is not deemed as a vital factor. As mentioned
before the peripherals needed for the project are available for both platforms
so this is not the deciding factor either.\nl

In conclusion the group has chosen the Lego NXT platform for use in the project
based on the following reasons:
\begin{itemize}
  \item It is easier to physically build a prototype with Lego parts.
  \item Less time is spent on integration and more time spent on the software
\end{itemize}

Now the platform choices have been analysed in regards to the project and a
platform has been selected. In the next chapter, the available sensors will be
described, analysed and tested in order to select the most ideal sensor for the
project.
