\chapter{Platform Choice}\label{PlatformC}
doe sit work?
The purpose of this chapter will be to analyse the different hardware platforms available
and ultimately select one for the project. The platforms to be examined are Lego
NXT Mindstorms and Arduino, due to the fact that these platforms are offered by
the university and that they are suited for an embedded system.
First the platform the platforms will be described briefly, then they will be
compared based on a set of criteria. In conclusion a platform for the project
will be chosen.

\section{Evaluation Criteria}
The comparison of NXT and Arduino are based on the following criteria:

\begin{itemize}
  \item Building - The turret needs to be physically constructed, thus an
  important factor is the need for a simple construction process, which can
  allow for more focus on the software aspect.
  \item Integration of modules - As mentioned before, the group doesn't want to spend too
  much time on the physical aspect of construction, so the platform should try to minimize
  the time spent on things like soldering and integration of modules.
  \item I/O Ports - The number of available I/O Ports needs to be able to
  support a several sensors and motors.
  \item Processing power - The processing power needs to be taken into account
  since it has an effect on how complex the developed program can be.
  \item Availability - The availability of the platform is a factor since we
  need to be able to get components for the platform.
\end{itemize}

\section{Lego NXT Mindstorms}
The Lego NXT Mindstorms is the successor to the first robotics-platform released
by Lego the Robotics Invention System \citep{NXTHistory}. Although the NXT
product-line contains the NXT 1.0, 2.0 and the newer EV3 brick, in this project
the EV3 will be disregarded since this is not being offered by the university.
Regarding the NXT 1.0 and 2.0 brick, they are in fact identical from a hardware
perspective and so we will consider them the same in the scope of this project
\citep{NXTversions}\nl

The Lego NXT Mindstorms kit consists of the NXT brick and a set of
compatible sensors and actuators. The NXT brick \citep[P.70]{NXTStats} consists
of:

\begin{itemize}
  \item ARM 32-bit processor running at 48MHz with 256KB Flash storage and 64KB RAM
  \item 8-bit AVR running at 8MHz with 4KB Flash and 512B RAM
  \item 4 input ports compatible with the I2C standard and 3 output ports
\end{itemize}

The Lego NXT comes with the NXT-G programming environment which is a visual
programming language for use with the NXT platform. There exist a variety of
unofficial programming languages for the NXT, such as leJOS(Java) and
RobotC(C) which makes it possible to write programs using the more popular programming
languages \citep{NXTProgrammingLanguage}.

\begin{itemize}
  \item The external peripherals come programmed and assembled - all the user
  has to do is handle the data received by the NXT brick.
  \item The NXT can be combined with Lego Technics to use as a building
  platform. This makes construction simple, as it is not necessesary to
  manifacture your own mechanical parts.
\end{itemize}

\section{Arduino}
The Arduino was born as a tool for fast prototyping, aimed at people lacking a
background in electronics and programming. Over the years the arduino has
evolved from offering simple 8-bit boards to offering a variety of different
boards suiting different needs such as memory available, processing power,
number of ports on the platform, etc. The arduino comes with its
own programming language and IDE \citep{ArduinoIntro}. A large number of modules are available for
the arduino provided by Arduino themselves and 3rd party vendors, allowing for
construction of diverse and complex systems \citep{ArduinoComponents}. For the
sake of comparison we will use the Arduino Uno, since this is the board being
offered by the university.\nl

The Arduino Uno consists of:

\begin{itemize}
  \item ATmega328P processor 16 MHz clockfrequency
  \item 32 kB Flash storage
  \item 2 kB SRAM
  \item A large variety of 3rd party peripherals are available for the platform
  \citep{ArduinoComponents}.
  \item Due to the popularity of the platform, there is a large community
  surrounding it, providing a large amount of libraries to use with the
  modules.
  \item If the library doesn't exist or contain the necesary functionality, it
  will have to be written, thus time will have to be invested in order to make
  it work.
\end{itemize}

\section{Combination of platforms}
Instead of using only a single platform, it is possible to combine them. There
are two ways to achieve this. One option is to connect the platforms using the
I2C protocol. This can be achieved relatively simple using a breadboard, a
resistor and 2 cables for each platform \citep{ArduinoNXT}. The advantage of
this approach is the fact that we can combine the two platforms and their
respective modules as we please. The drawback would be that the setup is quite
clumsy since they are connected by cables to a breadboard, and might limit the
mobility of the turret. It will also require more effort on our side, to ensure
proper compatibility between the platforms.
The second option is to use the arduino platform with modules made for the NXT.
This can be achieved by using a by using a special shield, that extends the
arduino with ports that are compatible with the NXT modules. This would allow
us to have the customizability of the arduino while still being able to use the
proprietary sensors from the NXT. However the NXT shield is limited as it only
supports up to two motors \citep{NXTShield}. Another issue with using the
arduino is the ability to build it into the product, as it is not designed to
be built into Lego the same way the NXT is.\nl

\section{Platform comparison}
Based on a set of criteria the two prototyping platforms will be evaluated, and
in conclusion a platform will be selected:

\begin{enumerate}
  \item Building - In this regard the NXT wins big, since it can be combined
  with the entire Lego Technics system making it easy to construct a
  technical platform. Constructing a turret with arduino would require us to
  manufacture the parts necessary for the cannon. The NXT also has the
  advantage that the various pieces can be combined in a lot of ways, allowing
  change in the design without necessarily needing new parts.
  \item Integration of modules - The Lego NXT peripherals are built and
  configured to be used with a Lego NXT brick, thus resulting in a much simpler
  and streamlined process, being almost plug-and-play. The Arduino often
  requires more work in regards to integrating the modules with the board.
  \item I/O ports - The Arduino wins, since not only does it far surpass
  the NXT in number of available ports, it also allows flexibility in
  whether the ports are input or output and it also has both analog
  and digital ports.
  \item Processing power - When comparing the two platforms, the NXT is
  superior, both in terms of clockspeed and memory.
  \item Availability - The Arduino offers a far wider selection of peripherals
  than the NXT, although everything is produced by a 3rd party manufacturer. In
  reality, most of the modules needed for the project such as sensors are
  available for both platforms.
\end{enumerate}

Since the university provides the platforms the cost aspect is not deemed as a
vital factor. As mentioned before the parts needed for the project are
available for both platforms so this is not the deciding factor either.

In conclusion the group has chosen the Lego NXT platform for use in the project
based on the following reasons:
\begin{itemize}
  \item Easier to physically build a prototype with Lego parts
  \item Availability of the modules or cost of the platform isn't really a
  factor, since they both suit the needs of the project.
  \item Less time spent on module integration and more time on software
\end{itemize}

Now the platform choices have been analysed in regards to the project and a
platform has been selected. In the next chapter, the available sensors will be
described, analysed and tested in order to select the most ideal sensor for the
project.
