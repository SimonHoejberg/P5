\chapter{Platform Choice}

In this chapter we will analyse the different hardware platforms available.
Based on a set of criteria the platforms will be compared and in conclusion a
platform for the project will be chosen. The platforms to be examined are the
Lego NXT Mindstorms and Arduino, due to the fact that these are the
platforms offered by the university and that they are suited for an embedded
system. Before we examine the platforms we will decide on a set of criteria to
evaluate the platforms on:

\begin{enumerate}
  \item How easy is the platform to build and code on?
  \item How restricted is the platform in terms of processing power and memory?
  \item How many 3rd party and propietary modules does the platform have access
  to?
\end{enumerate}

\section{Lego NXT Mindstorms}
\fix{The Lego NXT Mindstorms is the successor to the first robotics-platform
released by Lego the Robotics Invention System.}{Start of sales talk} Although
the NXT line also contains 2 other platforms the NXT 1.0 and the EV3, in this project only the
NXT 2.0 will be considered since this is the platform being offered by the
university \Source.\nl

The Lego NXT Mindstorms kit consists of the NXT brick and a set of
compatible sensors and actuators. The NXT brick\citep[P.70]{NXTStats} consists
of:

\begin{itemize}
  \item ARM 32-bit processor running at 48MHz with 256KB Flash storage and 64KB RAM
  \item 8-bit AVR running at 8MHz with 4KB Flash and 512B RAM
  \item 100x60 pixel LCD display
  \item Four buttons
  \item Speaker that plays sound files with a sampling rate of 8kHz
  \item 4 input ports compatible with the I2C standard and 3 output ports
\end{itemize}

The Lego NXT comes with the NXT-G programming environment which is a visual
programming language for use with the NXT platform. There exist a variety of
unofficial programming languages for the NXT which makes it possible to use the
write programs using the more popular programming languages.

\begin{itemize}
  \item The external peripherals come programmed and assembled - all the user
  has to do is handle the data received by the NXT brick.
  \item The NXT can be combined with Lego Technics to use as a building
  platform. This makes construction simple, as it is not necessesary to
  manifacture your own mechanical parts.
  \item The NXT has a very limited amount of I/O ports.
\end{itemize}

\section{Arduino}
The Arduino was born as a tool for fast prototyping, aimed at people lacking a
background in electronics and programming. Over the years the arduino has
evolved from offering simple 8-bit boards to offering a variety of different
boards suiting different needs such as memory available, processing power,
number of ports on the platform, etc. The arduino comes with its own
programming language and an IDE. A large number of modules are available for
the arduino allowing the construction of diverse and complex systems\Source.

\begin{itemize}
	\item Due to the popularity of the arduino platform a large variety of 3rd
	party peripherals are available for the platform
	\item Due to the popularity of the platform, there is also a large amount of
	libraries for use with the modules
	\item If the library doesn't exist or contain the necesary functionality, it
	will have to be written, thus time will have to be invested in order to make
	it work
\end{itemize}

\section{Platform comparison}
Based on a set of criteria the two prototyping platforms will be evaluated, and
in conclusion a platform will be selected:

\begin{enumerate}
  \item I/O ports - The Arduino wins, since not only does it far surpass
  the NXT in number of available ports, it also allows flexibility in
  whether the ports are input or output and analog or digital.
  \item Processing power - When comparing the two platforms, the NXT is
  superior. Although the Arduino offers a selection of boards with varying
  processors and memory, allowing us to select the arduino board that suits our
  needs best. In reality if one of the more powerful arduino boards is chosen,
  the difference between the platforms in processor-speed and memory would be
  neglible.
  \item Availability - The Arduino offers a far wider selection of peripherals
  than the NXT, although everything is produced by a 3rd party manufacturer. In
  reality, most of the modules needed for the project such as sensors are
  available for both platforms.
  \item Integration of modules - The Lego NXT peripherals are built and
  configured to be used with a Lego NXT brick, thus resulting in a much simpler
  and streamlined process, being almost plug-and-play. The Arduino often
  requires more work in regards to integrating the modules with the board.
  \item Customization - Based on this criteria we consider the Arduino to be
  more flexible. Because of the more widespread use of the Arduino, the
  likelyhood of someone already having constructed what you need is big. There
  is also the option for tampering further with the modules with a breadboard
  and the electrical components such as resistors, capacitors, etc.
  \item Building - In this regard the NXT wins big, since it can be combined
  with the entire Lego Technics system in constructing a technical platform.
  Constructing a turret with arduino would require us to manufacture the parts
  necessary for the cannon. The Lego NXT also has the advantage that the various
  pieces can be combined in a lot of ways, allowing change in the design
  without necessarily needing new parts.
\end{enumerate}

Since the university provides the platforms the cost aspect isn't deemed as a
vital factor. As mentioned before the parts needed for the project are
available for both platforms so this is not the deciding factor either. When
choosing the platform the group focused on the following criteria:
\begin{itemize}
  \item Building - Since a physical turret will have to be constructed, in
  order to simplify the mechanical process and allowing further focus on the
  software aspect.
  \item Integration of modules - As before, the group doesn't want to spend too
  much time on the physical aspect of construction so things like soldering and
  integration of modules is inconvenient.
  \item I/O Ports - the number of available I/O Ports needs to be able to
  support a range of sensors and some motors. For this reason the NXT is a
  little limiting, compared to the arduino, although the use of a multiplexer
  can allow expansion of existing ports.
\end{itemize}

Another option would be to combine the two, this can be done by using an NXT
shield for the NXT. This would allow to have the customizability of the arduino
while still being able to use the proprietary sensors from the NXT. However the
NXT shield is limited as it only supports up to two motors. Another issue with
using the arduino is the ability to build it into the product, as it is not
designed to be built into Lego as the NXT is.\nl

In conclusion the group has chosen the Lego NXT platform for use in the project
based on the following criteria:
\begin{itemize}
  \item Easier to physically build a prototype with Lego parts
  \item Availability of the modules or cost of the platform isn't really a
  factor, since they both suit the needs of the project.
  \item Less time spent on module integration and more time on software
\end{itemize}

Now the platform choices have been analysed in regards to the project and a
platform was selected. In the next chapter, the available sensors will be
described, analysed and tested in order to select the most ideal sensor for the
project.