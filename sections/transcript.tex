\textbf{Claus:} Where is the storage located compared to the production?\nl

\textbf{Michal:} The production line itself is over here (points through a door
into a the production hall).
What shall we say, the actual production, welding and such, is unfortunately
down on the other side of the road. There is a bit of logistics, but here we
gather things in modules. That way you assemble...\nl

\textbf{Claus:} Do you then assemble the modules below in the field? (over on
the other side of the road )\nl

\textbf{Michal:} Yes, welding and all, and then from the painter, and up here
and then it is assembled.\nl

\textbf{Claus:} How is it (the store) structured when you need to find a
product, do you know exactly where things are?\nl

\textbf{Michal:} Yes, there's location for everything we have in stock. So it
will be placed at that location. A pallet has just exactly the pallet-space so
we can find the things again. Should more of the product arrive, we know exactly
where we can find it.\nl

\textbf{Claus:} I can see there are a lot of numbers and things like that.\nl

\textbf{Michal:} Yes it says the location on it, like 5A and 5B, so you can see
just what it is. These are things that are broken (points to
another shelf with parts), incorrect from a supplier or processed incorrectly or
something else.\nl

\textbf{Claus:} I was wondering how is it like to get things acquired? Do they
come (from production) and say that we need this and this, and then you obtain
it for them?\nl

\textbf{Michal:} We have a plan that says that we have to produce 300 machines a
year, and we will do that of course.  There are of course different models. Just
as there are different car models, there are of course also different
exxcavators, what we call an MX'er which is rotary machine.
If the plan says that in 2015 we have to make 90 of this one model, 80 of this
model and 100 of this model. And then \"sp\" is created such that it reflects
what we expect to sell. So we produce for orders only, and only after the
machine is sold it will be produced. This is of course because there is all
different kinds of extra equipment the buyer may want. It's like in a car where
there is a seat warmer, but in our machines it is likely some automatic system
or the like.\nl

\textbf{Claus:} So you buy components based on how many units you expect to
produce?\nl

\textbf{Michal}: Yes we do, procurement is of course a part of everything now.
Some components like engines for example, which are right here (pointing to a
good collection of engines), well there is a lead time, delivery time of a half
year.
In some of the dfficult parts, it is hard to suddenly change how many is needed,
for example if you sell a lot more of this one model because it is a different
engine, so we have a bit of a problem, it can be a little harder to deal with.
But otherwise it is all structured based on our plan. Otherwise it is done by
our planning department, for example if this one machine needs to be completed on
December 1, well then we look back, then we should aim to have assembled this
one machine six weeks before, so it can get all the way through production and
be ready for December 1st. Then we get a message from our hand-scanners where we
can see, well this must be delivered to production. It must be delivered on
these yellow carts we have there (pointing toward the entrance we came from
where there’s a number yellow carts, some loaded with components other empty).
It is these that the machines are assembled on.\nl

\textbf{Claus:} What if there is suddenly a huge unexpected order, does that
affect production? Is the production then on hold? And can you get all the
needed parts within a reasonable timeframe?\nl

\textbf{Michal:} On most things, there is a reasonable delivery times. We also
have agreements with some suppliers that they need to deliver within 14 days or
three weeks depending on the component, but we are pretty much locked in on the
plan (to make 300 machines per year) because there are some components that are
very difficult to get home, it is not only the engines, it is also gearboxes, it
can be shafts and stuff. Obviously we buy components from some of the big
companies, but we are not a huge company because, so we are not as prioritized,
as other companies. What we need may be something like 100 of one part, and 100
of another part, and 50 of a third. But then the larger company may use 10,000
parts, because they are much bigger than us, so we are of course not as high a
priority as them. So it can be a problem if there is some huge orders that say
that there is a customer who wants to have 40 machines and that they want them
now, and that is when we will be in trouble.\nl

% -- Del 2 --

\textbf{Claus:} Do you often experience that you are missing parts for the
different machinery?\nl

\textbf{Michal:} Yes we do. I cannot remember the exact number but when you are
talking about digging/loading machines such as we have over here, there is about
3000, 4000 different parts, if we split them up into the individually tiny
detail, before when can have a complete machine. Some things we weld ourselves,
for example the hood that needs to be sandblasted or painted.  We make
everything from scratch so there is many different parts which all have a unique
number. So when you look at all the parts to make the hood, there is a lot of
different small parts. So it cannot be avoided with this many parts that we run
in to problems now and then.\nl

\textbf{Claus:} Where do you put the machines if a part is missing?\nl

\textbf{Michal:} They won’t be released if big components are missing. Then the
planning says that they won’t start the production of the machine, because we
cannot start something we can’t complete.\nl

\textbf{Claus:} So do you experience that you are missing a part during the
production?\nl

\textbf{Michal:} Yes you never know if the engine that has been delivered is the
right one, or if it has a malfunction.\nl

\textbf{Claus:} When you need parts in the production how do you register it?\nl

\textbf{Michal:} The machine is released and sent to the work centers, the
different employees can see the task they have to work on. The warehouse or
another person then gets the parts.\nl

\textbf{Claus:} What if there is a faulty component?\nl

\textbf{Michal:} If there is a faulty component, this will be registered and
there will be made a complaint. These are then used to check if the supplier
delivers a good product. Human faults can also occur if the number of part taken
from the warehouse is wrong.\nl

\textbf{Claus:} How often is inventory taken?\nl

\textbf{Michal:} I was working at a different part of the warehouse, where the
inventory was taken once a year, but here it’s counted continuously. All the
parts that can be counted will be, for example if there is 20 of a part it is
counted but if there is 200 it will not be.\nl

% --- Del 3 ---

\textbf{Michal:} What should i say, 200 of the most expensive items we count,
and usually it adds up \nl

\textbf{Us:} Okay \nl

\textbf{Claus:} It usaully adds up?\nl
 
\textbf{Michal:} Yes it does\nl

\textbf{Claus:} It is not often that items goes missing?\nl

\textbf{Michal:} It can of course happen, but...\nl

\textbf{Claus:} But its not a big problem?\nl

\textbf{Michal:} No\nl

\textbf{Christoffer:} Not in the expensive end? \nl

\textbf{Michal:} It is not, of course it is offen the big items that also are
the most expensive. A machine we make should always have an engine, it is
suddenly not gonna need two.\nl

\textbf{Us:} No \nl

\textbf{Christoffer:} It is maybe more the small items like bolts?\nl

\textbf{Michal:} It is mostly the small items.\nl

\textbf{Claus:} Yes \nl

\textbf{Claus:} What about when you order parts. It takes some time for it to
arrvie. So how often do you order new parts regarding the masterplan. Do you
order it all at once?\nl

\textbf{Michal:}  No\nl

\textbf{Claus:} And get all the parts...\nl

\textbf{Michal:}  No the masterplan is always up to date so we order new parts
every day.\nl

\textbf{Claus:} Every day? \nl

\textbf{Michal:} The purchasing department orders new parts, but with the big
manufactors it all happens automatically. The purchasing department considered
acording to the masterplan which parts and when to order them from the smaller
manufactors. If the masterplan changes the department need to reconsider.\nl

\textbf{Claus:} Yes \nl

\textbf{Michal:} This also happens mostly automatic. When they have placed an
order it gets sent to the manufactor.\nl

\textbf{Claus:} But they need to greenlight it? \nl

\textbf{Michal:} Yes but not like they need to pick up the phone. It all
happens inside the computer system.\nl

\textbf{Claus:} So does the system work like all the sections of the company is
connected though the computersystem so if a error happens her, does all sections
of the company know it? \nl

\textbf{Michal:} If I am gathering the parts for a machine and I need 10 screws
and the warehouse only has 5 screws. Then we adjust the stock down to 5 and then
I have a backorder of 5 screws.\nl

\textbf{Claus:} Yes \nl

\textbf{Michal:} It only pops up to the purchasing department after a nightly
run. When it pops up it does it with alarm because it is in backorder. Then they
can take action on it and see if there is something they can do about it.\nl

\textbf{Claus:} Does it happen often that a part is in backorder?\nl
 
\textbf{Michal:} Well yes it happens often. We had our problems but there has
also been the financial crisis but we are on our way up again. But because of
the financial crisis workers had been fired and the stock has been run nearly
dry to save money. So backorders happens often\nl

\textbf{Claus:} Is it the big items that are in backorder? Or is it the small
items? \nl

\textbf{Michal:} It is many of the small items \nl

\textbf{Claus:} It is mostly the small items?\nl

\textbf{Michal:} Because the bigger items like motors all run in fixed
frameworks\nl

\textbf{Claus:} Yes \nl

\textbf{Michal:} It is run by the masterplan and is ordered, because it is half
a year out in the future we talking\nl

\textbf{Claus:} Yes \nl

\textbf{Michal:} So we need to predict what we sell and what not. Items like
antennas we get the day after the order. \nl

\textbf{Claus:} It is mostly the small items, do you have a number of suppliers
where you keep an eye on them, so if an item goes in backorder there is the
possibility to get the item from another supplier? \nl

\textbf{Michal:} Yes we have that option on most of the items\nl

\textbf{Claus:} Yes \nl

\textbf{Michal:} In the moment work starts on a new machine, the contact to
some suppliers are made about what they can offer in a category of items, then
the decision is made on which supplier and what item we buy.\nl

\textbf{Claus:} Yes and they have different prices?\nl

\textbf{Michal:} Yes different prices so if one can not supply us, then we are
forced to buy the rest or all from the supplier to keep the wheels spinning. \nl

\textbf{Us:} Yes \nl

\textbf{Michal:} But that does not happen often, the suppliers do really want
to supply us, so they do everything they can to comply with our plan
which they also get, but only the parts we expect to buy from them so they can
produce enough.\nl

% -- Del 4 – (Michal / Lagerchef?, Jeg har skrevet Michal)

\textbf{Michal:} Yeah, if it all comes together.\nl

\textbf{Claus:} If I could get back to those backorders, if you find that you
lack components and they then have to send some over, is that another two weeks
of transport?\nl

\textbf{Michal:} That depends, some you can have the very next day if they are
in stock. Because if it is a Danish supplier or an international supplier from
Spain or Germany. Depending on where the supplier is located in the world, I
mean the components I ordered from Italy earlier today are arriving tomorrow, if I'm
willing to pay for it. Otherwise they will arrive within the next couple of
days.\nl

\textbf{Claus:} If you would have to pick, what do you think the biggest problem
you have in the warehouse is?\nl

\textbf{Michal:} Hmm, as we're going now, in terms production, because we are
following the plan, when we sell a machine it has to be delivered as fast as
possible. This causes the needed personnel in the warehouse to fluctuate. But we
would like to work relatively stable, but we can't, because suddenly there are
people who got nothing to do, if we are not prepared to gather.\nl

\textbf{Michal:} It is the same with purchases, because suddenly we have 3 weeks
of summer vacation. A supplier in Italy has the whole of August off, while we
are off the whole of July.  Then suddenly we have supplies from Italy coming in
on 5 fully loaded trucks. That is not something we can just put in the
warehouse.
So this causes some instability, and the same with the backorders, both bought and
self-produced components. This gives quite a bit more to do, because when we
have to gather parts for a machine at 200 parts, I would prefer to find all of
them at once. But often I have to start with finding the first 185 parts and then
wait on the system telling me that a component we had on backorder has arrived,
and then go find it.  So in theory, every single time one of the last 15
components arrive, I have to go fairly far to find out who actually needs it.\nl

\textbf{Claus:} So you would rather have all the components and then deliver
them all at once?\nl

\textbf{Michal:} Yes, that is far faster. We have taken actions this spring to
remove the masterplan to figure out where the actual problems are located. It is
especially at the self-produced components that we have an issue.\nl

\textbf{Claus:} So it is simply the masterplan that is the problem? \nl

\textbf{Michal:} Yeah, it is not actually during purchase. \nl

\textbf{Claus:} It should be more like, when the order arrives?\nl

\textbf{Michal:} Exactly. That is probably our biggest issue. Luckily we are
about to get a new EDP-system that can help with some of the stock-related
problems, with some kind of intelligent system where it can tell us, this pallet
is actually empty. There is also a problem when a new ware arrives. We will take
a walk around the facility before you leave so you can see it. Okay, when a new
ware arrives, I have a man in here, and he is like; where the hell do I find
the space for this thing? Because it is not like we have a lot of space. We
probably don't have enough. We lack some space, so that we have to look for
empty spaces for new wares.\nl

\textbf{Claus:} So you don't have a map of where the components are? \nl

\textbf{Michal:} Yeah, it is kind of... \nl

\textbf{Claus:} So you fill it up dynamically?\nl

\textbf{Michal}: We probably should, but we are not. We have our EDP, it is not
entirely new any longer, and it does not have the control-options that tell us
that a pallet is empty, or this one product is used all the time, why is it
located up under the roof? The only way we can see that, is by noticing, this
pallet is very dusty. Or reverse, this pallet is used all the time, why is it up
under the roof? 

\textbf{Claus}: So you can't see when it is bought or anything? \nl

\textbf{Michal:} We can, but it is not easily accessible. There is no automatic
way to do it. There is no excel spreadsheet to just take a look at.\nl

\textbf{Michal:} Take this, we want to divide our stock into something called A,
B and C. Where A is components we use all the time, B is used often enough and C
is used seldom. The same with stock-location, which w have divided into A, B
and C, so we would like the A-components to be on an A-location, where A is the
closest and close the ground so they are easy to get too. B those are the ones
that are a bit further away and where we might need to get up into a height of 2 meters,
and C should be just under ceiling. But right now it is kind of where there is
room. So for instance, this pallet is dusty, why is it on an A-location? So I am
looking forward to the new system, so we can sort it out.\nl

\textbf{Claus:} I certainly hope so. \nl

\textbf{Michal:} But it demands a bit of attention, and the EDP needs the
information to process it. So we need discipline, since you've got to tell the
EDP that you are moving this location over her, so it's going to be exciting.\nl

\textbf{Claus:} Right, I am out of anymore questions, so if there is anything
else?\nl

\textbf{Michal:} How about we go for a quick walk around the warehouse, so you
can get a better feel of it?\nl

% -- Del 5 --
\textbf{Michal:} This is our goods reception, here our computer system can
immediately see that goods has come inhouse and trucks coming. Then we offload
it, so our computer system can register it has come, it is not controlled, but
it is registered, so that the purchases and who might have to use some of the
bits, they can see if we have got it in the house, and then we have to make a
reception control and everything afterwards...\nl

\textbf{Claus:} ...So you can make sure, the goods are in order…\nl

\textbf{Michal:} ...The goods are in order, yes\nl

\textbf{Michal:} This is where we have all the pieces on the shelves, where we
have such a 4 and 3 rows were there are 100 drawers in each.\nl

\textbf{Claus:} All the screws are here, and so the bigger parts they are...\nl

\textbf{Michal:} ...Yep, all the small parts is here and so on.\nl

\textbf{Claus:} Is it only the screws here, that is automated, do you have any
plans for automation of shelves, like screws \nl

\textbf{Michal:} No \nl

\textbf{Claus:} No, because it will be too expensive.\nl

\textbf{Michal:} Yes, and our problem is that it is very, very different parts,
there are not that many standard things, because our number of machines it is
not so high, so if we just ... 300 is maybe too much, that many product we may
not even do, it is perhaps only 200 here in Støvring ... and many of the things
are the customer order, and it may be in one place, one day, next day it is
another place. No years are alike.\nl

\textbf{Claus:} You also make some of the parts yourselves?\nl

\textbf{Michal:} Yes, that we make ourselves. \nl

\textbf{Claus:} So must take it in the old way \nl

\textbf{Michal:} Yes, we are. If we have to speed up, things we use the most
must be in the right place, that means getting them closer to the ground...\nl

\textbf{Claus:} And things used less, higher up.\nl

\textbf{Michal:} Yeah \nl

\textbf{Claus:} Down there, it is just part of the warehouse\nl

\textbf{Michal:} Yes, it is. \nl

\textbf{Claus:} It is only large things, that can not get up on the shelves?\nl

\textbf{Michal:} There are many things, like giant plastic screens and so on.\nl

\textbf{Claus:} Wheels \nl

\textbf{Michal:} Yes, and large plastic screens and..\nl

\textbf{Claus:} A little too big and too heavy to put on the shelf.\nl

\textbf{Michal:} Yep, you can see it is not because there is too much space
here.
\nl

\textbf{Claus:} Yes \nl

\textbf{Michal:} So we are fighting against that, and it is obviously
not super effective. you may need to move something, to find something else.\nl

\textbf{Claus:} Well, there is a order, such that more heavy things are down
here and the less heavy stuff is over there.\nl

\textbf{Michal:} Yeah, then again there is logistic of it all. \nl

\textbf{Claus:} What are the numbers? \nl

\textbf{Michal:} Yes numbers, you can see the white notes on the shelves ...
well that is specific documentation that tells there is the thing. is one
documentation not enough, well. \nl

\textbf{Claus:} It gets one more. \nl

\textbf{Michal:} A small sticky note that says G on the pallet there, it is such
a reserve logistic thing, which says simply that there are 20 of them over here
on this location ... so you can find your way around, for some times, then they
come home in batches, then there is maybe five pallets at once, so it can not be
put there. \nl

\textbf{Claus:} It must still be a little confusing to find your way around,
with all the numbers here\nl

\textbf{Michal:} Yeah it is,  there are also many, around 30000 Part
Numbers\nl

\textbf{Claus:} Do you have to look up things,  in order to find them.\nl

\textbf{Michal:} When we get our paper those we pick after, the location is
written on, it simply states, first is the location, and then stand number.  we
need one part, and then you have to scan the barcode and then enter that you
take a piece, one piece or 10 pieces, you have to enter it, so you should not be
able to make picking errors if one can say so, because you have to scan items\nl

\textbf{Claus:} What if i is started to pick and part not on ...... but must
have 5 but there are only three, what do you do\nl

\textbf{Michal:} Then we have to go to our computer system and find out what the
hell it just happened, it's likely that the goods receipt has not received
something. they can receive something 5 min ago which is not distributed to its
place yet, and at the moment there are items received go pick out, so in theory
you can still go for things that has not been distributed and it can obviously
be what has happened, huh hell it's not true, we just got 10 home and we must
use the 5 and 5, well then you go in and says the order from yesterday where we
had to find five, well, did we found five, or did we take seven, so there will
be a little bit like sherlock Holmes, why, and then we'll then adjust\nl

\textbf{Claus:} You can spend a lot of time to run back and forth here\nl

\textbf{Michal:} In theory, yes, now it is not insanely much, but yes it can
\nl

\textbf{Claus:} But the dead time, there will be \nl

\textbf{Michal:} Wasted time there will be, yes... especially when
we are so many, you have to be really careful that you do not have some of
those loose ends, so once in a while you have to complete it, there is no use
in taking that pallet, and putting it right here in the back, because I have to
examine whether there is something that is right, because there may be
another one that: “where the hell is that pallet” he comes in to me, and why is
it missing, and I do not know about him over here, taken this out, so there is little discipline
when we have those 10 men, 11 men making things done and that is done
very simply. \nl

\textbf{Claus:} Well, that is it, thank you for the interview.\nl

\textbf{Michal:} You are welcome \newpage



