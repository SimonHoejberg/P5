\chapter{Projektil-fysik}
Ved affyring af et projektil er det et antal informationer som er n�dvendige for
pr�cist at kunne ramme et m�l. Disse informationer kommer i form af
matematiske udregninger og fysiske forst�elser. Dette er n�dvendigt, da den
producerede kanon har et begr�nset udvalg af data til r�dighed, som skal
behandles og anvendes til at v�re i stand til at ramme det udvalgte m�l.\nl

P� baggrund af denne problemstilling, vil dette kapitel vil blive brugt til, at
redeg�re for de fysikfaglige elementer, der er relevante for affyringen af et
projektil. Dette vil hovedsageligt blive d�kket ved at pr�sentere og forklare de
fysiske formler der har at g�re med kasteparabler. Disse formler er udvalgt,
da den vigtigste del af affyringen er, at udregne med hvilken kraft og vinkel
projektilet skal affyres for effektivt at ramme et givent m�l.\nl

\section{Kasteparabel}
For at kunne udregne et projektils kasteparabel, skal man have adgang til et
antal informationer. Disse informationer er:
\begin{itemize}
  \item Starthastighed $v_0$ ($m/s$)
  \item Affyringsvinkel $a$ ($grader$)
  \item Tyngdeacceleationen $g$ ($m/s^2$)
  \item 
\end{itemize}


