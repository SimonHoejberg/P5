\section{Projectile Motion}\label{ProjMotion}
When firing a projectile it is possible to determine its theoretical trajectory
by using the formula for projectile motion. This theorem is important to the
project as it can be used to map the projectiles relative location to the turret
over a given time frame. This formula is split into two parts, one for
calculating the x coordinate and one for the y coordinate. The coordinates are
used to represent distance and height of the projectile. The formulas are
presented in \citep[Ch. 6.2]{OrbitBA} and are as follows:

\begin{equation}\label{slength}
x=v_0*cos(a)*t
\end{equation}
\begin{equation}\label{sheight}
y=v_0*sin(a)*t-\frac{1}{2}*g*t^2
\end{equation}

The formulas use the following variables:
\begin{itemize}
  \item $v_0\ (m/s)$ - Starting Velocity
  \item $a\ (degrees)$ - Angle of Fire
  \item $g\ (m/s^2)$ - Gravitational Acceleration (Constant)
  \item $x\ (m)$ - Distance
  \item $y\ (m)$ - Height
  \item $t\ (s)$ - Time
\end{itemize}
In addition, the following variables are used in the following sections:
% tried to make it more thorough and added x_0
\begin{itemize}
  \item $y_0\ (m)$ - Height offset of the turrets muzzle
  \item $x_0\ (m)$ - Distance offset of the turrets muzzle
\end{itemize}

By using this theorem, we can use the projectiles initial velocity, combined
with the angle of firing, to determine how far the object will fly.\nl

It should be noted that the formulas \autoref{slength} and \autoref{sheight} do
not take the effect of air resistance on the projectile into account. 

% The topic of air resistance and the relevance thereof is discussed in
% \fix{\autoref{AirResDisc}}{Where is it?}.

\subsection{Calculating Distance}\label{CalcDist}
In order to calculate the distance at which the projectile will touch the ground
we need to combine \autoref{slength} and \autoref{sheight}. This is done in
order to express the horisontal distance $\mathbf{x}$ as a function of the
height $\mathbf{y}$, as the height can then be set to zero in order to determine
where the projectile will land. This point can be seen at the end of
arrow $\mathbf{p}$ in \autoref{DistRepresent}. To do this we first need to take
\autoref{sheight} and express the time $\mathbf{t}$ as a function of the height
$\mathbf{y}$. As the intention is for $\mathbf{y=0}$, this is also assumed in
the resulting formula:
\begin{equation}\label{Eq1}
0=y_0+v_0*t*sin(a)-\frac{1}{2}*g*t^2
\end{equation}

%Note: vi skal nok bruge opdateret billede (quality and current version eller
% skifte det ud med en tegning)
\figx[0.2]{DistRepresent}{Projectile trajectory on the x- and y-axis.}
Now we can isolate for $\mathbf{t}$ in \autoref{Eq1}. Using the quadratic
formula gives two results, but assuming a positive value for $\mathbf{a}$ we get
the following expression:
\begin{equation}\label{Eq2}
t=\frac{v_0*sin(a)}{g}+\frac{\sqrt{v_0^2*sin^2(a)+2*g*y_0}}{g}
\end{equation}

Now we can substitute \autoref{Eq2} into \autoref{slength}. The resulting
formula expresses the distance at which the projectile will land $\mathbf{x}$ as
a function of the angle $\mathbf{a}$:
\begin{equation}\label{FinEq}
x\ =\ v_0*cos(a)* \left(\frac{v_0*sin(a)}{g} + \frac{
\sqrt{v_0^2*sin(a)^2+2*g*y_0}}{g}\right)
\end{equation}
We can rewrite \autoref{FinEq} for simplicity. We also add $\mathbf{x_0}$ to
the result in order to represent any displacement on the x-axis:
\begin{equation}\label{FinEq2}
x\ =\ \frac{v_0*cos(a)}{g}* \left(
v_0*sin(a)+\sqrt{v_0^2*sin(a)^2+2*g*y_0}\right)+x_0
\end{equation}

Another important point is that the variables $\mathbf{y_0}$ and $\mathbf{x_0}$
changes dynamically based on angle of the turret. This is because the muzzle of
the turret changes its vertical and horizontal position based on the angle. This
can be seen in \autoref{FinalPhysTurret}.
\figx[0.18]{FinalPhysTurret}{Turret marked with the variables a, r, h, $x_0$
and $y_0$}

In order to calculate $y_0$ for any given angle we can make use of the sine for
the angle. Looking at \autoref{FinalPhysTurret} we can formulate the following
formula:
\begin{equation}\label{Y0Eq}
y_0\ =\ h\ +\ r\ *\ sin(a)
\end{equation}

In order to calculate $x_0$ we can use the cosine for the angle to determine the
distance. We can formulate the following formula:
\begin{equation}\label{Y0Eq}
x_0\ =\ r\ *\ cos(a)
\end{equation}

Substituting $y_0$ into \autoref{FinEq2} and adding $x_0$ gives
us the following formula:
\begin{equation}\label{FinEq3}
x\ =\ \frac{v_0*cos(a)}{g}* \left(
v_0*sin(a)+\sqrt{v_0^2*sin(a)^2+2*g*(h+r*sin(a))}\right)+r*cos(a)
\end{equation}

Now, using \autoref{FinEq3} it is possible to calculate the
distance the projectile will travel given the initial velocity $\mathbf{v_0}$,
the angle $\mathbf{a}$, the turrets height $\mathbf{h}$ and the length of the
turret $\mathbf{r}$.

\subsection{Angle-Distance Relation}
When firing, the turret has one primary variable it can use to aim,
namely the angle at which the projectile is fired. Given a constant, initial
velocity of the projectile, the travelled distance and air time is proportional
to the firing angle $\mathbf{a}$. This relation is shown in \autoref{FinEq2}
where $\mathbf{cos(a)}$ and $\mathbf{sin(a)}$ are used as coeffecients. As
such, the furthest possible shot would be achieved using $\mathbf{a=45\
degrees}$ such that $\mathbf{sin(45)\approx 0.71}$ and $\mathbf{cos(45)\approx
0.71}$. The relation between firing angle and travelled distance is exemplified
in \autoref{ProjMot2.png}.

% This variable is
% theoretically enough to accurately fire the turret, but it does present some
% problems when it comes to calculating the trajectory and the offset needed for
% targets in motion.\nl


\figx[0.35]{ProjMot2.png}{Distance relative to angle of fire.}

\subsection{Projectile Air Time}
When it comes to hitting a target in motion we need another crucial piece of information,
namely how long a given projectile takes to reach its target.
In \autoref{CalcDist} we determined a formula for determining the distance on
the $\mathbf{x-axis}$ at which the projectile would hit the ground. As we
already determined a value for $\mathbf{x}$, we can simply use
\autoref{slength}, namely:
\begin{equation}
x=v_0*cos(a)*t
\end{equation}
By isolating for the time in this formula, we can use our value for $\mathbf{x}$
to determine how long it takes for at given projectile to reach the target. This
results in the following formula:
\begin{equation}\label{TimeEq}
t=\frac{x}{v_0*cos(a)}
\end{equation}

% \subsection{Air Resistance (Not Done)}\label{AirResDisc}
% For the purpose of calculating the trajectory of the projectile in \autoref{} a
% formula has been used which does not account for the effect of air resistance.
% This section will be used to discuss why this is the case, and to argue why the
% effect of air resistance is negligible in the case of firing the chosen
% projectile.\nl
% 
% In cases where a projectile is fired at high speeds, and is travelling for a
% longer period of time it is important to factor in the effect of air resistance,
% as this can have a significant effect on the trajectory of the projectile. On
% the other hand when we have relatively slow moving projectiles which are only mid
% air for a short duration, the effect of air resistance becomes much less
% important.\nl
% 
% % Ville gerne vise dette ved at udregne med og uden luftmodstand, men fuck
% % differentialligninssystemer.
% Another reason for using the simplified formula which disregards air resistance
% is that the Lego Nxt has limited processing power, and that the turret has
% strict deadlines. This leads to a situation where the speed of calculation is
% more important than the unnecessary precision of factoring in air resistance.
% 

% All of these variables can be deduced from two key pieces of information; the
% starting velocity, the angle of fire and the height of firing. For the
% constructed turret the angle of fire varies from shot to shot, but the starting
% velocity would always be constant. As the angle of fire is easily observed, the
% only key piece of information needed is the starting velocity. In order to
% determine this velocity, a number of experiments were made. These are presented
% and explained in \autoref{}. The data from these experiments have resulted in
% the following initial velocity:\nl
% \begin{equation}
% v_0\ =\ 4.87\ m/s
% % \end{equation}

% In addition to the initial velocity, the height of firing needs to be
% determined. As the height of the cannon barrel changes based on the angle of
% fire, this can be calculated using trigonometry.
%
% By using the value for $v_0$, and the equation for the height of firing, the
% projectiles trajectory can be determined.
% When firing a projectile there is a number of informations which are needed in
% order to hit a target. This information consists of calculations and an
% understanding of the physics that go into firing a projectile. This is
% necessary due to the fact that the canon has a limited amount of input that
% needs to be processed in order to hit the target.

% Based on this problem, this chapter will be used to explain the physics needed
% to fire the projectile. This will mainly be covered by presenting and explaining
% the equations used to calculate the projectile motion. These equations has been
% chosen du to tha fact that calculating the force and angle of the projectile is
% the most important part when firing a projectile in order to hit a target.\nl
%
% \section{Kasteparabel}
% For at kunne udregne et projektils kasteparabel, skal man have adgang til et
% antal informationer. Disse informationer er:
%
% To be able to calculate the projectile motion, a number of informations i
% needed. These informations are:
% \begin{itemize}
%   \item Starting velocity $v_0$ ($m/s$)
%   \item Angle of fire $a$ ($degrees$)
%   \item Gravitational accelleration $g$ ($m/s^2$)
%   \item Height of firing $y_0$ ($m$)
%   \item Distance $d$ ($m$)
%   \item Time in the air $t$ ($s$)
% \end{itemize}
%
% If air drag is excluded, a number of statements can be made based on these
% informations:\nl
% When the projectile is fires it is not affected by any external forces other
% than the gravity. This means that the velocity can be calculated as:
% $v_x = v_0 * cos(a)$\nl
%
% Dette udsagn passer også på\ldots \nl
%
% Mere tekst\ldots \nl
%
% Disse udregninger giver os så formlen:\nl
% $v_t = (\frac{v_0\ *\ cos(a)}{-g\ *\ t\ +\ v_0\ *\ sin(a)})$
%
% \subsection{Diskussion: Luftmodstand}
% Denne subsection vil blive brugt til at diskutere hvorfor luftmodstanden kan
% ignoreres for den valgte type af projektil.
%
