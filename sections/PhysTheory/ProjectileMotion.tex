\section{Projectile Motion}\label{ProjMotion}
When fireing a projectile it is possible to determine its exact trajectory by
using the formula for projectile motion. This theorem is important to the
project as it can be used to map the projectiles relative location to the turret
over a given time frame. This formula is split into two parts for calculating
the x- and y-coordinate respectively. The formulas are as follows:
\begin{equation}\label{slength}
x=v_0*cos(a)*t
\end{equation}
\begin{equation}\label{sheight}
y=v_0*sin(a)*t-\frac{1}{2}*g*t^2
\end{equation}

The formulas use the following variables:
\begin{itemize}
  \item $v_0$ ($m/s$) - Starting Velocity 
  \item $a$ ($degrees$) - Angle of Fire 
  \item $g$ ($m/s^2$) - Gravitational Acceleration (Constant)
  \item $x$ ($m$) - distance
  \item $t$ ($s$) - Time
\end{itemize}
In addition, the following variables are used in the following sections:
\begin{itemize}
  \item $y_0$ ($m$) - Height of Fireing 
  \item $y$ ($m$) - height
\end{itemize}

By using this theorem, we can use the projectiles initial velocity, combined
with the angle of fireing, to determine where the fired projectile will
land.\nl
It should be noted that the formulas \autoref{slength} and \autoref{sheight} do
not take into accound any effect air resistance could have on the projectiles
trajectory. The topic of air resistance and the relevance thereof is discussed
in \autoref{AirResDisc}.

\subsection{Calculating Distance}\label{CalcDist}
In order to calculate the distance at which the projectile will touch the ground
we need to combine \autoref{slength} and \autoref{sheight}. This is done in
order to express the horisontal distance $\mathbf{y}$ as a function of the
height $\mathbf{x}$. To do this we need to take \autoref{sheight} and express
the time $\mathbf{t}$ as a function of the height $\mathbf{y}$. This is done as the
resulting expression can be substituted into \autoref{slength}. Doing this
and substituting for $\mathbf{t}$ in \autoref{slength} gives the equation:
\begin{equation}\label{FinEq}
x\ =\ v_0*cos(a)*\frac{v_0*sin(a)+\sqrt{v_0^2*sin(a)^2-2*g*y_0}}{g}
\end{equation}

Now, \autoref{FinEq} can be used to calculate the distance the projectile will
travel given the initial velocity $\mathbf{v_0}$, the angle $\mathbf{a}$ and
the turrets height $\mathbf{y_0}$.

\subsection{Angle-Distance Relation}
When fireing the turret has one primary variable it can use to aim,
namely the angle at which the projectile is fired. This variable is
theoretically enough to accurately fire the turret, but it does present some
problems when it comes to calculating the trajectory and the offset needed for
targets in motion.\nl

Given a constant initial velocity of the projectile, the distance travelled and
air time is proportional to the fireing angle. This relation is shown in
\autoref{FinEq} where $\mathbf{cos(a)}$ and $\mathbf{sin(a)}$ are used as
coeffecients. As such, the furthest possible shot would be achieved using
$\mathbf{a=45\ degrees}$ such that $\mathbf{sin(45)\approx 0.71}$ and
$\mathbf{cos(45)\approx 0.71}$.

The relation between fireing angle and travelled distance is exemplified in
\autoref{ProjMot.png}.

\figx[0.25]{ProjMot.png}{Distance relative to angle of fire.}

\subsection{Projectile Air Time}
When it comes to hitting a target in motion we need another crucial information,
namely how long a given projectile takes to reach its target.
In \autoref{CalcDist} we determined a formula for determining the distance on
the $\mathbf{x-axis}$ at which the projectile would hit the ground. As we
already determined a value for $\mathbf{x}$, we can simply use
\autoref{slength}, namely:
\begin{equation}
x=v_0*cos(a)*t
\end{equation}
By isolating for the time in this formula, we can use our value for $\mathbf{x}$
to determine how long it takes for at given projectile to reach the target. This
results in the following formula:
\begin{equation}\label{TimeEq}
t=\frac{x}{v_0*cos(a)}
\end{equation}


% All of these variables can be deduced from two key pieces of information; the
% starting velocity, the angle of fire and the height of fireing. For the
% constructed turret the angle of fire varies from shot to shot, but the starting
% velocity would always be constant. As the angle of fire is easily observed, the
% only key piece of information needed is the starting velocity. In order to
% determine this velocity, a number of experiments were made. These are presented
% and explained in \autoref{}. The data from these experiments have resulted in
% the following initial velocity:\nl
% \begin{equation}
% v_0\ =\ 4.87\ m/s
% % \end{equation}

% In addition to the initial velocity, the height of fireing needs to be
% determined. As the height of the cannon barrel changes based on the angle of
% fire, this can be calculated using trigonometry. 
% 
% By using the value for $v_0$, and the equation for the height of fireing, the
% projectiles trajectory can be determined. 
% When fireing a projectile there is a number of informations which are needed in
% order to hit a target. This information consists of calculations and an
% understanding of the physics that go into fireing a projectile. This is
% necessary due to the fact that the canon has a limited amount of input that
% needs to be processed in order to hit the target.

% Based on this problem, this chapter will be used to explain the physics needed
% to fire the projectile. This will mainly be covered by presenting and explaining
% the equations used to calculate the projectile motion. These equations has been
% chosen du to tha fact that calculating the force and angle of the projectile is
% the most important part when fireing a projectile in order to hit a target.\nl
% 
% \section{Kasteparabel}
% For at kunne udregne et projektils kasteparabel, skal man have adgang til et
% antal informationer. Disse informationer er:
% 
% To be able to calculate the projectile motion, a number of informations i
% needed. These informations are:
% \begin{itemize}
%   \item Starting velocity $v_0$ ($m/s$)
%   \item Angle of fire $a$ ($degrees$)
%   \item Gravitational accelleration $g$ ($m/s^2$)
%   \item Height of fireing $y_0$ ($m$)
%   \item Distance $d$ ($m$)
%   \item Time in the air $t$ ($s$)
% \end{itemize}
% 
% If air drag is excluded, a number of statements can be made based on these
% informations:\nl
% When the projectile is fires it is not affected by any external forces other
% than the gravity. This means that the velocity can be calculated as:
% $v_x = v_0 * cos(a)$\nl
% 
% Dette udsagn passer også på\ldots \nl
% 
% Mere tekst\ldots \nl
% 
% Disse udregninger giver os så formlen:\nl
% $v_t = (\frac{v_0\ *\ cos(a)}{-g\ *\ t\ +\ v_0\ *\ sin(a)})$ 
% 
% \subsection{Diskussion: Luftmodstand}
% Denne subsection vil blive brugt til at diskutere hvorfor luftmodstanden kan
% ignoreres for den valgte type af projektil.
% 
