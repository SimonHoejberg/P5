\chapter{Projektil-fysik}
When fireing a projectile there some information is needed in order to hit a
target. These informations consists of calculations and an understanding of the
physics that go into fireing a projectile. This is necessary due to the fact
that the canon has a limited amount of input that needs to be processed in order
to hit the target.

Based on this problem, this chapter will be used to explain the physics needed
to fire the projectile. This will mainly be covered by presenting and explaining
the equations used to calculate the projectile motion. These equations has been
chosen du to tha fact that calculating the force and angle of the projectile is
the most important part when fireing a projectile in order to hit a target.\nl


\section{Kasteparabel}
For at kunne udregne et projektils kasteparabel, skal man have adgang til et
antal informationer. Disse informationer er:

To be able to calculate the projectile motion, a number of informations i
needed. These informations are:
\begin{itemize}
  \item Starting velocity $v_0$ ($m/s$)
  \item Angle of fire $a$ ($degrees$)
  \item Gravitational accelleration $g$ ($m/s^2$)
  \item Height of fireing $y_0$ ($m$)
  \item Distance $d$ ($m$)
  \item Time in the air $t$ ($s$)
\end{itemize}

If air drag is excluded, a number of statements can be made based on these
informations:\nl
When the projectile is fires it is not affected by any external forces other
than the gravity. This means that the velocity can be calculated as:
$v_x = v_0 * cos(a)$\nl

Dette udsagn passer også på\ldots \nl

Mere tekst\ldots \nl

Disse udregninger giver os så formlen:\nl
$v_t = (\frac{v_0\ *\ cos(a)}{-g\ *\ t\ +\ v_0\ *\ sin(a)})$

\subsection{Diskussion: Luftmodstand}
Denne subsection vil blive brugt til at diskutere hvorfor luftmodstanden kan
ignoreres for den valgte type af projektil.

