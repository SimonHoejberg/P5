\section{Projectile Motion}
When fireing a projectile it is possible to determine its exact trajectory by
using the formula for projectile motion. This formula can be used to map the
projectiles location over a given time frame. This is done using the following
information:
\begin{itemize}
  \item Starting Velocity $v_0$ ($m/s$)
  \item Angle of Fire $a$ ($degrees$)
  \item Gravitational Acceleration $g$ ($m/s^2$)
  \item Height of Fireing $y_0$ ($m$)
  \item Distance $d$ ($m$)
  \item Time $t$ ($s$)
\end{itemize}

All of these variables can be deduced from two key pieces of information; the
starting velocity, the angle of fire and the height of fireing. For the
constructed turret the angle of fire varies from shot to shot, but the starting
velocity would always be constant. As the angle of fire is easily observed, the
only key piece of information needed is the starting velocity. In order to
determine this velocity, a number of experiments were made. These are presented
and explained in \autoref{}. The data from these experiments have resulted in
the following initial velocity:\nl
\begin{equation}
v_0\ =\ 4.87\ m/s
\end{equation}

In addition to the initial velocity, the height of fireing needs to be
determined. As the height of the cannon barrel changes based on the angle of
fire, this can be calculated using trigonometry.

By using the value for $v_0$, and the equation for the height of fireing, the
projectiles trajectory can be determined.


% When fireing a projectile there is a number of informations which are needed in
% order to hit a target. This information consists of calculations and an
% understanding of the physics that go into fireing a projectile. This is
% necessary due to the fact that the canon has a limited amount of input that
% needs to be processed in order to hit the target.

% Based on this problem, this chapter will be used to explain the physics needed
% to fire the projectile. This will mainly be covered by presenting and explaining
% the equations used to calculate the projectile motion. These equations has been
% chosen du to tha fact that calculating the force and angle of the projectile is
% the most important part when fireing a projectile in order to hit a target.\nl
% 
% \section{Kasteparabel}
% For at kunne udregne et projektils kasteparabel, skal man have adgang til et
% antal informationer. Disse informationer er:
% 
% To be able to calculate the projectile motion, a number of informations i
% needed. These informations are:
% \begin{itemize}
%   \item Starting velocity $v_0$ ($m/s$)
%   \item Angle of fire $a$ ($degrees$)
%   \item Gravitational accelleration $g$ ($m/s^2$)
%   \item Height of fireing $y_0$ ($m$)
%   \item Distance $d$ ($m$)
%   \item Time in the air $t$ ($s$)
% \end{itemize}
% 
% If air drag is excluded, a number of statements can be made based on these
% informations:\nl
% When the projectile is fires it is not affected by any external forces other
% than the gravity. This means that the velocity can be calculated as:
% $v_x = v_0 * cos(a)$\nl
% 
% Dette udsagn passer også på\ldots \nl
% 
% Mere tekst\ldots \nl
% 
% Disse udregninger giver os så formlen:\nl
% $v_t = (\frac{v_0\ *\ cos(a)}{-g\ *\ t\ +\ v_0\ *\ sin(a)})$ 
% 
% \subsection{Diskussion: Luftmodstand}
% Denne subsection vil blive brugt til at diskutere hvorfor luftmodstanden kan
% ignoreres for den valgte type af projektil.
% 
