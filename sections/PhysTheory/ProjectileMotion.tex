\chapter{Projektil-fysik}
Ved affyring af et projektil er det et antal informationer som er nødvendige for
præcist at kunne ramme et mål. Disse informationer kommer i form af
matematiske udregninger og fysiske forståelser. Dette er nødvendigt, da den
producerede kanon har et begrænset udvalg af data til rådighed, som skal
behandles og anvendes til at være i stand til at ramme det udvalgte mål.\nl

På baggrund af denne problemstilling, vil dette kapitel vil blive brugt til, at
redegøre for de fysikfaglige elementer, der er relevante for affyringen af et
projektil. Dette vil hovedsageligt blive dækket ved at præsentere og forklare de
fysiske formler der har at gøre med kasteparabler. Disse formler er udvalgt,
da den vigtigste del af affyringen er, at udregne med hvilken kraft og vinkel
projektilet skal affyres for effektivt at ramme et givent mål.\nl

\section{Kasteparabel}
For at kunne udregne et projektils kasteparabel, skal man have adgang til et
antal informationer. Disse informationer er:
\begin{itemize}
  \item Starthastighed $v_0$ ($m/s$)
  \item Affyringsvinkel $a$ ($grader$)
  \item Tyngdeacceleationen $g$ ($m/s^2$)
  \item Affyringshøjde $y_0$ ($m$)
  \item Afstand $d$ ($m$)
  \item Flyvnings-tid $t$ ($s$)
\end{itemize}

Hvis man undtager luftmodstanden kan man ved hjælp af disse informationer
opstille en række udsagn:\nl
Efter projektilet er affyret vil det ikke være påvirket af andre eksterne
krafter en tyngdekraften. Derfor kan dens hastighed langs x-aksen skrives som:
$v_x = v_0 * cos(a)$\nl

Dette udsagn passer også på 

\subsection{Diskussion: Luftmodstand}
Denne subsection vil blive brugt til at diskutere hvorfor luftmodstanden kan
ignoreres for den valgte type af projektil.

