\section{Firing Offset}
As described in \autoref{InitialProblem} the cannon needs to calculate an
offset to be able to hit a target in motion. In order to calculate this we will
need the air time for the projectile and the velocity of the target\nl

The offset can be calculated using \autoref{FinEq} and \autoref{TimeEq} which
was derived in \autoref{ProjMotion}. By first using \autoref{FinEq} to
calculate the distance to the target, then inputting the result into
\autoref{TimeEq} to get the air time for the projectile. With these calculations
it is possible to determine the distance the target has travelled in the time it
takes the projectile to get to it, simply by multiplying the air time with the
velocity of the target. With this information it is then possible to use
Pythagoras theorem to calculate the length from the turret to the current
position of the target. All of this can be used together with the
\fix{cosine}{Is it sine or cosine} relation to calculate how much the turret
should turn to be able to hit the target. This topic is further explained during the implementation in
\autoref{ImplLabel}.

\section{Conclusion}
Based on the mathematical and physical models explained in this chapter, it
should be possible to predict the trajectory of the projectile when fired from
the turret. Based on this assumption, a test (\autoref{AppendixDistTest}) was
conducted in order to determine the accuracy of the formula. The test concluded
the following:
\begin{itemize}
  \item The muzzle speed of the projectile is 4.82m/s.
  \item Using the determined muzzle speed, \autoref{FinEq3} is unable to
  accurately calculate the distance of the projectile.
  \item An alternate approach can be used to determine the angle, namely mapping
  the test data into a formula using a trendline.  
\end{itemize}

Based on this tests conclusions, we have chosen to use the following formula for
determining which angle should be used to hit the target:

\begin{equation}\label{angleCalc}
Angle\ =\ abs(0.2573 * distance - 22.85)
\end{equation}


With all of this in mind, the following chapters will be used to present
and discuss hardware choices and turret designs.
