\section{Firing Offset}
As described in \autoref{InitialProblem} the cannon needs to calculate an
offset to be able to hit a target in motion. In order to calculate this we will
need the air time for the projectile and the velocity of the target\nl

The offset can be calculated using \autoref{FinEq} and \autoref{TimeEq} which
was derived in \autoref{ProjMotion}. By first using \autoref{FinEq} to
calculate the distance to the target, then inputting the result into
\autoref{TimeEq} to get the air time for the projectile. With these calculations
it is possible to determine the distance the target has travelled in the time it
takes the projectile to get to it, simply by multiplying the air time with the
velocity of the target. With this information it is then possible to use
pythagoras theorem to calculate the length from the turret to the current
position of the target. All of this can be used together with the cosine
relation to calculate how much the turret should turn to be able to hit the
target. This topic is further explained during the implementation in
\autoref{ImplLabel}.

\section{Conclusion}
Based on the mathematical and physical models explained in this chapter, it
should be possible to predict the trajectory of the projectile when fired from
the turret. With this in mind, the following chapters will be used to present
and discuss hardware choices and turret designs.
